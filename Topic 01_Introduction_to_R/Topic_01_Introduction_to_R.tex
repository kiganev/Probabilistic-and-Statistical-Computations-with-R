\documentclass[10pt]{beamer}
\usetheme{CambridgeUS}
%\usetheme{Boadilla}
\definecolor{myred}{RGB}{163,0,0}
%\usecolortheme[named=blue]{structure}
\usecolortheme{dove}
\usefonttheme[]{professionalfonts}
\usepackage[english]{babel}
\usepackage{amsmath,amsfonts,amssymb}
\usepackage{mathtools}
\usepackage{commath}
\usepackage{xcolor}
\usepackage{tikz}
\usepackage{pgfplots}
\pgfplotsset{compat=newest,compat/show suggested version=false}
\usetikzlibrary{arrows,shapes,calc,backgrounds}
\usepackage{bm}
\usepackage{textcomp}
%\usepackage{gensymb}
%\usepackage{verbatim}
\usepackage{mathrsfs}  
\usepackage{paratype}
\usepackage{mathpazo}
\usepackage{listings}
\usepackage{csvsimple}
\usepackage{booktabs}

\newcommand{\cc}[1]{\texttt{\textcolor{blue}{#1}}}

\DeclareMathOperator{\adj}{adj}
\DeclareMathOperator{\tr}{tr}
\DeclareMathOperator{\diag}{diag}
\DeclareMathOperator{\E}{\mathsf{E}}
\DeclareMathOperator{\var}{\mathsf{Var}}
\DeclareMathOperator{\cov}{\mathsf{Cov}}
\DeclareMathOperator{\corr}{\mathsf{Corr}}
\DeclareMathOperator{\plim}{plim}
\DeclareMathOperator{\rank}{rank}


\definecolor{ttcolor}{RGB}{0,0,1}%{RGB}{163,0,0}
\definecolor{mygray}{RGB}{248,249,250}

% Number theorem environments
\setbeamertemplate{theorem}[ams style]
\setbeamertemplate{theorems}[numbered]

% Reset theorem-like environments so that each is numbered separately
\usepackage{etoolbox}
\undef{\definition}
\theoremstyle{definition}
\newtheorem{definition}{\translate{Definition}}

% Change colours for theorem-like environments
\definecolor{mygreen1}{RGB}{0,96,0}
\definecolor{mygreen2}{RGB}{229,239,229}
\setbeamercolor{block title}{fg=white,bg=mygreen1}
\setbeamercolor{block body}{fg=black,bg=mygreen2}

\lstdefinestyle{numbers}{numbers=left, stepnumber=1, numberstyle=\tiny, numbersep=10pt}
\lstdefinestyle{MyFrame}{backgroundcolor=\color{yellow},frame=shadowbox}

\lstdefinestyle{rstyle}%
{language=R,
	basicstyle=\footnotesize\ttfamily,
	backgroundcolor = \color{mygray},
	commentstyle=\slshape\color{green!50!black},
	keywordstyle=\bfseries\color{blue!50!black},
	identifierstyle=\color{blue},
	stringstyle=\color{orange},
	%escapechar=\#,
	rulecolor = \color{mygray}, 
	showstringspaces = false,
	showtabs = false,
	tabsize = 2,
	emphstyle=\color{red},
	frame = single}

\lstset{language=R,frame=single}    

\hypersetup{colorlinks, urlcolor=blue, linkcolor = myred}

\AtBeginSection{\frame{\sectionpage}}

% Remove Section 1, Section 2, etc. as titles in section pages
\defbeamertemplate{section page}{mine}[1][]{%
	\begin{centering}
		{\usebeamerfont{section name}\usebeamercolor[fg]{section name}#1}
		\vskip1em\par
		\begin{beamercolorbox}[sep=12pt,center]{part title}
			\usebeamerfont{section title}\insertsection\par
		\end{beamercolorbox}
	\end{centering}
} 

\setbeamertemplate{section page}[mine] 

\beamertemplatenavigationsymbolsempty

\title{R403: Probabilistic and Statistical Computations\\ with R}
\subtitle{Topic 1: \textcolor{myred}{Introduction to R. Common Data Types and Structures}}
\author{Kaloyan Ganev}
\date{2022/2023}

\begin{document}
\frame{\titlepage}

\begin{frame}
\tableofcontents[hideallsubsections]
\end{frame}

\hypertarget{introductory-notes}{%
\section{Introductory Notes}\label{introductory-notes}}

\begin{frame}[fragile]
\frametitle{History of R}
\begin{itemize}
	\item R is a dialect of the S language
	
	\item S was developed in the 1970s in the Bell Labs

	\item Initially written in Fortran, rewritten in C in the late 1980s

	\item Currently available through a commercial product called S-PLUS

	\item R was developed in 1991 by Ross Ihaka and Robert Gentleman (University of Auckland, New Zealand)
	
	\item First public announcement: 1993
	
	\item GNU GPL: 1995
	
	\item Version 1.0.0: 2000
\end{itemize}
\end{frame}

\begin{frame}[fragile]
\frametitle{How to Get the Software}
\begin{itemize}
	\item Available for free from \url{https://www.r-project.org/}
	
	\item All major OS's supported
	
	\item Lots of contributed packages available from CRAN mirrors

	\item Non-CRAN packages will most probably not be used throughout the course
	
	\item Check out documentation links on the R Project site for what R and additional packages do
\end{itemize}
\end{frame}



\begin{frame}[fragile]
\frametitle{GUIs}
\begin{itemize}
	\item Although R is command-line driven, there are some GUIs that provide more convenience in working with it
	
	\item Some examples:
	\begin{itemize}
		\item R Commander (\url{http://socserv.mcmaster.ca/jfox/Misc/Rcmdr/})

		\item Deducer (\url{http://www.deducer.org/pmwiki/index.php?n=Main.DeducerManual?from=Main.HomePage})

		\item JGR (\url{http://www.rforge.net/JGR/})
		
		\item RKWard (\url{https://rkward.kde.org/})
		
		\item \textcolor{red}{RStudio} (\url{https://www.rstudio.com/})
		
		\item Jamovi (\url{https://www.jamovi.org})
		
		\item BlueSky Statistics (\url{https://www.blueskystatistics.com/})
	\end{itemize}
\end{itemize}
\end{frame}

\begin{frame}[fragile]
\frametitle{Advantages of R}
\begin{itemize}
	\item Free and open-source

	\item Flexible: you can do whatever you want, no pre-programmed black boxes
	
	\item Pre-built packages so that you don't have to reinvent the wheel
	
	\item Enormous graphing capabilities, publication-quality graphical output
	
	\item Active world-wide community (check out for example \href{https://stackoverflow.com/}{Stack Overflow}, \href{https://stackexchange.com/}{Stack Exchange}, etc.)
	
	\item Huge amounts of online resources (manuals, videos, hints, etc.), many of them freely available

	\item etc.
\end{itemize}
\end{frame}

\begin{frame}[fragile]
\frametitle{RStudio: a Quick Overview}
\begin{figure}
	\centering
	\includegraphics[scale=0.23]{rstudio.png}
\end{figure}
\end{frame}

\section{Basic Capabilities of R}

\begin{frame}[fragile]{Before We Begin\ldots}
\begin{itemize}
	\item Set your working directory: this is where R reads and writes files (if you don't instruct it otherwise explicitly)
	
	\item By default, the working directory is your home directory; type to see:
	\begin{lstlisting}[style = rstyle, breaklines]
	getwd()
	\end{lstlisting}

	\item To permanently change the working directory, in RStudio go to \texttt{Tools -> Global Options\ldots} and there click Browse next to \texttt{Default working directory}
	
	\item If you would like to change it on a case-by-case basis, use:
	\begin{lstlisting}[style = rstyle, breaklines]
	setwd(<path to directory>)
	\end{lstlisting}
\end{itemize}
\end{frame}

\begin{frame}[fragile]
\frametitle{The RStudio Console}
\begin{itemize}
	\item This is where commands are typed and issued to R

	\item Commands are written and executed one at a time
	
	\item A simple command:
	\begin{lstlisting}[style = rstyle, breaklines]
	dir()
	\end{lstlisting}
	
	to list the contents of your current working directory
\end{itemize}
\end{frame}

\begin{frame}[fragile]
\frametitle{The RStudio Console (2)}
\begin{itemize}
	\item R as a calculator:
	\begin{lstlisting}[style = rstyle, breaklines]
	5 + 6
	3 * 4
	12 / 7
	2 ^ 5
	11 %% 5
	1 / Inf
	0 / 0
	\end{lstlisting}

	\item Doing this, however, you immediately lose the result (cannot use it later)
\end{itemize}
\end{frame}

\begin{frame}{Objects and Object Types}
\begin{itemize}
	\item R objects: containers (memory locations, to be fully correct) to keep information for later usage
	
	\item Basic types of objects:
	\begin{itemize}
		\item Vectors
		
		\item Matrices

		\item Lists

		\item Arrays
		
		\item Factors
		
		\item Data frames
	\end{itemize}
\end{itemize}
\end{frame}

\begin{frame}[fragile]
\frametitle{Assigning Values to Objects}
\begin{itemize}
	\item Assignment: putting something in the container; examples:
	\begin{lstlisting}[style = rstyle, breaklines]
	x1 <- 10 # numeric
	x2 <-5L # integer
	x3 <- 6 + 2i # complex
	y <- FALSE # logical
	z <- "Hello World!" # character
	\end{lstlisting}

	\item Print objects: type name or type
	\begin{lstlisting}[style = rstyle, breaklines]
	print(<object name>)
	\end{lstlisting}
\end{itemize}
\end{frame}

\begin{frame}[fragile]
\frametitle{Object Attributes}
\begin{itemize}
	\item Use
	\begin{lstlisting}[style = rstyle, breaklines]
	attributes(<object name>)
	\end{lstlisting}
	to explore; for example:
	
	\begin{lstlisting}[style = rstyle, breaklines]
	z1 <- rnorm(625)
	z2 <- matrix(z1, nrow = 25)
	attributes(z2)
	\end{lstlisting}

	\item Attributes can include names, length, dimensions, dimnames, class, etc.
\end{itemize}

\end{frame}

\section{Basic Data Types}
\begin{frame}
\frametitle{Basic Data Types}
\begin{itemize}
	\item You saw them already on the previous slides:
	\begin{itemize}
		\item Numeric

		\item Integer

		\item Complex

		\item Logical
		
		\item Character
	\end{itemize}

	\item Those are also called \textit{atomic data types}
	
	\item Such ``atoms'' are used to construct various objects and data structures in R
\end{itemize}
\end{frame}

\begin{frame}[fragile]{Basic Data Types (2)}
\begin{itemize}
	\item To get the type of an object:
	\begin{lstlisting}[style = rstyle, breaklines]
	mode(<name of object>) # low level
	class(<name of object>) # high level
	\end{lstlisting}

	\item To check whether an object is of a specific type, e.g.:
	\begin{lstlisting}[style = rstyle, breaklines]
	is.numeric(x) # returns TRUE or FALSE
	\end{lstlisting}
\end{itemize}
\end{frame}

\begin{frame}[fragile]
\frametitle{Type Coercion}
\begin{itemize}
	\item Means forceful transforming one data type into another
	
	\item Check out:
	\begin{lstlisting}[style = rstyle, breaklines]
	as.numeric(FALSE) # Try also with TRUE
	as.character(102)
	as.integer(105.88)
	as.complex(3.14)
	as.numeric("458")
	as.numeric("This is text.")
	\end{lstlisting}

	\item Same logic followed for all other objects, too
\end{itemize}
\end{frame}



\begin{frame}[fragile]{The Workspace}
\begin{itemize}
	\item Once created, an object is placed in the workspace
	
	\item It stays there until removed, e.g.~with the \texttt{rm()} command
	
	\item Can be seen in the top-right tab labelled ``Environment''
	
	\item Can also be listed by using the command line:
	\begin{lstlisting}[style = rstyle, breaklines]
	ls()
	\end{lstlisting}
	
	\item You can set various options for working with R, e.g.~the number of digits to display after the decimal sign:
	\begin{lstlisting}[style = rstyle, breaklines]
	options(digits = 2)
	\end{lstlisting}
	
	\item Check out also:
	\begin{lstlisting}[style = rstyle, breaklines]
	help(options)
	\end{lstlisting}
\end{itemize}
\end{frame}


\begin{frame}[fragile]{R Scripts}
\begin{itemize}
	\item In essence, text files to hold R commands

	\item Saved with the .R extension
	
	\item Provide convenience and reproducibility
	
	\item In RStudio, they can be executed line by line, in blocks, or entirely

	\item To include an existing script into a new script:
	\begin{lstlisting}[style = rstyle, breaklines]
	source(<script name>)
	\end{lstlisting}
\end{itemize}
\end{frame}

\section{Vectors, matrices and arrays}
\subsection{Vectors}

\begin{frame}[fragile]
\frametitle{Vectors in R}
\begin{itemize}
	\item Unlike in mathematics where vectors are strictly comprised of numbers, in R vectors are ordered collections of data elements
	
	\item In other words, you can have a numeric vector, an integer vector, a character vector, etc.
	
	\item Created with the \texttt{c()} command:
	\begin{lstlisting}[style = rstyle, breaklines]
	x1 <- c(1, 2, 3, 4)
	x2 <- seq(1, 10)
	x3 <- seq(from = 2, to = 3, by = 0.01)
	y <- c("One", "Two", "3")
	z <- c(TRUE, FALSE, T, F)		
	\end{lstlisting}
\end{itemize}
\end{frame}

\begin{frame}[fragile]
\frametitle{Vectors in R (2)}
\begin{itemize}
	\item Note that R treats single elements also as vectors!
	\begin{lstlisting}[style = rstyle, breaklines]
	is.vector(x3)
	is.vector(3)
	is.vector("text")
	\end{lstlisting}
	
	\item What is specific is that a vector can contain only one basic type of data
	
	\item For example, a vector can hold only complex numbers; every other number is converted (coerced) to complex
	\begin{lstlisting}[style = rstyle, breaklines]
	y <- c(1.02, 3 + 0.5i) # Check out the next line, too}
	y <- c(1.02, 3 +0.5i, "text")
	\end{lstlisting}
\end{itemize}
\end{frame}

\begin{frame}[fragile]
\frametitle{Vectors in R (3)}
\begin{itemize}
	\item Vectors in real-life applications contain information on something
	
	\item In order to not forget what info is contained in each element, elements can be named in R
	\begin{lstlisting}[style = rstyle, breaklines]
	lunch_costs <- c(3.30, 5.91, 2.75)
	names(lunch_costs) <- c("Soup", "Main course", "Dessert")
	print(lunch_costs)
	str(lunch_costs)
	\end{lstlisting}
\end{itemize}
\end{frame}

\begin{frame}[fragile]{Creating Empty Vectors}
\begin{itemize}
	\item One way to create is with the \texttt{vector()} function
	
	\item You need to specify type (numeric, logical, character, integer, double):
	\begin{lstlisting}[style = rstyle, breaklines]
	em1 <- vector("numeric")
	em2 <- vector("character")
	em3 <- vector("logical")
	\end{lstlisting}

	\item Note that some default values are imputed if you additionally specify length as an argument
	\begin{lstlisting}[style = rstyle, breaklines]
	nem1 <- vector("numeric", length = 20)
	\end{lstlisting}

	\item You can get empty vectors also through:
	\begin{lstlisting}[style = rstyle, breaklines]
		em1a <- numeric()
		em2a <- character()
		em3a <- logical()
	\end{lstlisting}
\end{itemize}
\end{frame}


\begin{frame}[fragile]{Mathematical Operations with Vectors}
\begin{itemize}
	\item All operations are carried out element-wise; check out:
	\begin{lstlisting}[style = rstyle, breaklines]
	v1 <- c(1, 2, 3)
	v2 <- c(4, 5, 6)
	v1 + 3.5
	v1 * 2
	v1 / 3
	v1 - v2
	v1 * v2
	v1 ^ v2
	\end{lstlisting}
\end{itemize}
\end{frame}

\begin{frame}[fragile]{Mathematical Operations with Vectors (2)}
\begin{itemize}
	\item What if vectors are of different sizes? Recycling
	\begin{lstlisting}[style = rstyle, breaklines]
	v3 <- c(7, 8, 9, 10)
	v4 <- v1 + v3
	v5 <- v1 * v3
	\end{lstlisting}

	\item Safest strategy: avoid recycling if possible
\end{itemize}
\end{frame}

\begin{frame}[fragile]
\frametitle{Element Sums and Vector Comparison}
\begin{itemize}
	\item If you need to find the sum of all elements in a vector:
	\begin{lstlisting}[style = rstyle, breaklines]
	total_lunch <- sum(lunch_costs)
	total_lunch	
	\end{lstlisting}

	\item Compare with another vector:
	\begin{lstlisting}[style = rstyle, breaklines]
	lunch_costs_alt <- c(3.10, 6.25, 2.90)
	lunch_costs > lunch_costs_alt
	\end{lstlisting}
\end{itemize}
\end{frame}

\subsection{Factors}
\begin{frame}[fragile]
	\frametitle{Factors}
	\begin{itemize}
		\item Statistical models work with two types of variables: quantitative and qualitative
		
		\item Quantitative variables are expressed in numbers in a straightforward manner
		
		\item In general such variables are continuous range and can therefore have infinitely many values\footnote{Sometimes, their value range can be though restricted.}
		
		\item Qualitative variables, at the same time, usually can take on only a limited number of values
		
		\item Those values are usually non-numerical, and are known as \textit{categories}\footnote{Therefore quality variables are often mentioned as \textit{categorical variables}.}
		
		\item Since computers understand only numbers, categorical values need to be converted to numbers (usually non-negative digits)
	\end{itemize}
\end{frame}

\begin{frame}[fragile]
	\frametitle{Factors (2)}
	\begin{itemize}
		\item In R, categorical variables are implemented through the usage of \textit{factors}
		
		\item Each possible value (category) of a factor is known as a \textit{level}
		
		\item Levels are stored as integers
		
		\item Factors can be \textit{unordered} or \textit{ordered}
		
		\item Ordering could be imposed by the nature of the data (e.g. weekdays, months, temperature, etc.)
	\end{itemize}
\end{frame}

\begin{frame}[fragile]
	\frametitle{Factors (3)}
	\begin{itemize}
		\item In the following example, a qualitative variable describing the hair colour of a sample of people is created:
		\begin{lstlisting}[style = rstyle, breaklines]
		hair_col <- c("black", "brown", "black", "blond", "red", "brown", "black")
		hair_col_f <- factor(hair_col)
		hair_col_f
		\end{lstlisting}
	
		\item This is an unordered factor: it does not matter which colour is put first, second, etc.
	\end{itemize}
\end{frame}

\begin{frame}[fragile]
	\frametitle{Factors (4)}
	\begin{itemize}
		\item To see how an ordered one can be created, see the next example:
		\begin{lstlisting}[style = rstyle, breaklines]
		temp <- c("freezing", "warm", "hot", "cold", "warm", "freezing")
		temp_f <- factor(temp, ordered = TRUE, levels = c("freezing", "cold", "warm", "hot"))
		temp_f
		\end{lstlisting}
	
		\item The way categories are displayed can be changed, too, by changing the labels of categories:
		\begin{lstlisting}[style = rstyle, breaklines]
		temp_f[2] < temp_f[4] # Comparison possible
		temp_f <- factor(temp, ordered = TRUE, levels = c("freezing", "cold", "warm", "hot"),
		labels = c("very cold", "cold", "warm", "hot")) # labels changed here
		temp_f 
		\end{lstlisting}
	\end{itemize}
\end{frame}

\subsection{Matrices}
\begin{frame}[fragile]
\frametitle{Matrices in R}
\begin{itemize}
	\item Matrices are an extension of vectors to two dimensions

	\item All other properties are maintained, incl. single basic data type and automatic coercion
	
	\item One way to create:
	\begin{lstlisting}[style = rstyle, breaklines]
	m1 <- matrix(1:12, nrow = 3) # or,
	m1 <- matrix(1:12, ncol = 4)
	\end{lstlisting}

	\item Note that elements are filled column by column by default; if you want it by rows:
	\begin{lstlisting}[style = rstyle, breaklines]
	m2 <- matrix(1:12, ncol = 4, byrow = TRUE)
	\end{lstlisting}
\end{itemize}
\end{frame}



\begin{frame}[fragile]{Matrices in R (2)}
\begin{itemize}
	\item Another way to create:
	\begin{lstlisting}[style = rstyle, breaklines]
	m3 <- cbind(lunch_costs, lunch_costs_alt)	
	\end{lstlisting}
	
	\item Change column names:
	\begin{lstlisting}[style = rstyle, breaklines]
	colnames(m3) <- c("Costs, option 1", "Costs, option 2")
	\end{lstlisting}

	\item Add a row of totals:
	\begin{lstlisting}[style = rstyle, breaklines]
	m3a <- rbind(m3, colSums(m3))
	rownames(m3a)[4] <- "Total"
	\end{lstlisting}

	\item  \texttt{rowSums()} is analogical to \texttt{colSums()} and also self-explanatory
\end{itemize}
\end{frame}

\begin{frame}[fragile]
\frametitle{Subsets (Slices) of Matrices}
\begin{itemize}
	\item Analogical to vectors, only one more dimension is added:
	\begin{lstlisting}[style = rstyle, breaklines]
	m3a[,1]
	m3a[2,]
	m3a[c(1, 3), 1]
	m3a[4, 2]
	m3a["Soup", "Costs, option 2"]
	m3a[c(T, F, F, F), c(T,F)]
	\end{lstlisting}
\end{itemize}
\end{frame}

\begin{frame}[fragile]{Mathematical Operations with Matrices}
\begin{itemize}
	\item Again, analogical to operations with vectors

	\item Check out:
	\begin{lstlisting}[style = rstyle, breaklines]
	m1 + 5
	m1 * 3
	m1 %% 3
	m1 ^ 2	
	\end{lstlisting}
	
	\item Two or more matrices: element-wise!
	\begin{lstlisting}[style = rstyle, breaklines]
	m1 + m2
	m1 / m2
	m1 %% m2
	m1 ^ m2 
	\end{lstlisting}
\end{itemize}
\end{frame}

\begin{frame}[fragile]
\frametitle{`Traditional' Matrix Algebra in R}
\begin{itemize}
	\item Transposition:
	\begin{lstlisting}[style = rstyle, breaklines]
	t(m1)
	\end{lstlisting}

	\item Multiplication (possible if matrices are conformable):
	\begin{lstlisting}[style = rstyle, breaklines]
	t(m1) %*% m2 # equivalent to
	crossprod(m1,m2)
	\end{lstlisting}
\end{itemize}
\end{frame}

\begin{frame}[fragile]
\frametitle{`Traditional' Matrix Algebra in R (2)}
\begin{itemize}
	\item Inverse matrix (if matrix square and non-singular):
	\begin{lstlisting}[style = rstyle, breaklines]
	m4 <- matrix(1:4, nrow = 2)
	solve(m4)
	\end{lstlisting}

	\item Eigenvalues and eigenvectors:
	\begin{lstlisting}[style = rstyle, breaklines]
	eig <- eigen(m4)
	eig$val
	eig$vec
	\end{lstlisting}
\end{itemize}
\end{frame}

\subsection{Arrays}

\begin{frame}[fragile]
\frametitle{Arrays in R}
\begin{itemize}
	\item Vectors are one-dimensional arrays
	
	\item Matrices are two dimensional arrays
	
	\item Higher dimensions are also possible
	\item A 3D array:
	\begin{lstlisting}[style = rstyle, breaklines]
	array1 <- array(1:343, dim = c(7, 7, 7))
	\end{lstlisting}
	
	\item You can name dimensions using \texttt{dimnames()}
\end{itemize}
\end{frame}


\section{Lists and Data Frames}

\begin{frame}[fragile]
\frametitle{Lists in R}

\begin{itemize}
	\item Vectors, matrices and arrays can hold information that is only of one type

	\item When it is necessary to have more than one data type, lists are used

	\item Created with the \texttt{list()} command
	
	\item Example:
	\begin{lstlisting}[style = rstyle, breaklines]
	list1 <- list(TRUE, "France", 21)
	\end{lstlisting}
\end{itemize}
\end{frame}

\begin{frame}[fragile]
\frametitle{Lists in R (2)}
\begin{itemize}
	\item Can be named
	\begin{lstlisting}[style = rstyle, breaklines]
	names(list1) <- c("Active", "Country", "Age")
	\end{lstlisting}

	\item Can also hold other complex structures such as vectors, matrices, etc., even other lists
	\begin{lstlisting}[style = rstyle, breaklines]
	list2 <- list(Appearances = c(T,F,T,T,T), Attributes = list1)
	\end{lstlisting}
\end{itemize}
\end{frame}

\begin{frame}[fragile]
\frametitle{Subsetting Lists}
\begin{itemize}
	\item Single square brackets return a list
	\begin{lstlisting}[style = rstyle, breaklines]
	list3 <- list2[1]
	\end{lstlisting}
	
	\item Double square brackets extract the generic element; in the current  example: a vector:
	\begin{lstlisting}[style = rstyle, breaklines]
	vec1 <- list2[[1]]
	\end{lstlisting}
\end{itemize}
\end{frame}

\begin{frame}[fragile]
\frametitle{Subsetting Lists (2)}
\begin{itemize}
	\item The same is achieved with the \$ sign:
	\begin{lstlisting}[style = rstyle, breaklines]
	vec2 <- list2$Appearances
	\end{lstlisting}

	\item The resulting vectors (or other types) can be further indexed to get their elements
	\begin{lstlisting}[style = rstyle, breaklines]
	list2[[1]][1]
	list2$Appearances[1]
	\end{lstlisting}
\end{itemize}
\end{frame}

\begin{frame}[fragile]
\frametitle{Subsetting lists (3)}
\begin{itemize}
	\item Subsetting is also possible by name:
	\begin{lstlisting}[style = rstyle, breaklines]
	list2["Attributes"]
	\end{lstlisting}

	\item Also, by logical values:
	\begin{lstlisting}[style = rstyle, breaklines]
	list1[c(FALSE, TRUE, FALSE)]
	\end{lstlisting}
\end{itemize}
\end{frame}

\begin{frame}[fragile]
\frametitle{Extending Lists}
\begin{itemize}
	\item Easiest done by means of the \$ sign
	\begin{lstlisting}[style = rstyle, breaklines]
	teams <- c("Real Madrid", "Barcelona", "Valencia")
	list2$clubs <- teams
	\end{lstlisting}

	\item Using single and double square brackets also possible although somewhat more difficult
\end{itemize}

\end{frame}

\begin{frame}
\frametitle{Data Frames}
\begin{itemize}
	\item A special type of list
	
	\item Contains vectors of the same length (same type not required)
	
	\item Correspond to the notion of datasets in other statistical packages
	
	\item Variables correspond to columns, while observations correspond to rows
\end{itemize}
\end{frame}

\begin{frame}[fragile]
\frametitle{Data Frames (2)}
\begin{itemize}
	\item Created from other objects, e.g.~matrices:
	\begin{lstlisting}[style = rstyle, breaklines]
	m4df <- matrix(1:9, nrow = 3)
	df1 <- as.data.frame(m4df)
	\end{lstlisting}

	\item Name variables:
	\begin{lstlisting}[style = rstyle, breaklines]
	names(df1) <- c("One", "Two", "Three")
	\end{lstlisting}

	\item Subsetting uses both vector and list approaches
\end{itemize}
\end{frame}

\begin{frame}[fragile]
\frametitle{Extending Data Frames}
\begin{itemize}
	\item Columns (new variables) and rows (new observations) can be added
	
	\item For example:
	\begin{lstlisting}[style = rstyle, breaklines]
	df1$Four <- c(10, 11, 12)
	totals <- data.frame(matrix(c(6, 15, 24, 33), 
		nrow = 1, ncol = 4))
	names(totals) <- c("One", "Two", "Three", "Four")
	df2 <- rbind(df1, totals)
	\end{lstlisting}
	
	\item Data frames can also be directly read from external sources (text, csv, databases, etc.)
	
	\item This is also the most common data structure used in statistical applications
\end{itemize}
\end{frame}

\end{document}

\documentclass[10pt]{beamer}
\usetheme{CambridgeUS}
%\usetheme{Boadilla}
\definecolor{myred}{RGB}{163,0,0}
%\usecolortheme[named=blue]{structure}
\usecolortheme{dove}
\usefonttheme[]{professionalfonts}
\usepackage[english]{babel}
\usepackage{amsmath,amsfonts,amssymb}
\usepackage{mathtools}
\usepackage{commath}
\usepackage{xcolor}
\usepackage{tikz}
\usepackage{pgfplots}
\pgfplotsset{compat=newest,compat/show suggested version=false}
\usetikzlibrary{arrows,shapes,calc,backgrounds}
\usepackage{bm}
\usepackage{textcomp}
%\usepackage{gensymb}
%\usepackage{verbatim}
\usepackage{mathrsfs}  
\usepackage{paratype}
\usepackage{mathpazo}
\usepackage{listings}
\usepackage{csvsimple}
\usepackage{booktabs}
\usepackage{longtable}

\newcommand{\cc}[1]{\texttt{\textcolor{blue}{#1}}}

\DeclareMathOperator{\adj}{adj}
\DeclareMathOperator{\tr}{tr}
\DeclareMathOperator{\diag}{diag}
\DeclareMathOperator{\E}{\mathsf{E}}
\DeclareMathOperator{\var}{\mathsf{Var}}
\DeclareMathOperator{\cov}{\mathsf{Cov}}
\DeclareMathOperator{\corr}{\mathsf{Corr}}
\DeclareMathOperator{\plim}{plim}
\DeclareMathOperator{\rank}{rank}


\definecolor{ttcolor}{RGB}{0,0,1}%{RGB}{163,0,0}
\definecolor{mygray}{RGB}{248,249,250}

% Number theorem environments
\setbeamertemplate{theorem}[ams style]
\setbeamertemplate{theorems}[numbered]

% Reset theorem-like environments so that each is numbered separately
\usepackage{etoolbox}
\undef{\definition}
\theoremstyle{definition}
\newtheorem{definition}{\translate{Definition}}

% Change colours for theorem-like environments
\definecolor{mygreen1}{RGB}{0,96,0}
\definecolor{mygreen2}{RGB}{229,239,229}
\setbeamercolor{block title}{fg=white,bg=mygreen1}
\setbeamercolor{block body}{fg=black,bg=mygreen2}

\lstdefinestyle{numbers}{numbers=left, stepnumber=1, numberstyle=\tiny, numbersep=10pt}
\lstdefinestyle{MyFrame}{backgroundcolor=\color{yellow},frame=shadowbox}

\lstdefinestyle{rstyle}%
{language=R,
	basicstyle=\footnotesize\ttfamily,
	backgroundcolor = \color{mygray},
	commentstyle=\slshape\color{green!50!black},
	keywordstyle=\color{blue},
	identifierstyle=\color{blue},
	stringstyle=\color{orange},
	%escapechar=\#,
	rulecolor = \color{mygray}, 
	showstringspaces = false,
	showtabs = false,
	tabsize = 2,
	emphstyle=\color{red},
	frame = single}

\lstset{language=R,frame=single}    

\hypersetup{colorlinks, urlcolor=blue, linkcolor = myred}

\AtBeginSection{\frame{\sectionpage}}

% Remove Section 1, Section 2, etc. as titles in section pages
\defbeamertemplate{section page}{mine}[1][]{%
	\begin{centering}
		{\usebeamerfont{section name}\usebeamercolor[fg]{section name}#1}
		\vskip1em\par
		\begin{beamercolorbox}[sep=12pt,center]{part title}
			\usebeamerfont{section title}\insertsection\par
		\end{beamercolorbox}
	\end{centering}
} 

\setbeamertemplate{section page}[mine] 

\beamertemplatenavigationsymbolsempty 

\title{R403: Probabilistic and Statistical Computations\\ with R}
\subtitle{Topic 12: \textcolor{myred}{Time Series Manipulation in R}}
\author{Kaloyan Ganev}

\date{2022/2023} 

\begin{document}
\maketitle

\begin{frame}[fragile]
\frametitle{Lecture Contents}
\tableofcontents
\end{frame}

\section{Introduction}
\begin{frame}[fragile]
\frametitle{What is a time series?}
\begin{itemize}
	\item An informal definition:
	\begin{definition}	
		A time series is a time-ordered series of data observations on one or more variables.
	\end{definition}
	\item Although it is brief, it suggests a lot of insight
	\item First, each observation belongs to a specified point in time and to no other; time values index the observations
	\item The number of observations cannot be indefinitely large, therefore we have discrete data
	\item Usually measurements are taken at equal intervals but there are exceptions to this rule
\end{itemize}
\end{frame}

\begin{frame}[fragile]
\frametitle{R's capabilities of working with time series}
\begin{itemize}
	\item Quite large and extensive
	\item Some of them shipped with the base version, others available through contributed packages
	\item We will start with the first type and then we will discuss some of the most frequently use ones of the second type
	\item Do not forget to pay a visit to the CRAN Task View of packages for a more comprehensive list
\end{itemize}
\end{frame}

\section{Base R time series capabilities}
\begin{frame}[fragile]
\frametitle{The \textbf{ts} class}
\begin{itemize}
	\item Represents regularly spaced time series
	\item Uses numeric time stamps
	\item Obtained through the conversion of numeric vectors
	\item Conversion is achieved by means of the \cc{ts()} command available is the automatically loaded \textbf{stats} package
	\item General construct:
	\begin{lstlisting}[style = rstyle, breaklines]
	ts(x, start=, end=, frequency=)
	\end{lstlisting}
	where \cc{x} is the input vector; the others are more or less self explanatory
\end{itemize}
\end{frame}

\begin{frame}[fragile]
\frametitle{An example of creating time series with \cc{ts()}}
\begin{itemize}
	\item We will use the \textbf{eurostat} package to download some data 
	\item Data are on GDP and its components for all EU Member States
	\item We choose to work with quarterly frequencies
	\item Load the package, alongside with \textbf{tidyverse} which will be used for data processing:
	
	\begin{lstlisting}[style = rstyle, breaklines]
	library(eurostat)
	library(tidyverse)	
	\end{lstlisting}
	
	\item First, we look for all datasets containing the string ``GDP'':

	\begin{lstlisting}[style = rstyle, breaklines]
	search1 <- search_eurostat("GDP", type = "dataset")
	\end{lstlisting}
	
	\item Then we download data on quarterly GDP and its components
	
	\begin{lstlisting}[style = rstyle, breaklines]
	data_gdpq <- get_eurostat("namq_10_gdp", time_format = "date")	
	\end{lstlisting}
\end{itemize}
\end{frame}

\begin{frame}[fragile]
\frametitle{An example of creating time series with \cc{ts()} (2)}
\begin{itemize}
	\item Using the functions of the \textbf{dplyr} package, we filter data on Bulgarian GDP and consumption only
	\item The values we choose are at 2010 prices, not seasonally adjusted
	\begin{lstlisting}[style = rstyle, breaklines]
	data_gdpq_bg <- data_gdpq %>%
	  filter(geo == "BG", 
         unit == "CLV10_MEUR",
         s_adj == "NSA",
         na_item %in% c("B1GQ", "P3")) %>%
	  select(time, values, na_item) %>%
	  spread(na_item, values)
	\end{lstlisting}
\end{itemize}
\end{frame}

\begin{frame}[fragile]
\frametitle{An example of creating time series with \cc{ts()} (3)}
\begin{itemize}
	\item Create names for the time series from the names of the variables in the data frame
	
	\begin{lstlisting}[style = rstyle, breaklines]
	names_ts <- character()

	for(i in names(data_gdpq_bg)[2:3]){
	  	names_ts[i] <- paste(i, ".ts", sep = "")
		}

	names_ts
	\end{lstlisting}

	\item Create the time series by assigning variables from the data frame to each name in the character vector:
	
	\begin{lstlisting}[style = rstyle, breaklines]
	for(i in 1:length(names_ts)){
  		assign(names_ts[i],ts(data_gdpq_bg[[i]], start = c(1995,1), frequency = 4))
	}	
	\end{lstlisting}
\end{itemize}
\end{frame}

\begin{frame}[fragile]
\frametitle{Properties of \textbf{ts} objects}
\begin{itemize}
	\item You can view any \textbf{ts} object by just printing it in the console, e.g.:+
	
	\begin{lstlisting}[style = rstyle, breaklines]
	B1GQ.ts
	\end{lstlisting}
	
	\item Check the class:
	
	\begin{lstlisting}[style = rstyle, breaklines]
	class(B1GQ.ts)
	\end{lstlisting}
	
	\item The start date, the end date, and the frequency of a time series can be viewed by means of:
	
	\begin{lstlisting}[style = rstyle, breaklines]
	start(B1GQ.ts)
	end(B1GQ.ts)
	frequency(B1GQ.ts)
	\end{lstlisting}
	
	\item You can also check the time interval between two consecutive observations:
	
	\begin{lstlisting}[style = rstyle, breaklines]
	deltat(B1GQ.ts)
	\end{lstlisting}
\end{itemize}
\end{frame}

\begin{frame}[fragile]
\frametitle{Subsetting \textbf{ts} objects}
\begin{itemize}
	\item The usual approaches used for vectors and lists are not appropriate
	\item They will not produce objects of the \textbf{ts} type
	\item Instead, the \cc{window()} command is used, e.g.:
	
	\begin{lstlisting}[style = rstyle, breaklines]
	B1GQ_smpl.ts <- window(B1GQ.ts, start = c(2010, 1), end = c(2019,2))
	
	B1GQ_smpl.ts
	\end{lstlisting}
	
	\item Check the sample class:
	\begin{lstlisting}[style = rstyle, breaklines]
	class(B1GQ_smpl.ts)
	\end{lstlisting}
\end{itemize}
\end{frame}

\begin{frame}[fragile]
\frametitle{Manipulating \textbf{ts} objects}
\begin{itemize}
	\item \textbf{ts} objects can be generated directly from a data frame:
	
	\begin{lstlisting}[style = rstyle, breaklines]
	mts1 <- ts(data_gdpq_bg[,2:3], start = c(1995, 1), frequency = 4)
	class(mts1)
	\end{lstlisting}	

	\item Individual \textbf{ts} objects can be combined using \cc{cbind()} or \cc{ts.union()}

	\item  Examples:
	
	\begin{lstlisting}[style = rstyle, breaklines]
	ts_comb1 <- cbind(B1GQ.ts, P3.ts)
	ts_comb2 <- ts.union(B1GQ.ts, P3.ts)
	\end{lstlisting}
	
	\item Check class:
	\begin{lstlisting}[style = rstyle, breaklines]
	class(ts_comb1)
	class(ts_comb2)
	\end{lstlisting}
\end{itemize}
\end{frame}

\begin{frame}[fragile]
\frametitle{Manipulating \textbf{ts} objects (2)}
\begin{itemize}
	\item Lagged series (note we are using the function from the \textbf{stats} package):
	
	\begin{lstlisting}[style = rstyle, breaklines]
	B1GQ_lag1.ts <- stats::lag(B1GQ.ts,-1)
	B1GQ_lag1.ts
	start(B1GQ_lag1.ts)
	\end{lstlisting}
	
	\item The default value is 1, however it shifts all observations one period backwards
	
	\item Using -1 shifts observations forward, thus matching current time with one-period lag value

	\item Get the series and its first lag together
	\begin{lstlisting}[style = rstyle, breaklines]
	mts2 <- ts.union(B1GQ.ts, B1GQ_lag1.ts)
	\end{lstlisting}
	
	\item Get the common sample data for two or more series:
	
	\begin{lstlisting}[style = rstyle, breaklines]
	mts3 <- ts.intersect(B1GQ.ts, B1GQ_lag1.ts)
	\end{lstlisting}
\end{itemize}
\end{frame}

\begin{frame}[fragile]
\frametitle{Manipulating \textbf{ts} objects (3)}
\begin{itemize}
	\item Differences of series:
	
	\begin{lstlisting}[style = rstyle, breaklines]
	diff(B1GQ.ts)
	\end{lstlisting}
	
	\item This will give you the first difference of the series
	
	\item You can specify a higher differencing order:
	
	\begin{lstlisting}[style = rstyle, breaklines]
	diff(B1GQ.ts, differences = 2)
	\end{lstlisting}
	
	\item Differences can be used in conjunction with lags (an option provided in the function):

	\begin{lstlisting}[style = rstyle, breaklines]
	diff(B1GQ.ts, lag = 4)
	\end{lstlisting}
	
	\item The latter will produce seasonal differences in the current example
\end{itemize}
\end{frame}

\section{Using contributed packages}
\subsection{The zoo package}
\begin{frame}[fragile]
\frametitle{The \textbf{zoo} package}
\begin{itemize}
	\item The \textbf{ts} class has some limitations, e.g. it cannot work with irregularly spaced time series
	\item This can cause unpleasant issues for example when you work with financial time series (especially daily or higher-frequency data\footnote{Monday to Friday are equally spaced but spacing between Friday and Monday is different, etc.})
	\item The package provides the \textbf{zoo} class of objects which handle this type of situations
	\item This class has similar (but much more powerful) functionality to that of \textbf{ts}
	\item Therefore, using \textbf{zoo} is very straightforward to use when one is familiar with \textbf{ts}
	\item Also, there is easy convertibility between the two classes
\end{itemize}
\end{frame}

\begin{frame}[fragile]
\frametitle{The \textbf{zoo} package (2)}
\begin{itemize}
	\item To create a \textbf{zoo} object, one needs data and a time-ordered index
	\item The data can be contained in a vector or in a matrix
	\item Let's take an example; create a random $120 \times 4$ matrix:
	
	\begin{lstlisting}[style = rstyle, breaklines]
	m1 <- matrix(rnorm(480), nrow = 120)
	colnames(m1) <- c("var1", "var2", "var3", "var4")
	\end{lstlisting}
	
	\item Create a time index of class \textbf{Date}:
	
	\begin{lstlisting}[style = rstyle, breaklines]
	idx1 <- seq(from = as.Date("2001-01-01"), 
            length.out = 120, by = "months")
	\end{lstlisting}
	
	\item Combine the data and the index in the \textbf{zoo} object:
	\begin{lstlisting}[style = rstyle, breaklines]
	zoo1 <- zoo(m1, order.by = idx1)
	\end{lstlisting}
	\item The index variable can be of any of the valid date and time classes
\end{itemize}
\end{frame}

\begin{frame}[fragile]
\frametitle{The \textbf{zoo} package (3)}
\begin{itemize}
	\item In order to extract the index of a \textbf{zoo} object, type:
	\begin{lstlisting}[style = rstyle, breaklines]
	index(zoo1)
	\end{lstlisting}
	\item The data can be extracted via:
	\begin{lstlisting}[style = rstyle, breaklines]
	coredata(zoo1)
	\end{lstlisting}
	\item Of course, both are assignable to new objects
	\item As with \textbf{ts} objects, you can find start and end dates by:
	\begin{lstlisting}[style = rstyle, breaklines]
	start(zoo1)
	end(zoo1)
	\end{lstlisting}
\end{itemize}
\end{frame}

\begin{frame}[fragile]
\frametitle{The \textbf{zoo} package (4)}
\begin{itemize}
	\item The \cc{summary()} command works seamlessly with \textbf{zoo} objects:
	\begin{lstlisting}[style = rstyle, breaklines]
	summary(zoo1)
	\end{lstlisting}
	\item The same goes for the \cc{str()} function:
	\begin{lstlisting}[style = rstyle, breaklines]
	str(zoo1)
	\end{lstlisting}
	\item For larger \textbf{zoo} objects, it is convenient to look at only some observations and not at all of them:
	\begin{lstlisting}[style = rstyle, breaklines]
	head(zoo1)
	tail(zoo1)
	\end{lstlisting}
	(the latter two work also for other objects)
\end{itemize}
\end{frame}

\begin{frame}[fragile]
\frametitle{The \textbf{zoo} package (5)}
\begin{itemize}
	\item Subsetting (sampling) is (not surprisingly) done by means of the \cc{window()} function:
	\begin{lstlisting}[style = rstyle, breaklines]
	zoo2 <- window(zoo1, start = as.Date(2005-11-15), end = as.Date(2007-08-30))
	\end{lstlisting}
	\item You can check the type of the new object to see that the data type is preserved
	\item An individual series can be selected in two ways:
	\begin{lstlisting}[style = rstyle, breaklines]
	zoo1[,1]
	zoo1$var1
	\end{lstlisting}
	(the former uses matrix notation, and the latter -- list notation)
	\item Individual elements (observations) can be selected using their matrix indices or, again, list notation:
	\begin{lstlisting}[style = rstyle, breaklines]
	zoo1[2,3]
	zoo1$var3[2] 	
	\end{lstlisting} 
\end{itemize}
\end{frame}

\begin{frame}[fragile]
\frametitle{The \textbf{zoo} package (6)}
\begin{itemize}
	\item Lags and differences is performed in the same way as with \textbf{ts} objects:
	\begin{lstlisting}[style = rstyle, breaklines]
	stats::lag(zoo1, -1)
	diff(zoo1, differences = 1)
	\end{lstlisting}
	\item It is possible to merge several \textbf{zoo} objects:
	\begin{lstlisting}[style = rstyle, breaklines]
	zoo3 <- cbind(zoo1$var2, zoo1$var1, stats::lag(zoo1$var1, -1)) 
	# or
	zoo3 <- merge(zoo1$var2, zoo1$var1, stats::lag(zoo1$var1, -1))
	\end{lstlisting}
	\item It is preferable however to use \cc{merge()} as it works as expected for cases when different objects have different time indices (not valid for \cc{cbind()})
\end{itemize}
\end{frame}

\begin{frame}[fragile]
\frametitle{The \textbf{zoo} package (7)}
\begin{itemize}
	\item External data can be read directly to \textbf{zoo} objects
	\item This is done by means of the \cc{read.zoo()} command which essentially is a wrapper around the \cc{read.table()} command
	\item Example (use again the GDP data):
	\begin{lstlisting}[style = rstyle, breaklines]
	data_gdpq_bg.zoo <- read.zoo("data_gdpq_bg.csv", 
                      index.column = 1, 
                      FUN = as.Date,
                      sep = ",",
                      header = T)
	\end{lstlisting}
\end{itemize}
\end{frame}

\begin{frame}[fragile]
\frametitle{The \textbf{zoo} package (8)}
\begin{itemize}
	\item In this example, \cc{index.column = 1} refers to the column where date and time information lies
	\item \cc{FUN = as.Date} specifies the function to apply to this column so that it is converted to the \textbf{Date} class (\textbf{yearmon} and \textbf{yearqtr} are also provided by \textbf{zoo})
	\item See the documentation for a complete list of options
	\item Also, note that the options are extensible by the list of the options of \cc{read.table()}
	\item Similarly to \cc{write.table()}, there is a \cc{write.zoo()} command:
	\begin{lstlisting}[style = rstyle, breaklines]
	write.zoo(data_gdpq_bg.zoo, "data_gdpq_bg.zoo.csv", index.name = "Time period", sep = ",")
	\end{lstlisting}
\end{itemize}
\end{frame}

\begin{frame}[fragile]
\frametitle{The \textbf{zoo} package (9)}
\begin{itemize}
	\item Finally, let's take examples of conversion between \textbf{zoo} and \textbf{ts}
	\item Direct conversion from \textbf{zoo} to \textbf{ts} is no longer possible, so you have to use a data frame as a medium
	
	\begin{lstlisting}[style = rstyle, breaklines]
	ts_conv <- as.data.frame(data_gdpq_bg.zoo)
	class(ts_conv)
	ts_conv <- ts(ts_conv, start = c(1995,1), frequency = 4)
	\end{lstlisting}
	
	\item To convert from \textbf{ts} to \textbf{zoo}:
	
	\begin{lstlisting}[style = rstyle, breaklines]
	zoo_conv <- as.zoo(ts_comb1)
	zoo_conv
	index(zoo_conv)
	class(zoo_conv)
	\end{lstlisting}
\end{itemize}
\end{frame}

\subsection{The xts package}
\begin{frame}[fragile]
\frametitle{The \textbf{xts} package}
\begin{itemize}
	\item The \textbf{xts} class is an extension of the \textbf{zoo} class
	\item Decrypted as ``extensible time series''
	\item Possesses a wider set of functions allowing more versatile data processing, manipulation, and conversion
	\item This is achieved alongside with greater user-friendliness, simplicity and usability
	\item Each \textbf{xts} objects has three components:
	\begin{itemize}
		\item A vector of times and/or dates
		\item The core data which is again a matrix 
		\item Attributes which include an index of times and dates and time zone format
	\end{itemize}
\end{itemize}
\end{frame}

\begin{frame}[fragile]
\frametitle{The \textbf{xts} package (2)}
\begin{itemize}
	\item To create an \textbf{xts} object, the \cc{xts()} command is used:
	\begin{lstlisting}[style = rstyle, breaklines]
	xts1 <- xts(data_gdpq_bg[,2:3], order.by = data_gdpq_bg$time, descr = "first xts object")
	\end{lstlisting}
	\item Obviously, it can contain a \cc{descr} attribute for easier reference
	\item Check its class to see that it is created on top of a \textbf{zoo} structure:
	\begin{lstlisting}[style = rstyle, breaklines]
	class(xts1)
	\end{lstlisting}
	\item Data can be queried/sampled by character matching, e.g.:
	\begin{lstlisting}[style = rstyle, breaklines]
	xts1["2010"] # select all quarters of 2010
	xts1["2010-01"] # select Q1 2010
	xts1["2010-01/2010-06"] # select Q1-Q2 2010
	\end{lstlisting}
\end{itemize}
\end{frame}

\begin{frame}[fragile]
\frametitle{The \textbf{xts} package (3)}
\begin{itemize}
	\item Let's create a date-time object in the POSIXct format using the \textbf{lubridate} package:
	\begin{lstlisting}[style = rstyle, breaklines]
	library(lubridate)
	nowtime <- now()
	idx2 <- nowtime + days(0:364)
	last(idx2)		
	\end{lstlisting}
	\item Create a data vector of the same length:
	\begin{lstlisting}[style = rstyle, breaklines]
	rnorm_data <- rnorm(length(idx2))
	\end{lstlisting}
	\item Create the \textbf{xts} object
	\begin{lstlisting}[style = rstyle, breaklines]
	xts2 <- xts(rnorm_data, order.by = idx2)
	colnames(xts2)[1] <- "data"
	head(xts2)
	\end{lstlisting}
\end{itemize}
\end{frame}

\begin{frame}[fragile]
\frametitle{The \textbf{xts} package (4)}
\begin{itemize}
	\item There are specialized loop functions for \textbf{xts} objects, e.g.:
	\begin{lstlisting}[style = rstyle, breaklines]
	apply.monthly(xts2, mean)
	apply.quarterly(xts2, sd)
	apply.yearly(xts2, sum)
	\end{lstlisting}
	\item You can also use an anonymous function in place of the above
	\item Merging \textbf{xts} objects is again done with \cc{merge()}, with some more additional options (not discussed here)
	\item Start time and ending time are obtained by means of:
	\begin{lstlisting}[style = rstyle, breaklines]
	start(xts2)
	end(xts2)
	\end{lstlisting}
\end{itemize}
\end{frame}

\begin{frame}[fragile]
\frametitle{The \textbf{xts} package (5)}
\begin{itemize}
	\item You can also calculate the number of days, weeks, months, etc. in an \textbf{xts} object:
	\begin{lstlisting}[style = rstyle, breaklines]
	ndays(xts2)
	nmonths(xts2)
	nyears(xts2)
	\end{lstlisting}
	\item Statistical calculations by time period are performed in the following manner:
	\begin{lstlisting}[style = rstyle, breaklines]
	endp1 <- endpoints(xts2,'weeks')
	period.apply(xts2, INDEX=endp1, mean)
	endp2 <- endpoints(xts2,'months')
	period.max(xts2, INDEX=endp2)
	\end{lstlisting}
	\item etc.
\end{itemize}
\end{frame}

\begin{frame}[fragile]
\frametitle{References}
\begin{itemize}
	\item Packages documentation
	\item Kabacoff, R. (2015): \textit{R in Action: Data analysis and graphics with R}, Manning, 2nd ed., Ch. 15
	\item Zhang, D. (2016): \textit{R for Programmers}, CRC Press, Ch. 2
\end{itemize}
\end{frame}

\end{document}
\documentclass[10pt]{beamer}
\usetheme{CambridgeUS}
%\usetheme{Boadilla}
\definecolor{myred}{RGB}{163,0,0}
%\usecolortheme[named=blue]{structure}
\usecolortheme{dove}
\usefonttheme[]{professionalfonts}
\usepackage[english]{babel}
\usepackage{amsmath,amsfonts,amssymb}
\usepackage{mathtools}
\usepackage{commath}
\usepackage{xcolor}
\usepackage{tikz}
\usepackage{pgfplots}
\pgfplotsset{compat=newest,compat/show suggested version=false}
\usetikzlibrary{arrows,shapes,calc,backgrounds}
\usepackage{bm}
\usepackage{textcomp}
%\usepackage{gensymb}
%\usepackage{verbatim}
\usepackage{mathrsfs}  
\usepackage{paratype}
\usepackage{mathpazo}
\usepackage{listings}
\usepackage{csvsimple}
\usepackage{booktabs}
\usepackage{longtable}

\newcommand{\cc}[1]{\texttt{\textcolor{blue}{#1}}}

\DeclareMathOperator{\adj}{adj}
\DeclareMathOperator{\tr}{tr}
\DeclareMathOperator{\diag}{diag}
\DeclareMathOperator{\E}{\mathsf{E}}
\DeclareMathOperator{\var}{\mathsf{Var}}
\DeclareMathOperator{\cov}{\mathsf{Cov}}
\DeclareMathOperator{\corr}{\mathsf{Corr}}
\DeclareMathOperator{\plim}{plim}
\DeclareMathOperator{\rank}{rank}


\definecolor{ttcolor}{RGB}{0,0,1}%{RGB}{163,0,0}
\definecolor{mygray}{RGB}{248,249,250}

% Number theorem environments
\setbeamertemplate{theorem}[ams style]
\setbeamertemplate{theorems}[numbered]

% Reset theorem-like environments so that each is numbered separately
\usepackage{etoolbox}
\undef{\definition}
\theoremstyle{definition}
\newtheorem{definition}{\translate{Definition}}

% Change colours for theorem-like environments
\definecolor{mygreen1}{RGB}{0,96,0}
\definecolor{mygreen2}{RGB}{229,239,229}
\setbeamercolor{block title}{fg=white,bg=mygreen1}
\setbeamercolor{block body}{fg=black,bg=mygreen2}

\lstdefinestyle{numbers}{numbers=left, stepnumber=1, numberstyle=\tiny, numbersep=10pt}
\lstdefinestyle{MyFrame}{backgroundcolor=\color{yellow},frame=shadowbox}

\lstdefinestyle{rstyle}%
{language=R,
	basicstyle=\footnotesize\ttfamily,
	backgroundcolor = \color{mygray},
	commentstyle=\slshape\color{green!50!black},
	keywordstyle=\color{blue},
	identifierstyle=\color{blue},
	stringstyle=\color{orange},
	%escapechar=\#,
	rulecolor = \color{mygray}, 
	showstringspaces = false,
	showtabs = false,
	tabsize = 2,
	emphstyle=\color{red},
	frame = single}

\lstset{language=R,frame=single}    

\hypersetup{colorlinks, urlcolor=blue, linkcolor = myred}

\AtBeginSection{\frame{\sectionpage}}

% Remove Section 1, Section 2, etc. as titles in section pages
\defbeamertemplate{section page}{mine}[1][]{%
	\begin{centering}
		{\usebeamerfont{section name}\usebeamercolor[fg]{section name}#1}
		\vskip1em\par
		\begin{beamercolorbox}[sep=12pt,center]{part title}
			\usebeamerfont{section title}\insertsection\par
		\end{beamercolorbox}
	\end{centering}
} 

\setbeamertemplate{section page}[mine] 

\beamertemplatenavigationsymbolsempty


\title{R403: Probabilistic and Statistical Computations\\ with R}
\subtitle{Topic 08: \textcolor{myred}{Data Management with the \textbf{dplyr} Package}}
\author{Kaloyan Ganev}

\date{2022/2023} 

\begin{document}
\maketitle

\begin{frame}[fragile]
\frametitle{Lecture Contents}
\tableofcontents
\end{frame}

\section{Introduction}
\begin{frame}[fragile]
\frametitle{Introduction}
\begin{itemize}
	\item Working with data frames in the `traditional' way can be cumbersome
	\item Luckily, additional packages that save a lot of time and trouble exist
	\item Two notable alternatives: the \textbf{dplyr} package, and the \textbf{data.table} package
	\item We will not discuss the latter (it is left to your curiosity)
	\item Our focus will be on the former
	\item Before that, we will discuss the `ecosystem' that it belongs to: the \textbf{tidyverse}
	\item \textcolor{red}{Note:} A whole book available here in HTML format: \url{http://r4ds.had.co.nz/}
\end{itemize}
\end{frame}

\section{The tidyverse}
\begin{frame}[fragile]
\frametitle{The \textbf{tidyverse}}
\begin{itemize}
	\item A group of packages that ``share an underlying design philosophy, grammar, and data structures'' (\url{https://www.tidyverse.org/})
	\item This is what in fact makes the ``ecosystem''
	\item Main purpose: data science, but naturally appropriate for other purposes, too
	\item Take a look at the package list; we will pay some attention to a couple of them initially, then we will switch to \textbf{dplyr}
	\item Over the course of the module, we will have thorough sessions on some other among them (\textbf{ggplot2} in particular)
	\item To install \textbf{tidyverse}:
	\begin{lstlisting}[style = rstyle]
	install.packages("tidyverse")
	\end{lstlisting}
\end{itemize}
\end{frame}

\begin{frame}[fragile]
\frametitle{The \textbf{readr} Package}
\begin{itemize}
	\item Load \textbf{tidyverse}:
	\begin{lstlisting}[style = rstyle]
	library(tidyverse)
	\end{lstlisting}
	\item \textbf{readr} has similar wrappers to the ones present in base R (recall the \cc{read.table()} function):
	\begin{itemize}
		\item \cc{read\_csv()}
		\item \cc{read\_csv2()}
		\item \cc{read\_tsv()}
		\item \cc{read\_delim()}
		\item \cc{read\_fwf()}
		\item \cc{read\_table()}: for reading fixed-width files, columns separated by white space
	\end{itemize}
\end{itemize}
\end{frame}

\begin{frame}[fragile]
\frametitle{The \textbf{readr} Package (2)}
\begin{itemize}
	\item An example:\footnote{Taken from \url{https://catalog.data.gov/dataset?res_format=CSV}}
	\begin{lstlisting}[style = rstyle, breaklines]
	data1 <- read_csv("Demographic_Statistics_By_Zip_Code.csv")
	\end{lstlisting}
	\item Read now the help on \cc{read\_csv()} to see what options are available
	\item Check:
	\begin{lstlisting}[style = rstyle, breaklines]
	class(data1)
	\end{lstlisting}
	\item Note that the latter does not produce just \cc{"data.frame"}
	\item The object that is created is a modern version of the data frame called \textit{tibble}
	\item The \textbf{tibble} package is also a member of the \textbf{tidyverse} family
\end{itemize}
\end{frame}

\begin{frame}[fragile]
\frametitle{The \textbf{tibble} package}
\begin{itemize}
	\item Advantages of tibbles: 
	\begin{itemize}
		\item Better looking when printed in the console
		\item Printing options can be controlled
		\item Referencing (slicing) works analogically to data frames
		\item Does not change the data type of inputs
		\item Does not change variable names
		\item Does not create row names
	\end{itemize}
	\item To convert a data frame to a tibble:
	\begin{lstlisting}[style = rstyle, breaklines]
	df1 <- as.data.frame(matrix(1:9, nrow = 3))	
	tb1 <- as_tibble(df1)
	\end{lstlisting}
\end{itemize}
\end{frame}

\section{Working with \textbf{dplyr}}
\begin{frame}[fragile]
\frametitle{Working with \textbf{dplyr}}
\begin{itemize}
	\item We will use the dataset that the book works with: flights from New York City in 2013
	\begin{lstlisting}[style = rstyle, breaklines]
	install.packages("nycflights13")
	library(nycflights13)
	print(flights)
	\end{lstlisting}
	\item Obviously, the data are organized in a tibble
	\item We can now, for example, select specific rows based on specified values
	\item In order to select all flights that took place on 27 July,
	\begin{lstlisting}[style = rstyle, breaklines]
	july27 <- filter(flights, month == 7, day == 27)	
	\end{lstlisting}
	\item Mind the double equality signs, they are logical checks!
\end{itemize}
\end{frame}

\begin{frame}[fragile]
\frametitle{Working with \textbf{dplyr} (2)}
\begin{itemize}
	\item Boolean operators that are used: \cc{\&} (and), \cc{|} (or), and \cc{!} (not)
	\item An \cc{or} example:
	\begin{lstlisting}[style = rstyle, breaklines]
	july7or27 <- filter(flights, (day == 7 | day == 27) & month == 7)
	\end{lstlisting}
	\item A \cc{not} example:
	\begin{lstlisting}[style = rstyle, breaklines]
	not_in_jan <- filter(flights, month != 1)
	\end{lstlisting}
	\item ...or another one:
	\begin{lstlisting}[style = rstyle, breaklines]
	not_in_winter <- filter(flights, !(month %in% c(1,2,12)))
	\end{lstlisting}
	\item Note in the latter the \cc{\%in\%} function that serves as a convenient shorthand
\end{itemize}
\end{frame}

\begin{frame}[fragile]
\frametitle{Working with \textbf{dplyr} (3)}
\begin{itemize}
	\item Data can be sorted in the following way:
	\begin{lstlisting}[style = rstyle, breaklines]
	sort1 <- arrange(flights, dep_delay, carrier)	
	\end{lstlisting}
	\item In this example, the data will be sorted first according to departure delay, and then by carrier
	\item Sorting in performed in ascending order
	\item To sort in descending order:
	\begin{lstlisting}[style = rstyle, breaklines]
	sort2 <- arrange(filter(flights, month == 1, day == 1), desc(dep_delay))	
	\end{lstlisting}
\end{itemize}
\end{frame}

\begin{frame}[fragile]
\frametitle{Working with \textbf{dplyr} (4)}
\begin{itemize}
	\item To select variables (columns) from a tibble/data frame:
	\begin{lstlisting}[style = rstyle, breaklines]
	new_tbl1 <- select(flights, year, month, day, flight)	
	\end{lstlisting}
	\item Shortcut for selecting adjacent columns:
	\begin{lstlisting}[style = rstyle, breaklines]
	new_tbl2 <- select(flights, year:day, flight)	
	\end{lstlisting}
	\item Select all columns with the exception of some:
	\begin{lstlisting}[style = rstyle, breaklines]
	new_tbl3 <- select(flights, year:day, -c(hour, minute))
	\end{lstlisting}
	\item Options to the \cc{select()} function such as \cc{starts\_with()}, \cc{ends\_with()}, \cc{contains()}, etc. can be used, e.g.
	\begin{lstlisting}[style = rstyle, breaklines]
	new_tbl4 <- select(flights, starts_with("flig"))
	\end{lstlisting}
\end{itemize}
\end{frame}

\begin{frame}[fragile]
\frametitle{Working with \textbf{dplyr} (5)}
\begin{itemize}
	\item Renaming variables is also simple (new name comes first):
	\begin{lstlisting}[style = rstyle, breaklines]
	flights1 <- rename(flights, airline = carrier)	
	\end{lstlisting}
	\item A way to rearrange columns:
	\begin{lstlisting}[style = rstyle, breaklines]
	flights2 <- select(flights, time_hour, everything())
	\end{lstlisting}
	\item Here, the \cc{time\_hour} variable is moved to the first position, \cc{everything()} plays the role of everything else
	\item New variables are created with
	\begin{lstlisting}[style = rstyle, breaklines]
	flights <- 	mutate(flights, travel_time = arr_time - dep_time)
	\end{lstlisting}
\end{itemize}
\end{frame}

\begin{frame}[fragile]
\frametitle{Working with \textbf{dplyr} (6)}
\begin{itemize}
	\item Grouped summaries:
	\begin{lstlisting}[style = rstyle, breaklines]
	by_airline <- group_by(flights, carrier)
	summarize(by_airline, delay = mean(dep_delay, na.rm = TRUE))	
	\end{lstlisting}
	\item In this example, the average departure delay by airline is calculated
	\item Grouping can be performed by more than one variable
	\item In such cases, summaries can be calculated per level of grouping
	\item Instead of performing operations one by one, the so-called pipe can be used, e.g.:
	\begin{lstlisting}[style = rstyle, breaklines]
	avg_delay <- flights %>% 
	  group_by(carrier) %>%
	  summarize(mean(dep_delay, na.rm = TRUE))	
	\end{lstlisting}
\end{itemize}
\end{frame}
\end{document}

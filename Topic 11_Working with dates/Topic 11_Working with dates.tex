\documentclass[10pt]{beamer}
\usetheme{CambridgeUS}
%\usetheme{Boadilla}
\definecolor{myred}{RGB}{163,0,0}
%\usecolortheme[named=blue]{structure}
\usecolortheme{dove}
\usefonttheme[]{professionalfonts}
\usepackage[english]{babel}
\usepackage{amsmath,amsfonts,amssymb}
\usepackage{mathtools}
\usepackage{commath}
\usepackage{xcolor}
\usepackage{tikz}
\usepackage{pgfplots}
\pgfplotsset{compat=newest,compat/show suggested version=false}
\usetikzlibrary{arrows,shapes,calc,backgrounds}
\usepackage{bm}
\usepackage{textcomp}
%\usepackage{gensymb}
%\usepackage{verbatim}
\usepackage{mathrsfs}  
\usepackage{paratype}
\usepackage{mathpazo}
\usepackage{listings}
\usepackage{csvsimple}
\usepackage{booktabs}
\usepackage{longtable}

\newcommand{\cc}[1]{\texttt{\textcolor{blue}{#1}}}

\DeclareMathOperator{\adj}{adj}
\DeclareMathOperator{\tr}{tr}
\DeclareMathOperator{\diag}{diag}
\DeclareMathOperator{\E}{\mathsf{E}}
\DeclareMathOperator{\var}{\mathsf{Var}}
\DeclareMathOperator{\cov}{\mathsf{Cov}}
\DeclareMathOperator{\corr}{\mathsf{Corr}}
\DeclareMathOperator{\plim}{plim}
\DeclareMathOperator{\rank}{rank}


\definecolor{ttcolor}{RGB}{0,0,1}%{RGB}{163,0,0}
\definecolor{mygray}{RGB}{248,249,250}

% Number theorem environments
\setbeamertemplate{theorem}[ams style]
\setbeamertemplate{theorems}[numbered]

% Reset theorem-like environments so that each is numbered separately
\usepackage{etoolbox}
\undef{\definition}
\theoremstyle{definition}
\newtheorem{definition}{\translate{Definition}}

% Change colours for theorem-like environments
\definecolor{mygreen1}{RGB}{0,96,0}
\definecolor{mygreen2}{RGB}{229,239,229}
\setbeamercolor{block title}{fg=white,bg=mygreen1}
\setbeamercolor{block body}{fg=black,bg=mygreen2}

\lstdefinestyle{numbers}{numbers=left, stepnumber=1, numberstyle=\tiny, numbersep=10pt}
\lstdefinestyle{MyFrame}{backgroundcolor=\color{yellow},frame=shadowbox}

\lstdefinestyle{rstyle}%
{language=R,
	basicstyle=\footnotesize\ttfamily,
	backgroundcolor = \color{mygray},
	commentstyle=\slshape\color{green!50!black},
	keywordstyle=\color{blue},
	identifierstyle=\color{blue},
	stringstyle=\color{orange},
	%escapechar=\#,
	rulecolor = \color{mygray}, 
	showstringspaces = false,
	showtabs = false,
	tabsize = 2,
	emphstyle=\color{red},
	frame = single}

\lstset{language=R,frame=single}    

\hypersetup{colorlinks, urlcolor=blue, linkcolor = myred}

\AtBeginSection{\frame{\sectionpage}}

% Remove Section 1, Section 2, etc. as titles in section pages
\defbeamertemplate{section page}{mine}[1][]{%
	\begin{centering}
		{\usebeamerfont{section name}\usebeamercolor[fg]{section name}#1}
		\vskip1em\par
		\begin{beamercolorbox}[sep=12pt,center]{part title}
			\usebeamerfont{section title}\insertsection\par
		\end{beamercolorbox}
	\end{centering}
} 

\setbeamertemplate{section page}[mine] 

\beamertemplatenavigationsymbolsempty

\title{R403: The R Language for Statistical Computing}
\subtitle{Topic 11: \textcolor{myred}{Working with dates}}
\author{Kaloyan Ganev}

\date{2022/2023} 

\begin{document}
\maketitle

\begin{frame}[fragile]
\frametitle{Lecture Contents}
\tableofcontents
\end{frame}

\section{Introduction}
\begin{frame}[fragile]
\frametitle{Introductory notes}
\begin{itemize}
	\item As in regular spreadsheet software, date information should be formatted in a specific way so that it is recognized as such
	\item R can handle dates and times in several ways
	\item We will discuss each one in turns
	\item Why do we need to know this?
	\item Answer: Time series analysis + the need to use different date and time formats and convert one to another
\end{itemize}
\end{frame}

\section{The \texttt{Date} class}
\begin{frame}[fragile]
\frametitle{The \texttt{Date} class}
\begin{itemize}
	\item This is R's simplest class for handling dates (but not times)
	\item Date information is stored in R as integers
	\item More specifically, ``day zero'' is Jan 1, 1970
	\item Every next date corresponds to the count of days from day zero
	\item Dates before date zero correspond to negative counts
	\item Uses the Gregorian calendar, so dates earlier than 1752 should be worked with very cautiously
\end{itemize}
\end{frame}

\begin{frame}[fragile]
\frametitle{The \texttt{Date} class (2)}
\begin{itemize}
	\item To get the current date, use:
	\begin{lstlisting}[style = rstyle, breaklines]
	Sys.Date()
	\end{lstlisting}
	\item (The output can be converted to a different format if necessary; more on that later)
	\item See that date information is formatted as follows: \texttt{yyyy-mm-dd}
	\item So, if you need to store date information in the built-in Date class, just write:
	\begin{lstlisting}[style = rstyle, breaklines]
	datevar1 <- as.Date("2016-10-08")
	\end{lstlisting}
	\item It is also directly acceptable to use slashes, i.e.: \texttt{yyyy/mm/dd}
	\begin{lstlisting}[style = rstyle, breaklines]
	datevar2 <- as.Date("2016/10/08")
	\end{lstlisting}
\end{itemize}
\end{frame}

\begin{frame}[fragile]
\frametitle{The \texttt{Date} class (3)}
\begin{itemize}
	\item If you work with data that comes with dates formatted differently, you could use it by specifying a format string 
	\item Format strings are composed of elements shown in the table below
	\bigskip	
	\begin{center}
	\begin{tabular}{cl}
	\toprule
	Code & Value\\
	\midrule
	\texttt{\%d} & Day of the month (decimal number)\\
	\texttt{\%m} & Month (decimal number)\\
	\texttt{\%b} & Month (abbreviated) \textcolor{red}{in the current locale}\\
	\texttt{\%B} & Month (full name) \textcolor{red}{in the current locale}\\
	\texttt{\%y} & Year (2 digit)\\
	\texttt{\%Y} & Year (4 digit)\\
	\bottomrule
	\end{tabular}
	\end{center}
\end{itemize}
\end{frame}

\begin{frame}[fragile]
\frametitle{The \texttt{Date} class (4)}
\begin{itemize}
	\item Examples:
	\begin{lstlisting}[style = rstyle, breaklines]
	Sys.getlocale()
	Sys.setlocale(category = "LC_ALL", locale = "English_United States.1252")
	as.Date("07/31/2016", format = "%m/%d/%Y")
	as.Date("December 26, 2011", format = "%B %d, %Y")
	as.Date("26MAR2011", format = "%d%b%Y")
	\end{lstlisting}
	\item If you import date data from MS Excel, note that it uses a different time origin (day zero is Dec 30, 1899)
	\item You can tell this to R, for example:
	\begin{lstlisting}[style = rstyle, breaklines]
	as.Date(42651, origin = "1899-12-30")
	\end{lstlisting}
\end{itemize}
\end{frame}

\begin{frame}[fragile]
\frametitle{The \texttt{Date} class (5)}
\begin{itemize}
	\item To see which number corresponds to a date:
	\begin{lstlisting}[style = rstyle, breaklines]
	as.numeric(datevar1)
	\end{lstlisting}
	\item You can also calculate differences, e.g. to find out how many days have passed since your birthday:
	\begin{lstlisting}[style = rstyle, breaklines]
	Sys.Date() - as.Date("1991-11-29")
	difftime(Sys.Date(), as.Date("1991-11-29"), units = "weeks") # could be "auto", "secs", "mins", "hours", "days", "weeks"
	\end{lstlisting}
\end{itemize}
\end{frame}

\begin{frame}[fragile]
\frametitle{Time Sequences with the Date Class}
\begin{itemize}
	\item Created using the \cc{seq()} function
	
	\textcolor{blue}{Example:} create a sequence of 20 dates starting from Oct 10, 2016:
	\begin{lstlisting}[style = rstyle, breaklines]
	tseq1 <- seq(as.Date("2016-10-10"), by = "days", length = 20)
	\end{lstlisting}
	
	\item It is also possible to do the following:
	\begin{lstlisting}[style = rstyle, breaklines]
	tseq2 <- seq(from = as.Date("2016-10-10"), to = as.Date("2017-10-10"), by = "3 weeks")
	\end{lstlisting}
\end{itemize}
\end{frame}

\section{The chron package}
\begin{frame}[fragile]
\frametitle{The \textbf{chron} package}
\begin{itemize}
	\item Should be installed additionally
	\item Allows working with times in addition to dates
	\item Cannot handle time zones and daylight savings time (yet)
	\item Simple examples
	\begin{lstlisting}[style = rstyle, breaklines]
	library(chron)
	dates1 <- dates(c("03/21/2002", "04/26/12", "01/11/14", "01/28/1915", "02/10/2016"))
	dates1[5] - dates1[4]
	times1 <- times(c("20:05:21", "19:29:59", "11:13:31", "10:29:14", "06:48:02"))
	datestimes1 <- chron(dates = dates1, times = times1)
	\end{lstlisting}
\end{itemize}
\end{frame}

\begin{frame}[fragile]
\frametitle{The \textbf{chron} package (2)}
\begin{itemize}
	\item You can also add or subtract numbers
	\begin{lstlisting}[style = rstyle, breaklines]
	datestimes1[2] + 3.5
	\end{lstlisting}
	\item \cc{range()}, \cc{diff()}, etc. are also possible to use
	\item As always, further details on additional functions can be found in the documentation
\end{itemize}
\end{frame}

\section{POSIX classes}
\begin{frame}[fragile]
\frametitle{POSIX classes}
\begin{itemize}
	\item What is POSIX? See \url{https://en.wikipedia.org/wiki/POSIX}
	\item In addition to all above, POSIX classes allow handling time zones
	\item In R there are two standard POSIX date-time classes: \texttt{POSIXct} and \texttt{POSIXlt}
	\item \texttt{POSIXct} (\texttt{ct} stands for ``calendar time''): date and time values are stored as the number of seconds since the origin (Jan 1, 1970)
	\item \texttt{POSIXlt} (\texttt{ct} stands for ``local time''): date and time values are stored as a list; separate elements are used to store seconds, minutes, hours, etc. 
	\item the POSIXct class is usually preferred
\end{itemize}
\end{frame}

\begin{frame}[fragile]
\frametitle{POSIX classes (2)}
\begin{itemize}
\item To get the current date and time in POSIXct form:
	\begin{lstlisting}[style = rstyle, breaklines]
	Sys.time()
	\end{lstlisting}
	\item Input format: 
	\begin{itemize}
		\item Dates are specified by first writing the year, then the month, and finally the day; the three elements are separated by dashes or slashes:
		\begin{lstlisting}[style = rstyle, breaklines]
		pos1 <- "2016-10-15"
		pos2 <- "2016/10/16"
		posdate1 <- as.POSIXct(pos1)
		posdate2 <- as.POSIXct(pos2)
		c(posdate1,posdate2)
		\end{lstlisting} 
		\item If you need to have also times, they are separated from date by a space and then hours, minutes, and seconds are entered separated by a colon:
		\begin{lstlisting}[style = rstyle, breaklines]
		pos3 <- "2016-10-15 15:45:37"
		posdate <- as.POSIXct(pos3)
		\end{lstlisting}
	\end{itemize}	
\end{itemize} 
\end{frame}

\begin{frame}[fragile]
\frametitle{POSIX classes (3)}
\begin{itemize}
	\item An example with using market tick data (high-frequency financial data):
	\begin{lstlisting}[style = rstyle, breaklines]
	tick_data <- read.csv("tick_data.csv", stringsAsFactors = FALSE)
	tick_data$DATE <- as.Date(tick_data$DATE, format = "%m/%d/%Y")
	tick_data$datetime <- paste(tick_data$DATE, tick_data$TIME, sep = " ")
	tick_data$datetime <- as.POSIXct(tick_data$datetime)
	tick_data <- subset(tick_data, select = c(5,3:4))
	\end{lstlisting}
\end{itemize}
\end{frame}

\begin{frame}[fragile]
\frametitle{POSIX classes (4)}
\begin{itemize}
	\item As in the built-in \texttt{Date} class, it is possible to specify the format of the date
	\item Also, there is a command option that allows to explicitly specify the time zone
	\item Example:
	\begin{lstlisting}[style = rstyle, breaklines]
	date0 <- as.POSIXct("10-10-2016", format = "%d-%m-%Y", tz = "EST")
	\end{lstlisting}
	\item When you print the result, see that the time zone is displayed as \texttt{EEST} (meaning Easter European Summer Time)
	\item This means that R automatically takes care of daylight saving time; compare with a December date:
	\begin{lstlisting}[style = rstyle, breaklines]
	date0 <- as.POSIXct("10-12-2016", format = "%d-%m-%Y", tz = "EST")
	\end{lstlisting}
	\item List of time zone abbreviations: \url{https://en.wikipedia.org/wiki/List_of_time_zone_abbreviations}
\end{itemize}
\end{frame}

\section{The \texttt{lubridate} package}
\begin{frame}[fragile]
\frametitle{The \textbf{lubridate} package}
\begin{itemize}
	\item Install it so that you can use it
	\item Provides greater ease of working with \texttt{POSIXt}, \texttt{Date} and \textbf{chron} objects
	\item Provides compatibility with many time-series objects specific to popular packages such as \textbf{zoo}, \textbf{xts}, \textbf{tSeries}, \textbf{fts}, etc.
	\item Documentation spans lots of pages
	\item Strings are read into R as POSIXct date-time objects
\end{itemize}
\end{frame}

\begin{frame}[fragile]
\frametitle{The \textbf{lubridate} package (2)}
\begin{itemize}
	\item Reading is implemented through a series of functions listed in the following table:
	\bigskip
	
	\begin{tabular}{ll}
	\toprule
	Function & Order of elements\\
	\midrule
	\texttt{ymd()} & Year, month, day\\
	\texttt{ydm()} & Year, day, month\\
	\texttt{mdy()} & Month, day, year\\
	\texttt{dmy()} & Day, month, year\\
	\texttt{hm()} & Hour, minute\\
	\texttt{hms()} & Hour, minute, second\\
	\texttt{ymd\_hms()}\footnote{And permutations of \texttt{ymd}.} & Year, month, day, hour, minute, second\\
	\bottomrule
	\end{tabular}
	\item Example of usage:
	\begin{lstlisting}[style = rstyle, breaklines]
	datetime0 <- dmy_hms("15-09-2016 13:45:01")
	\end{lstlisting}
\end{itemize}
\end{frame}

\begin{frame}[fragile]
\frametitle{The \textbf{lubridate} package (3)}
\begin{itemize}
	\item To set the time zone, two options are available
	\item First, let's get current time; with \textbf{lubridate} this is as easy as:
	\begin{lstlisting}[style = rstyle, breaklines]
	nowtime <- now()
	\end{lstlisting}
	\item If you want to change the way the time instant contained in \texttt{nowtime} is displayed (i.e. what was the time of that instant in a different time zone, say UTC\footnote{Universal Time Zone; same time as GMT.}):
	\begin{lstlisting}[style = rstyle, breaklines]
	with_tz(nowtime, tzone = "UTC")
	\end{lstlisting}
	\item If you want to change the actual instant of time contained in
\texttt{nowtime} but keep the displayed clock time:
	\begin{lstlisting}[style = rstyle, breaklines]
	force_tz(nowtime, tzone = "UTC")
	\end{lstlisting}
\end{itemize}
\end{frame}

\begin{frame}[fragile]
\frametitle{The \textbf{lubridate} package (4)}
\begin{itemize}
	\item To extract the numerical value of the month, day, etc. in \texttt{nowtime}:
	\begin{lstlisting}[style = rstyle, breaklines]
	month(nowtime)
	day(nowtime)
	hour(nowtime)
	\end{lstlisting}
	\item To extract respectively the abbreviated and full name:
	\begin{lstlisting}[style = rstyle, breaklines]
	month(nowtime, label = TRUE)
	month(nowtime, label = TRUE, abbr = FALSE)
	\end{lstlisting}
	\item The same stuff can be done for weekdays with the \cc{wday()} command
	\begin{lstlisting}[style = rstyle, breaklines]
	wday(nowtime, label = TRUE)
	wday(nowtime, label = TRUE, abbr = FALSE)
	\end{lstlisting}
\end{itemize}
\end{frame}

\begin{frame}[fragile]
\frametitle{The \textbf{lubridate} package (5)}
\begin{itemize}
	\item Time intervals can also be created, withing the \texttt{interval} class
	\begin{lstlisting}[style = rstyle, breaklines]
	yesttime <- nowtime - days(1)
	int1 <- interval(nowtime, yesttime)
	\end{lstlisting}
	\item Two intervals can be checked for overlapping with \cc{int\_overlap()} (returns a logical value)
	\item If they overlap, the overlapping period can be calculated with \cc{setdiff()}
\end{itemize}
\end{frame}

\begin{frame}[fragile]
\frametitle{The \textbf{lubridate} package (6)}
\begin{itemize}
	\item It is easy to do a lot of arithmetic with \textbf{lubridate} objects
	\item For this purpose, there are the so-called helper functions
	\item There are two types of helper functions, respectively for creating two classes of time spans: periods and durations
	\item The functions for creating periods are named after the plurals of time units, e.g.:
	\begin{lstlisting}[style = rstyle, breaklines]
	years(1) + hours(2) + minutes(15) # integer values only allowed
	\end{lstlisting}
	\item The functions for creating durations just add a ``d'' in front:
	\begin{lstlisting}[style = rstyle, breaklines]
	dyears(1) # exactly 365 days
	\end{lstlisting}
\end{itemize}
\end{frame}

\begin{frame}[fragile]
\frametitle{The \textbf{lubridate} package (7)}
\begin{itemize}
	\item Are two classes really needed?
	\item Check out and compare the results of these:
	\begin{lstlisting}[style = rstyle, breaklines]
	dmy(31012016) + years(1)
	dmy(31012016) + dyears(1)
	\end{lstlisting}
	\item Why we get a difference? 2016 is a leap year!
	\item \textcolor{red}{Note: \textbf{lubridate} is vectorized so it can be applied to vectors too; also it is possible to use it within functions}
\end{itemize}
\end{frame}

\end{document}



Spector
p. 20: MLE
p. 30: Combinatorics

http://www.burns-stat.com/r-database-interfaces/
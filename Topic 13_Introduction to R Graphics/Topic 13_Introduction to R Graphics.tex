\documentclass[10pt]{beamer}
\usetheme{CambridgeUS}
%\usetheme{Boadilla}
\definecolor{myred}{RGB}{163,0,0}
%\usecolortheme[named=blue]{structure}
\usecolortheme{dove}
\usefonttheme[]{professionalfonts}
\usepackage[english]{babel}
\usepackage{amsmath,amsfonts,amssymb}
\usepackage{mathtools}
\usepackage{commath}
\usepackage{xcolor}
\usepackage{tikz}
\usepackage{pgfplots}
\pgfplotsset{compat=newest,compat/show suggested version=false}
\usetikzlibrary{arrows,shapes,calc,backgrounds}
\usepackage{bm}
\usepackage{textcomp}
%\usepackage{gensymb}
%\usepackage{verbatim}
\usepackage{mathrsfs}  
\usepackage{paratype}
\usepackage{mathpazo}
\usepackage{listings}
\usepackage{csvsimple}
\usepackage{booktabs}
\usepackage{longtable}

\newcommand{\cc}[1]{\texttt{\textcolor{blue}{#1}}}

\DeclareMathOperator{\adj}{adj}
\DeclareMathOperator{\tr}{tr}
\DeclareMathOperator{\diag}{diag}
\DeclareMathOperator{\E}{\mathsf{E}}
\DeclareMathOperator{\var}{\mathsf{Var}}
\DeclareMathOperator{\cov}{\mathsf{Cov}}
\DeclareMathOperator{\corr}{\mathsf{Corr}}
\DeclareMathOperator{\plim}{plim}
\DeclareMathOperator{\rank}{rank}


\definecolor{ttcolor}{RGB}{0,0,1}%{RGB}{163,0,0}
\definecolor{mygray}{RGB}{248,249,250}

% Number theorem environments
\setbeamertemplate{theorem}[ams style]
\setbeamertemplate{theorems}[numbered]

% Reset theorem-like environments so that each is numbered separately
\usepackage{etoolbox}
\undef{\definition}
\theoremstyle{definition}
\newtheorem{definition}{\translate{Definition}}

% Change colours for theorem-like environments
\definecolor{mygreen1}{RGB}{0,96,0}
\definecolor{mygreen2}{RGB}{229,239,229}
\setbeamercolor{block title}{fg=white,bg=mygreen1}
\setbeamercolor{block body}{fg=black,bg=mygreen2}

\lstdefinestyle{numbers}{numbers=left, stepnumber=1, numberstyle=\tiny, numbersep=10pt}
\lstdefinestyle{MyFrame}{backgroundcolor=\color{yellow},frame=shadowbox}

\lstdefinestyle{rstyle}%
{language=R,
	basicstyle=\footnotesize\ttfamily,
	backgroundcolor = \color{mygray},
	commentstyle=\slshape\color{green!50!black},
	keywordstyle=\color{blue},
	identifierstyle=\color{blue},
	stringstyle=\color{orange},
	%escapechar=\#,
	rulecolor = \color{mygray}, 
	showstringspaces = false,
	showtabs = false,
	tabsize = 2,
	emphstyle=\color{red},
	frame = single}

\lstset{language=R,frame=single}    

\hypersetup{colorlinks, urlcolor=blue, linkcolor = myred}

\AtBeginSection{\frame{\sectionpage}}

% Remove Section 1, Section 2, etc. as titles in section pages
\defbeamertemplate{section page}{mine}[1][]{%
	\begin{centering}
		{\usebeamerfont{section name}\usebeamercolor[fg]{section name}#1}
		\vskip1em\par
		\begin{beamercolorbox}[sep=12pt,center]{part title}
			\usebeamerfont{section title}\insertsection\par
		\end{beamercolorbox}
	\end{centering}
} 

\setbeamertemplate{section page}[mine] 

\beamertemplatenavigationsymbolsempty 

\title{R403: Probabilistic and Statistical Computations\\ with R}
\subtitle{Topic 13: \textcolor{myred}{Introduction to R Graphics}}
\author{Kaloyan Ganev}

\date{2022/2023} 

\begin{document}
\maketitle

\begin{frame}[fragile]
\frametitle{Lecture Contents}
\tableofcontents
\end{frame}

\section{Introduction}
\begin{frame}[fragile]
\frametitle{Introduction}
\begin{itemize}
	\item Graphs are indispensable in studying and presenting data and results
	\item They provide a lot of insight and hints on where to go next with data analysis
	\item As you are aware, there are graphs to display any type of data
	\item R has enormous and flexible capabilities to chart data
	\item In general, there are two command types related to graphical output in R: commands to create a basic plot, and commands to tweak the output to one's liking
	\item We will discuss also specialized packages that have lots of tweaking pre-programmed so that you don't need to spend time on it but focus on more important stuff
\end{itemize}
\end{frame}

\section{Base R Plotting Capabilities}
\begin{frame}[fragile]
\frametitle{The \cc{plot()} Function}
\begin{itemize}
	\item Generate some random and some not-so-random data:
	\begin{lstlisting}[style = rstyle, breaklines]
	x <- 1000*runif(50)
	y <- 0.8*x + rnorm(50, mean = 0, sd = 75)
	\end{lstlisting}
	\item We will start with scatterplots, as they are usually among the most frequently used ones
	\item To plot $y$ against $x$, just type:
	\begin{lstlisting}[style = rstyle, breaklines]
	plot(x,y)	
	\end{lstlisting}
	\item Usually we have the response variable on the vertical axis, and the variable that influences it -- on the horizontal axis
	\item The plot is not so impressive but does the job (for now)
	\item We will use it as a basis for expanding on R's charting capabilities
\end{itemize}
\end{frame}

\begin{frame}[fragile]
\frametitle{Tweaking the Basic Plots}
\begin{itemize}
	\item Let's create a data frame to see how data frame data is used in plots:
	\begin{lstlisting}[style = rstyle, breaklines]
	df1 <- as.data.frame(cbind(x,y))
	\end{lstlisting}
	\item Rename the variables to something more meaningful:
	\begin{lstlisting}[style = rstyle, breaklines]
	colnames(df1) <- c("inc", "cons")
	\end{lstlisting}
	\item \ldots and plot:
	\begin{lstlisting}[style = rstyle, breaklines]
	plot(df1$inc, df1$cons)
	\end{lstlisting}
	\item The first thing that looks ugly are the axis labels; a chart title is also missing
	\item This can be corrected by adding some options to the graph:
	\begin{lstlisting}[style = rstyle, breaklines]
	plot(df1$inc, df1$cons, xlab = "Aggregate income", ylab = "Aggregate consumption", main = "Some macroeconomic aggregates")
	\end{lstlisting}
\end{itemize}
\end{frame}

\begin{frame}[fragile]
\frametitle{Tweaking the Basic Plots (2)}
\begin{itemize}
	\item The scales of the two variables can be tweaked by respectively \cc{xlim} and \cc{ylim} as in:
	\begin{lstlisting}[style = rstyle, breaklines]
	plot(df1$inc, df1$cons, xlab = "Aggregate income", ylab = "Aggregate consumption", main = "Some macroeconomic aggregates", xlim = c(0,2000), ylim = c(0, 2000))
	\end{lstlisting}
	\item Sometimes one would want to change the type of plotting character (empty dot by default)
	\item There are many available options which can be found here: \url{https://www.r-bloggers.com/2021/06/r-plot-pch-symbols-different-point-shapes-in-r/}
	\item Symbols are selected with the \cc{pch} option, their size is controlled with the \cc{cex} option, and their color -- with the \cc{col} option; see the following example:
	\begin{lstlisting}[style = rstyle, breaklines]
	plot(df1$inc, df1$cons, xlab = "Aggregate income", ylab = "Aggregate consumption", main = "Some macroeconomic aggregates", pch = 20, cex = 2, col = "red")
	\end{lstlisting}
\end{itemize}
\end{frame}

\begin{frame}[fragile]
\frametitle{Tweaking the Basic Plots (3)}
\begin{itemize}
	\item There are actually many more graphics options that can be utilized
	\item There is a default set of such parameters that is contained in a pre-specified list in R
	\item This list can be called with the \cc{par()} function
	\item That same function allows to set the parameters so that they can be used as default
	\item The latter also prevents lots of typing when no (significant) changes of your graphs are necessary, and also makes code much neater
	\item In order to be able to go back to the original parameters, sometimes it is a good idea to save them in an object in your workspace:
	\begin{lstlisting}[style = rstyle, breaklines]
	saved_par <- par()
	\end{lstlisting}
\end{itemize}
\end{frame}

\begin{frame}[fragile]
\frametitle{Tweaking the Basic Plots (4)}
\begin{itemize}
	\item We will use an example to demonstrate setting parameters with \cc{par()}
	\item The following code shows some playing with it:
	\begin{lstlisting}[style = rstyle, breaklines]
	par(
  		bg = "lightgray", # background of chart
			bty = "l", # box type around chart
			cex = 2.5, # magnification of symbols
			cex.axis = 0.4, # size of axis symbols relative to cex
			cex.lab = 0.5, # size of labels relative to cex
			cex.main = 0.67, # size of main title relative to cex
			col = "darkorange", # colour of symbols
 			col.axis = "blue", # colour of axes symbols
			col.lab = "blue", # colour of axis labels
			col.main = "darkgreen", # colour of main title
			family = "serif",
			fig = saved_par$fig,
			fin = saved_par$fin,
			mar = c(4,4,2,2),
			pch = 20 # symbol for plotting
			)
	\end{lstlisting}
\end{itemize}
\end{frame}

\begin{frame}[fragile]
\frametitle{Tweaking the Basic Plots (5)}
\begin{itemize}
	\item Now plot your figure with:
	\begin{lstlisting}[style = rstyle, breaklines]
	plot(df1$inc, df1$cons, xlab = "Aggregate income", ylab = "Aggregate consumption", main = "Some macroeconomic aggregates")
	\end{lstlisting}
	\item To switch back to default parameters:
	\begin{lstlisting}[style = rstyle, breaklines]
	par(saved_par)
	\end{lstlisting}
	\item Let's now add a categorical variable to the data frame and name it ``country'':
	\begin{lstlisting}[style = rstyle, breaklines]
	f1 <- factor(sample(c(1:3),length(x),replace = T))
	df1$country <- f1
	\end{lstlisting}
	\item We can now plot the same data but colour each country differently:
	\begin{lstlisting}[style = rstyle, breaklines]
	plot(df1$inc, df1$cons, xlab = "Aggregate income", ylab = "Aggregate consumption", main = "Some macroeconomic aggregates", col = df1$country, pch = 20, cex = 2)
	\end{lstlisting}
\end{itemize}
\end{frame}

\begin{frame}[fragile]
\frametitle{Line Plots}
\begin{itemize}
	\item Still the \cc{plot()} command is used
	\item As scatterplots use sorted observations, it is not advisable to turn them into line plots
	\item Instead, we take an individual variable and plot it, setting some options, too
	\item Start with a blanc plot to which we will later on add the lines:
	\begin{lstlisting}[style = rstyle, breaklines]
	plot(df1$inc, type = "l", lty = 0, ylab = "BGN", main = "Income and consumption")
	\end{lstlisting}
	\item \cc{type = "l"} is the option that makes it a line chart
	\item \cc{lty} sets the type of line to use: can be specified as an integer (0=blank, 1=solid (default), 2=dashed, 3=dotted, 4=dotdash, 5=longdash, 6=twodash) or as character strings "blank", "solid", "dashed", "dotted", "dotdash", "longdash", or "twodash")
\end{itemize}
\end{frame}

\begin{frame}[fragile]
\frametitle{Line Plots (2)}
\begin{itemize}
	\item To add the first variable as a line:
	\begin{lstlisting}[style = rstyle, breaklines]
	lines(df1$inc, type = "l", lty = 1, lwd = 2, col = "red")
	\end{lstlisting}
	\item Here, \cc{lwd} sets line width
	\item To add another line to the same graph:
	\begin{lstlisting}[style = rstyle, breaklines]
	lines(df1$cons, type = "l", lty = 6, lwd = 2, col = "blue")
	\end{lstlisting}
	\item Finally, we might add a legend:
	\begin{lstlisting}[style = rstyle, breaklines]
	legend(40,970, c("Income","Consumption"), lty=c(1,6), lwd=c(2,2), col=c("red","blue"))
	\end{lstlisting}
\end{itemize}
\end{frame}

\begin{frame}[fragile]
\frametitle{Plotting Graphs of Functions}
\begin{itemize}
	\item The \cc{curve()} function is used for that purpose
	\item Take the following function:
	\begin{lstlisting}[style = rstyle, breaklines]
	fun1 <- function(x){
		3*x^3 + 2*x^2 - 7*x + 11
		}
	\end{lstlisting}
	\item \ldots and plot it:
	\begin{lstlisting}[style = rstyle, breaklines]
	curve(fun1, -10,10, lwd = 2, col = "red")
	\end{lstlisting}
	\item To add coordinate axes:
	\begin{lstlisting}[style = rstyle, breaklines]
	abline(h = 0, lty = 2)
	abline(v = 0, lty = 2)			
	\end{lstlisting}	
\end{itemize}
\end{frame}

\begin{frame}[fragile]
\frametitle{Plotting Graphs of Functions (2)}
\begin{itemize}
	\item Too add more functions, e.g.:
	\begin{lstlisting}[style = rstyle, breaklines]
	fun2 <- function(x){
		1000*cos(x)
		}
	curve(fun2, -10, 10, add = T, lwd = 2, col = "blue")
	\end{lstlisting}
	\item Pay attention to the \cc{add = T} option!
\end{itemize}
\end{frame}

\begin{frame}[fragile]
\frametitle{Pie Charts}
\begin{itemize}
	\item Created with the \cc{pie()} function
	\item Example:
	\begin{lstlisting}[style = rstyle, breaklines]
	pie_data <- c(19,24,28,17,35)
	pie_labels <- c("Apples", "Pears", "Cherries", "Oranges", "Bananas")
	pie(pie_data, pie_labels, col = rainbow(length(pie_data)), main = "Fruit consumption")
	\end{lstlisting}
	\item 3D pie charts are also possible, e.g. using the \textbf{plotrix} package:
	\begin{lstlisting}[style = rstyle, breaklines]
	library(plotrix)
	pie3D(pie_data, labels = pie_labels, explode = 0.2, radius = 0.8, main = "Fruit consumption in 3D", labelcex = 1, labelcol = "blue")
	\end{lstlisting}
\end{itemize}
\end{frame}

\begin{frame}[fragile]
\frametitle{Bar Plots}
\begin{itemize}
	\item Created with the \cc{barplot()} function
	\item Both vertical and horizontal bars can be produced
	\item An example of vertical bars:
	\begin{lstlisting}[style = rstyle, breaklines]
	barplot(df1$inc, border = "darkgreen", col = "orange", xlab = "Index", ylab = "BGN", main = "Income levels")
	\end{lstlisting}
	\item Same example but with horizontal bars:
	\begin{lstlisting}[style = rstyle, breaklines]
	barplot(df1$inc, horiz = T, border = "darkgreen", col = "orange", ylab = "Index", xlab = "BGN", main = "Income levels")
	\end{lstlisting}
	
\end{itemize}
\end{frame}

\begin{frame}[fragile]
\frametitle{Bar Plots (2)}
\begin{itemize}
	\item To plot two variables together:
	\begin{lstlisting}[style = rstyle, breaklines]
	bar_data1 <- sample(30:50, 3)
	bar_data2 <- sample(30:40, 3)
	barplot(cbind(bar_data1,bar_data2), col = c("darkred", "darkorange", "yellow"), names.arg = c("My variable", "Your variable"))
	\end{lstlisting}
	\item The same, with bars beside each other:
	\begin{lstlisting}[style = rstyle, breaklines]
	barplot(cbind(bar_data1,bar_data2), col = c("darkred", "darkorange", "yellow"), names.arg = c("My variable", "Your variable"), beside = T)		
	\end{lstlisting}
	\item Of course, there are many more parameters available
\end{itemize}
\end{frame}

\begin{frame}[fragile]
\frametitle{Histograms}
\begin{itemize}
	\item Generated by means of \cc{hist()} and its options
	\item Example:
	\begin{lstlisting}[style = rstyle, breaklines]
	z <- rnorm(500)
	hist(z, border = "darkblue", col = "orange", breaks = 20, freq = FALSE)
	\end{lstlisting}
	\item If \cc{freq = TRUE} then counts are displayed instead of the density which is output in the above example
	\item Check this one out too:
	\begin{lstlisting}[style = rstyle, breaklines]
	hist(z, border = "darkblue", col = "darkgreen", density = 40, breaks = 20, freq = FALSE)
	\end{lstlisting}
	\item Here, the \cc{density} option provides the density of shading lines used; their angle can be controlled with \cc{angle}
\end{itemize}
\end{frame}

\begin{frame}[fragile]
\frametitle{Box Plots}
\begin{itemize}
	\item Created with \cc{boxplot()}
	\item Simple example:
	\begin{lstlisting}[style = rstyle, breaklines]
	boxplot(df1$inc, col = "#D85625") # uses an HTML colour
	\end{lstlisting}
	\item The example from my ANOVA lecture:
	\begin{lstlisting}[style = rstyle, breaklines]
	avgbuy <- read.csv("three_stores.csv")
	boxplot(avgbuy, col = c("red","green","blue"))
	\end{lstlisting}
\end{itemize}
\end{frame}

\begin{frame}[fragile]
\frametitle{Pairs Plots}
\begin{itemize}
	\item Each pairs plot is actually a matrix of scatter plots
	\item Used when your data set contains more than two variables and you would like to explore visually the (possible) association between any two variables
	\item To illustrate, we will use Fisher's \texttt{iris} dataset which is readily available in R
	\item Graphs are produced with:
	\begin{lstlisting}[style = rstyle, breaklines]
	iris_df <- as.data.frame(iris)
	pairs(iris_df[,1:4], col = iris_df[,5])	
	\end{lstlisting}	
\end{itemize}
\end{frame}

\begin{frame}[fragile]
\frametitle{Plotting 3D Surfaces}
\begin{itemize}
	\item R has some core functionality in this respect
	\item It is realized through the \cc{persp()} function
	\item A simple example:
	\begin{lstlisting}[style = rstyle, breaklines]
	dome <- function(x,y){
		-(x^2 + y^2)
		}
	x <- seq(from=-3, to = 3, by=0.1)
	y <- seq(from=-3, to = 3, by=0.1)
	z <- outer(x,y,dome)
	persp(x,y,z,col="blue",theta=70,phi=-10)
	\end{lstlisting}
\end{itemize}
\end{frame}

\begin{frame}[fragile]
\frametitle{Multiple Plots in One Graph}
\begin{itemize}
	\item To do that, you need first to create a special matrix which will hold the plots
	\item In fact, this matrix already exists and you've been using it all along
	\item It is just of size $1 \times 1$
	\item What is necessary is to resize it
	\item This is done again with the \cc{par()} function using the \cc{mfrow} option
	\item For example, in order to create a graph that has four plots in it ($2 \times 2$):
	\begin{lstlisting}[style = rstyle, breaklines]
	par(mfrow = c(2,2))
	plot(rnorm(100), type = "p")
	plot(rnorm(100), type = "l")
	plot(rnorm(100), type = "s")
	plot(rnorm(100), type = "b")
	\end{lstlisting}
\end{itemize}
\end{frame}

\begin{frame}[fragile]
\frametitle{Multiple Plots in One Graph (2)}
\begin{itemize}
	\item What if you need an unequal number of plots in each row/column of the graph?
	\item This is achieved with the \cc{layout()} function
	\item Its argument is a matrix of integers in which each unique integer stands for a single object
	\item For example, to create a graph which has one plot in its first column and two plots in the second column, you use the following:
	\begin{lstlisting}[style = rstyle, breaklines]
	pos_m <- matrix(c(1,1,2,3), nrow = 2)
	layout(pos_m)
	plot(rnorm(100), type = "p")
	plot(rnorm(100), type = "l")
	plot(rnorm(100), type = "s")
	\end{lstlisting}
	\item To return to the single-plot layout you can for example type:
	\begin{lstlisting}[style = rstyle, breaklines]
	par(mfrow = c(1,1))
	\end{lstlisting}
\end{itemize}
\end{frame}

\begin{frame}[fragile]
\frametitle{Saving Graphs to Disk}
\begin{itemize}
	\item It is possible to export and save you graphical output to various graphical formats
	\item This is a very convenient feature which allows you to later on use graphs in your documents (paper, thesis, etc.)
	\item The functionality is provided by the \textbf{grDevices} package (comes with the base distribution)\footnote{See the full documentation here: \url{https://stat.ethz.ch/R-manual/R-devel/library/grDevices/html/00Index.html}.}
	\item Each export format is treated as a device
	\item Supported popular formats are BMP, JPEG, PNG, TIFF, PDF, Postscript, etc.
\end{itemize}
\end{frame}

\begin{frame}[fragile]
\frametitle{Saving Graphs to Disk (2)}
\begin{itemize}
	\item There are two ways of exporting and saving
	\item First, you turn on the relevant device:
	\begin{lstlisting}[style = rstyle, breaklines]
	png("linegr1.png", height=600, width=800)
	\end{lstlisting}
	\item Then you create the graph:
	\begin{lstlisting}[style = rstyle, breaklines]
	plot(rnorm(100), type = "l")
	\end{lstlisting}
	\item Finally, you turn off the device:
	\begin{lstlisting}[style = rstyle, breaklines]
	dev.off()
	\end{lstlisting}
	\item In this case, the graph is not visualised in R
\end{itemize}
\end{frame}

\begin{frame}[fragile]
\frametitle{Saving Graphs to Disk (3)}
\begin{itemize}
	\item The second approach boils down to copying and exporting the latest graphical output:
	\begin{lstlisting}[style = rstyle, breaklines]
	plot(rnorm(1000), type = "l", lwd = 1.5, col = "red", main = "Gaussian white noise")
	dev.copy2pdf(file = "wn.pdf", height=6, width=8)
	dev.copy2ps(file = "wn.eps", height=6, width=8)
	\end{lstlisting}
	\item Note that in the last two examples (pdf and ps) no \cc{dev.off()} is necessary
\end{itemize}
\end{frame}

\section{An introduction to the lattice package}
\begin{frame}[fragile]
\frametitle{The \textbf{lattice} Package}
\begin{itemize}
	\item R has a parallel graphics system called \textbf{grid}
	\item This system provides only low-level graphics functions but does not allow to produce complete plots
	\item \textbf{lattice} is one of the packages\footnote{\textbf{ggplot2} is the other one considered later.} that builds upon this system by providing high-level functions that allow creating complete graphs
	\item Load it with:
	\begin{lstlisting}[style = rstyle, breaklines]
	library(lattice)
	\end{lstlisting}
\end{itemize}
\end{frame}

\begin{frame}[fragile]
\frametitle{The \textbf{lattice} Package (2)}
\begin{itemize}
	\item The most basic plot is very similar to the ones considered with base R:
	\begin{lstlisting}[style = rstyle, breaklines]
	xyplot(cons ~ inc, data = df1)
	\end{lstlisting}
	\item Analogically, they can be tweaked further, e.g. through:
	\begin{lstlisting}[style = rstyle, breaklines]
	df1$index <- c(1:length(df1$cons))
	xyplot(cons ~ index, data = df1, type="o", lty = 2, pch=24, main="Income vs. consumption")
	\end{lstlisting}
	\item Bar charts are produced with \cc{barchart()}:
	\begin{lstlisting}[style = rstyle, breaklines]
	barchart(cons ~ index, horizontal = FALSE, data = df1, main="Consumption", col = "darkred")
	\end{lstlisting}
	\item Box (and whiskers) plots 
	\begin{lstlisting}[style = rstyle, breaklines]
	rnd1 <- rnorm(1000)
	bwplot(rnd1, col = "red", fill = "green")
	\end{lstlisting}
\end{itemize}
\end{frame}

\begin{frame}[fragile]
\frametitle{The \textbf{lattice} Package (3)}
\begin{itemize}
	\item Histograms and density plots
	\begin{lstlisting}[style = rstyle, breaklines]
	histogram(rnd1, col = "orange")
	densityplot(rnd1, lwd = 3)
	\end{lstlisting}
	\item Q-Q plots:
	\begin{lstlisting}[style = rstyle, breaklines]
	qqmath(rnd1)
	\end{lstlisting}
	\item Other capabilities of the \textbf{lattice} package may be explored by checking out the demo:
	\begin{lstlisting}[style = rstyle, breaklines]
	demo(lattice)
	\end{lstlisting}
	\item There is also a package called \textbf{latticeExtra} which (as its name tells) extends \textbf{lattice}'s capabilities and range of graphs produced (see \url{http://latticeextra.r-forge.r-project.org/#panel.quantile&theme=default})
\end{itemize}
\end{frame}

\section{An Introduction to the ggplot2 Package}
\begin{frame}[fragile]
\frametitle{The \textbf{ggplot2} Package}
\begin{itemize}
	\item As usual, in order to use it, first install it
	\item The name of the package comes from the title of a book whose conceptual models it implements (\emph{The Grammar of Graphics} by L. Wilkinson)
	\item There are two ways to create graphics using the \textbf{ggplot2} package:
	\begin{itemize}
		\item Using the \cc{qplot()} command -- to create quick plots
		\item Using the \cc{ggplot()} command (and the associated ones) to use the full potential of the package
	\end{itemize}
	\item We will discuss each of the two in turn
\end{itemize}
\end{frame}

\begin{frame}[fragile]
\frametitle{Using \cc{qplot()}}
\begin{itemize}
	\item \cc{qplot()} is similar in its application to \cc{plot()}
	\item One needs to specify only the data to be used:
	\begin{lstlisting}[style = rstyle, breaklines]
	qplot(cons, inc, data = df1, main = "Consumption vs. income")
	\end{lstlisting}
	\item Quick plots are pre-formatted with default package settings
	\item There are some more tweaking options unused above but it is a better idea, if you need further graph customization (or, in fact, in principle), to learn more and use \cc{ggplot()}
\end{itemize}
\end{frame}

\begin{frame}[fragile]
\frametitle{The \cc{ggplot()} Command}
\begin{itemize}
	\item \textbf{ggplot2} sticks, as already mentioned, to the paradigm of the \emph{The Grammar of Graphics} book
	\item This paradigm specifies that instead of using a separate function for each type of graph, graphs should be constructed by means of a small set of functions -- where each function produces a graph component
	\item In other words, graphs are being constructed by means of layers and layer elements, each one using a specific function and its options
	\item The first essential command is \cc{ggplot()}: it creates an empty plot
	\item In a way, you can think of the empty plot as of a painter's canvas with a prime (ground) layer applied to it
	\item When created, it waits for the painting to be created on it
\end{itemize}
\end{frame}

\begin{frame}[fragile]
\frametitle{The \cc{aes()} and \cc{geom()} Commands}
\begin{itemize}
	\item \cc{aes()} stands for \emph{aesthetics}
	\item Specifies how variables in the data are mapped to visual properties (aesthetics) of the so-called \emph{geoms}
	\item \emph{Geoms} themselves are the graphics shapes (lines, dots, bars, etc.) used to display the data
	\item The \cc{geom()} command in its various forms adds the geoms to the graph
	\bigskip\\
	\tikzstyle{block} = [rectangle, draw, fill=orange!40, 
    text width=5em, text centered, rounded corners, minimum height=4em]
\tikzstyle{line} = [draw, -latex']

\begin{center}
\begin{tikzpicture}[node distance = 2cm, auto]
    % Place nodes
    \node[block] (data) {data};
    \node[block, right of = data, node distance = 3cm] (aes) {aesthetic};
    \node[block, right of = aes, node distance = 3cm] (geom) {geom};
    % Draw edges
    \path [line] (data) -- (aes);
    \path [line] (aes) -- (geom);
\end{tikzpicture}
\end{center} 
\end{itemize}
\end{frame}

\begin{frame}[fragile]
\frametitle{A Simple Example}
\begin{itemize}
	\item Take once again our consumption and income example (with random data)
	\item To create the base layer of the graph and save the result in a graph object, type the following:
	\begin{lstlisting}[style = rstyle, breaklines]
	gg_graph1 <- ggplot(df1)
	\end{lstlisting}
	\item Then add the aesthetics together with the preferred shape
	\begin{lstlisting}[style = rstyle, breaklines]
	gg_graph1 <- gg_graph1 + geom_point(aes(x = inc, y = cons), size = 6, colour = "red")
	\end{lstlisting}
	\item By this, we are actually adding a new layer to the graph (using the plus operator)
	\item In the above, we also added some extra formatting options
\end{itemize}
\end{frame} 

\begin{frame}[fragile]
\frametitle{List of Geoms and Aesthetics}
%\lstinputlisting[style = rstyle, breaklines]{./data/table_commands.txt}
\begin{small}
\begin{tabular}{lll}
Geom & Description & Aesthetics\\
\hline
\cc{geom\_point()} & Data symbols & x, y, shape,fill\\
\cc{geom\_line()} & Line (ordered on x) & x, y, linetype\\
\cc{geom\_path()} & Line (original order) & x, y, linetype\\
\cc{geom\_text()} & Text labels & x, y, label, angle, hjust, vjust\\
\cc{geom\_rect()} & Rectangles & xmin, xmax, ymin, ymax, fill, linetype\\
\cc{geom\_polygon()} & Polygons & x, y, fill, linetype\\
\cc{geom\_segment()} & Line segments & x, y, xend, yend, linetype\\
\cc{geom\_bar()} & Bars & x, fill, linetype, weight\\
\cc{geom\_histogram()} & Histogram & x,fill,linetype,weight\\
\cc{geom\_boxplot()} & Boxplots & x, y, fill, weight\\
\cc{geom\_density()} & Density & x, y, fill, linetype\\
\cc{geom\_contour()} & Contour lines & x, y, fill, linetype\\
\cc{geom\_smooth()} & Smoothed line & x, y, fill, linetype\\
ALL & color, size, group & \\
\hline
\end{tabular}
\end{small}
\bigskip\\
(Table borrowed from Murrell (2012), p. 152)
\end{frame}

\begin{frame}[fragile]
\frametitle{Some More Examples with \cc{ggplot()}}
\begin{itemize}
	\item \cc{ggplot()} can be used to plot histograms and densities; for a demo, let's use the data found in the \texttt{iris} dataset
	\item The two sets of commands are as follows:
	\begin{lstlisting}[style = rstyle, breaklines]
	gg_hist <- ggplot(iris_df)
	gg_hist <- gg_hist + geom_histogram(aes(x = Petal.Length, color=Species, fill=Species), alpha=I(0.5))
		
	gg_dens <- ggplot(iris_df)
	gg_dens <- gg_dens + geom_density(aes(x = Petal.Length, color=Species, fill=Species, alpha=I(0.5)))
	\end{lstlisting}
	\item In order to make a box plot:
	\begin{lstlisting}[style = rstyle, breaklines]
	gg_box <- ggplot(iris_df)
	gg_box <- gg_box + geom_boxplot(aes(x = Species, y = Petal.Length, fill=Species, alpha=I(0.5)))
	\end{lstlisting}
	\item Here, \cc{alpha} controls transparency
	\item etc.
\end{itemize}
\end{frame}

\begin{frame}[fragile]
\frametitle{Setting Scales in \textbf{ggplot2}}
\begin{itemize}
	\item Scales concern axes and legends
	\item Usually scales are appropriately set by the \textbf{ggplot2} package so there is no real need to tweak the limits of variables' values
	\item However, through adjusting scales' parameters it is for example possible to change the automatically set axis labels
	\item An example:
	\begin{lstlisting}[style = rstyle, breaklines]
	gg_points <- ggplot(df1)
	gg_points <- gg_points + geom_point(aes(x = inc, y = cons, color = country, fill=country), size = I(6), alpha=I(0.5))
	gg_points <- gg_points + scale_x_continuous(name="Household income (EUR)") + 
          scale_y_continuous(name="Household consumption (EUR)") 
	\end{lstlisting}		
\end{itemize}
\end{frame}

\begin{frame}[fragile]
\frametitle{Setting Scales in \textbf{ggplot2} (2)}
\begin{itemize}
	\item If still a change in an axis' limits is needed, it is achieved through the \cc{limits = c(x\_value, y\_value)} option
	\item The colours of dots, respectively the legend, can be done in the following way:
	\begin{lstlisting}[style = rstyle, breaklines]
	gg_points <- gg_points +  scale_colour_manual(values = c("orange","darkblue", "darkgreen"))
	\end{lstlisting}
\end{itemize}

\tikzstyle{block} = [rectangle, draw, fill=orange!40, 
    text width=5em, text centered, rounded corners, minimum height=4em]
\tikzstyle{line} = [draw, -latex']
\begin{center}
\begin{tikzpicture}[node distance = 2cm, auto]
    % Place nodes
    \node[block] (data) {data};
    \node[block, right of = data, node distance = 3cm] (scale) {scale};
    \node[block, right of = scale, node distance = 3cm] (aes) {aesthetic};
    \node[block, right of = aes, node distance = 3cm] (geom) {geom};
    % Draw edges
    \path [line] (data) -- (scale);
    \path [line] (scale) -- (aes);
    \path [line] (aes) -- (geom);
\end{tikzpicture}
\end{center}
\end{frame}

\begin{frame}[fragile]
\frametitle{List of Scale Types and Parameters}
\begin{small}
\begin{tabular}{lll}
	Scale & Description & Parameters\\
	\hline
	\cc{scale\_x\_continuous()} & Continuous axis & expand, trans\\
	\cc{scale\_x\_discrete()} & Categorical axis\\
	\cc{scale\_x\_date()} & Date axis & major, minor, format\\
	\cc{scale\_shape()} & Symbol shape legend\\
	\cc{scale\_linetype()} & Line pattern legend\\
	\cc{scale\_color\_manual()} & Symbol/line color legend & values\\
	\cc{scale\_fill\_manual()} & Symbol/bar fill legend & values\\
	\cc{scale\_size()} & Symbol size legend & trans, to\\
	ALL & &	 name, breaks, labels, limits\\
	\hline
\end{tabular}
\end{small}
(Table borrowed from Murrell (2012), p. 157)
\end{frame}

\begin{frame}[fragile]
\frametitle{Statistical Transformations}
\begin{itemize}
	\item It is possible to map aesthetics not directly to raw (untransformed) data but to their transformations via statistical functions
	\item For example, this adds a polynomial regression estimate to the graph:
	\begin{lstlisting}[style = rstyle, breaklines]
	gg_points <- gg_points + stat_smooth(aes(x = inc, y = cons), method=lm, formula = y ~ poly(x,2), level=0.95)
	\end{lstlisting}
\end{itemize}
\end{frame}

\begin{frame}[fragile]
\frametitle{Facets}
\begin{itemize}
	\item Data can be broken into subsets and then a separate plot for each subset can be made
	\item For this purpose, \cc{facet\_wrap()} is used 
	\item This function requires a formula as an argument 
	\item The formula provides a description of the the variable that will be used to
subset the data
	\item We will use the income and consumption data frame to illustrate this
	\begin{lstlisting}[style = rstyle, breaklines]
	gg_facets <- ggplot(df1)
	gg_facets <- gg_facets + geom_point(aes(x=inc, y=cons), colour = "red", size = I(3))
	gg_facets <- gg_facets + facet_wrap(~ country, nrow=2)
	\end{lstlisting}
\end{itemize}
\end{frame}

\begin{frame}[fragile]
\frametitle{Themes}
\begin{itemize}
	\item As it already became clear, \textbf{ggplot2} separates graph elements into data and non-data ones
	\item The data-related elements are represented by geoms, and the appearance of geoms is controlled by aesthetics
	\item Themes are collections of graphical parameters to control non-data elements
	\item We show some examples of themes in code but we will not dig into the details
	\item Details, however, allow customization of individual theme elements 
\end{itemize}
\end{frame}

\begin{frame}[fragile]
\frametitle{Annotations}
\begin{itemize}
	\item Annotations are in general text labels placed over graphs for displaying additional information not directly inferable from the data
	\item Of course, this means placing an additional layer over the graph
	\item Check out the following two:
	\begin{lstlisting}[style = rstyle, breaklines]
	gg2 <- gg2 + geom_point(aes(x = Sepal.Width, y = Sepal.Length)) +
		geom_text(aes(x = Sepal.Width, y = Sepal.Length, label = Species, color = Species))
		
	gg2 <- gg2 + geom_point(aes(x = Sepal.Width, y = Sepal.Length)) + 
    	geom_label(aes(x = 		Sepal.Width, y = Sepal.Length, label = Species, color = Species))
	\end{lstlisting}
\end{itemize}
\end{frame}

\begin{frame}[fragile]
\frametitle{Annotations (2)}
\begin{itemize}
	\item An annotation layer can also be created in the following way:
	\begin{lstlisting}[style = rstyle, breaklines]
	gg3 <- ggplot(df1)
	gg3 <- gg3 + geom_point(aes(x = inc, y = cons), size = 6, colour = "red")
	gg3 <- gg3 + annotate("text", x = 100, y = 700, label = "A text annotation", size = I(6), color = "blue")

	gg3 <- gg3 + annotate("rect", xmin = 100, xmax =  300, ymin = 0, ymax = 300, alpha = 0.2)

	gg3 <- gg3 + annotate("segment", x = 600, xend =  900, y = 800, yend = 0, color = "darkgreen")
	\end{lstlisting}
\end{itemize}
\end{frame}

\section{An introduction to the ggvis package}
\begin{frame}[fragile]
\frametitle{The \textbf{ggvis} Package}
\begin{itemize}
	\item Created by the RStudio team, a quite new project
	\item Current version is 0.4 (far from version 1.0 but still very promising)
	\item Like \textbf{ggplot2}, it is based on the \emph{Grammar of Graphics} philosophy
	\item However, in addition it combines the above with the \textbf{Vega} model\footnote{\url{https://vega.github.io/vega/}} which allows to draw raster graphics in the HTML 5 canvas or vector graphics in the svg format using JavaScript
	\item This makes it possible to render graphics in a standard web browser while allowing interactive plots
	\item As RStudio in fact possesses many features of browsers, it is possible to visualize graphics directly in it
\end{itemize}
\end{frame}

\begin{frame}[fragile]
\frametitle{The \textbf{ggvis} Package (2)}
\begin{itemize}
	\item The full (available as of now) documentation can be found here: \url{http://ggvis.rstudio.com/}
	\item We will not consider the details of the package as they are not directly relevant to the programme courses
	\item We will only take a look at the online examples
\end{itemize}
\end{frame}

\section{An Introduction to the rbokeh Package}
\begin{frame}[fragile]
\frametitle{The \textbf{rbokeh} Package}
\begin{itemize}
	\item \textbf{rbokeh} is similar to \textbf{ggvis}
	\item Also quite new, current version 0.5.0
	\item Uses the \textbf{Bokeh} library which has interfaces to Python, Scala, R, and Julia
	\item Again, we will refrain from exploring its syntax, instead we will look again at the online examples: \url{https://hafen.github.io/rbokeh/}
	\item As you can see there, development is in its relatively early stages 
	\item Nevertheless, it is a promising project, too
\end{itemize}
\end{frame}

\section{Plotting time series}
\begin{frame}[fragile]
\frametitle{Plotting Time Series}
\begin{itemize}
	\item Plotting time series is in general not different from plotting other objects
	\item However, here we discuss it for two purposes:
	\begin{enumerate}
		\item To make a smooth transition to real-life data
		\item To make a quick review of plotting approaches using time series objects
	\end{enumerate}
	\item Start with places where many datasets can be found so that we can use them for illustration purposes: \url{https://www.quora.com/Where-can-I-find-large-datasets-open-to-the-public}
	\item We will for the examples the Quandl database
	\item This database has an R interface to directly download data to your R IDE
\end{itemize}
\end{frame}

\begin{frame}[fragile]
\frametitle{Quandl}
\begin{itemize}
	\item In order to explore the contents of Quandl, it is necessary to create an account at \url{https://www.quandl.com/}
	\item Other than that, you don't need login credentials to download data to R
	\item Install the package and load it:
	\begin{lstlisting}[style = rstyle, breaklines]
	library(Quandl)
	\end{lstlisting}
	\item Note that it also loads automatically \textbf{xts} (and, of course, \textbf{zoo})
	\item Let's download data on the price of gold from the Bundesbank database:
	\begin{lstlisting}[style = rstyle, breaklines]
	data1 <- Quandl("BUNDESBANK/BBK01_WT5511")
	\end{lstlisting}
	\item A dataframe is created
	\end{itemize}
\end{frame}

\begin{frame}[fragile]
\frametitle{Plotting the Data}
\begin{itemize}
	\item We can use standard plotting functionality to plot the data:
	\begin{lstlisting}[style = rstyle, breaklines]
	plot(data1$Date, data1$Value, type = "l")
	\end{lstlisting}
	\item The plot will be roughly correct except for the fact that it does not refer to `real' time series data
	\begin{lstlisting}[style = rstyle, breaklines]
	class(data1)
	\end{lstlisting}
	\item It is not a good idea to convert to a \textbf{ts} object as the data is of irregular frequency
	\item Therefore, it is better to have them in the \textbf{xts} format:
	\begin{lstlisting}[style = rstyle, breaklines]
	xts1 <- as.xts(data1$Value, order.by = data1$Date)
	names(xts1)[1] <- "gold_price"
	\end{lstlisting}
\end{itemize}
\end{frame}

\begin{frame}[fragile]
\frametitle{Plotting the Data (2)}
\begin{itemize}
	\item It turns out so that \textbf{xts} introduces its own plotting functionality after loading
	\item Check out:
	\begin{lstlisting}[style = rstyle, breaklines]
	plot(xts1[,1], col = "red", main = "Gold Price")
	\end{lstlisting}
	\item Actually, \cc{plot()} is in this case \cc{plot.xts()}
	\item Good, but not very customizable
	\item Let's plot the same series with \textbf{lattice}
	\begin{lstlisting}[style = rstyle, breaklines]
	library(lattice)
	xyplot(xts1, type=c("l","g"), xlab="",main="Gold Price")
	\end{lstlisting}
	\item Now create a new \textbf{xts} object containing also percentage changes besides levels:
	\begin{lstlisting}[style = rstyle, breaklines]
	xts2 <- merge(xts1[,1],diff.xts(log(xts1[,1])))
	colnames(xts2)[2] <- "change"
	\end{lstlisting}
\end{itemize}
\end{frame}

\begin{frame}[fragile]
\frametitle{Plotting the Data (3)}
\begin{itemize}
	\item To plot the two series together but in split plots:
	\begin{lstlisting}[style = rstyle, breaklines]
	xyplot(xts2, type=c("l","g"), strip = strip.custom(factor.levels = c("Gold Price, USD","Percentage change")), main="Gold Price")
	\end{lstlisting}
	\item To tweak it further, save the \textbf{lattice} parameters for later use with:
	\begin{lstlisting}[style = rstyle, breaklines]
	savepar <- trellis.par.get()
	\end{lstlisting}
	\item Then set the new parameters and plot:
	\begin{lstlisting}[style = rstyle, breaklines]
	trellis.par.set(strip.background = list(col="#0080ff"))
	xyplot(xts2, type=c("l","g"), strip = strip.custom(factor.levels = c("Gold Price, USD","Percentage change")), par.strip.text = list(col="white", font = 2), main="Gold Price")
	\end{lstlisting}
	\item To return to the old parameters:
	\begin{lstlisting}[style = rstyle, breaklines]
	trellis.par.set(savepar)
	\end{lstlisting}
\end{itemize}
\end{frame}

\begin{frame}[fragile]
\frametitle{Plotting the Data (4)}
\begin{itemize}
	\item The last plotting approach that will be considered explicitly is that using \textbf{ggplot2}
	\begin{lstlisting}[style = rstyle, breaklines]
	gg1 <- ggplot(data = xts2) + 
		geom_line(aes(x = Index, y = gold_price), colour = "#0080ff") + 
		theme_bw() +
		ggtitle("Gold price, USD") + 
		theme(plot.title = element_text(face="bold", colour = "red"))
	gg1

	gg2 <- ggplot(xts2) +
		geom_line(aes(x = Index, y = change), colour = "#0080ff") + 
		theme_bw() +
		ggtitle("Change, %") + 
		theme(plot.title = element_text(face="bold", colour = "red"))
	gg2
	\end{lstlisting}
\end{itemize}
\end{frame}

\begin{frame}[fragile]
\frametitle{Plotting the Data (5)}
\begin{itemize}
	\item With the \textbf{gridExtra} package, for example, it is possible to combine the above two graphs into a single one:
	\begin{lstlisting}[style = rstyle, breaklines]
	library(gridExtra)
	grid.arrange(gg1, gg2, ncol=1) 
	\end{lstlisting}
\end{itemize}
\end{frame}

\begin{frame}[fragile]
\frametitle{Plotting the Data (6)}
\begin{itemize}
	\item What if we want several series in a single plot?
	\item Let's first download some more data (exchange rates, unimportant otherwise):
	\begin{lstlisting}[style = rstyle, breaklines]
	data2 <- Quandl("BOE/XUDLGBD")
	data3 <- Quandl("BOE/XUDLADD")
	\end{lstlisting}
	\item Generate \textbf{xts} objects:
	\begin{lstlisting}[style = rstyle, breaklines]
	xts3 <- as.xts(data2$Value, order.by = data2$Date)
	xts4 <- as.xts(data3$Value, order.by = data3$Date)
	\end{lstlisting}
	\item Merge them and change column names:
	\begin{lstlisting}[style = rstyle, breaklines]
	xts5 <- merge.xts(xts3, xts4)
	colnames(xts5) <- c("GBP_USD", "AUD_USD")
	\end{lstlisting}
\end{itemize}
\end{frame}

\begin{frame}[fragile]
\frametitle{Plotting the Data (8)}
\begin{itemize}
	
	\item Make the graph:
	\begin{lstlisting}[style = rstyle, breaklines]
	ggplot(xts5)  +
		geom_line(aes(x = Index, y = GBP_USD, color = "GBP/USD")) +
		geom_line(aes(x = Index, y = AUD_USD, color = "AUD/USD")) +
		scale_color_manual("", values = c("darkred", "darkblue")) + 
		theme_bw() + 
		xlab("") + 
		ylab("") +
		annotate("rect", 
        	xmin = as.Date("2008-01-01"),
        	xmax = as.Date("2014-12-18"),
        	ymin = min(xts5[,1]),
        	ymax = max(xts5[,2]), fill = "orange", alpha = I(0.2))
	\end{lstlisting}
\end{itemize}
\end{frame}

\begin{frame}[fragile]
\frametitle{Don't Miss Out!}
\begin{center}
\Huge{Plotly!}
(\url{https://plot.ly/})
\end{center}
R package: \url{https://cran.r-project.org/web/packages/plotly/index.html}
\end{frame}

\begin{frame}[fragile]
\frametitle{References}
\begin{itemize}
	\item Sarkar, D. (2008): \emph{Lattice: Multivariate Data Visualization with R}, Springer
	\item Teutonico, D. (2015): \emph{ggplot2 Essentials}, Packt Publishing
	\item Wickham, H. (2009): \emph{ggplot2: Elegant Graphics for Data Analysis}, Springer
	\item Zuur, A., E. Ieno and E. Meesters (2009): \emph{A Beginner’s Guide to R}, Springer, Ch. 5
\end{itemize}
\end{frame}
\end{document}

http://finzi.psych.upenn.edu/library/rgl/html/persp3d.function.html
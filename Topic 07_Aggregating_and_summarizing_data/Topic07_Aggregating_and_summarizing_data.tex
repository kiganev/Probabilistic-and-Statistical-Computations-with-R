\documentclass[10pt]{beamer}
\usetheme{CambridgeUS}
%\usetheme{Boadilla}
\definecolor{myred}{RGB}{163,0,0}
%\usecolortheme[named=blue]{structure}
\usecolortheme{dove}
\usefonttheme[]{professionalfonts}
\usepackage[english]{babel}
\usepackage{amsmath,amsfonts,amssymb}
\usepackage{mathtools}
\usepackage{commath}
\usepackage{xcolor}
\usepackage{tikz}
\usepackage{pgfplots}
\pgfplotsset{compat=newest,compat/show suggested version=false}
\usetikzlibrary{arrows,shapes,calc,backgrounds}
\usepackage{bm}
\usepackage{textcomp}
%\usepackage{gensymb}
%\usepackage{verbatim}
\usepackage{mathrsfs}  
\usepackage{paratype}
\usepackage{mathpazo}
\usepackage{listings}
\usepackage{csvsimple}
\usepackage{booktabs}
\usepackage{longtable}

\newcommand{\cc}[1]{\texttt{\textcolor{blue}{#1}}}

\DeclareMathOperator{\adj}{adj}
\DeclareMathOperator{\tr}{tr}
\DeclareMathOperator{\diag}{diag}
\DeclareMathOperator{\E}{\mathsf{E}}
\DeclareMathOperator{\var}{\mathsf{Var}}
\DeclareMathOperator{\cov}{\mathsf{Cov}}
\DeclareMathOperator{\corr}{\mathsf{Corr}}
\DeclareMathOperator{\plim}{plim}
\DeclareMathOperator{\rank}{rank}
\DeclareMathOperator{\mad}{MAD}
\DeclareMathOperator{\med}{median}


\definecolor{ttcolor}{RGB}{0,0,1}%{RGB}{163,0,0}
\definecolor{mygray}{RGB}{248,249,250}

% Number theorem environments
\setbeamertemplate{theorem}[ams style]
\setbeamertemplate{theorems}[numbered]

% Reset theorem-like environments so that each is numbered separately
\usepackage{etoolbox}
\undef{\definition}
\theoremstyle{definition}
\newtheorem{definition}{\translate{Definition}}

% Change colours for theorem-like environments
\definecolor{mygreen1}{RGB}{0,96,0}
\definecolor{mygreen2}{RGB}{229,239,229}
\setbeamercolor{block title}{fg=white,bg=mygreen1}
\setbeamercolor{block body}{fg=black,bg=mygreen2}

\lstdefinestyle{numbers}{numbers=left, stepnumber=1, numberstyle=\tiny, numbersep=10pt}
\lstdefinestyle{MyFrame}{backgroundcolor=\color{yellow},frame=shadowbox}

\lstdefinestyle{rstyle}%
{language=R,
	basicstyle=\footnotesize\ttfamily,
	backgroundcolor = \color{mygray},
	commentstyle=\slshape\color{green!50!black},
	keywordstyle=\color{blue},
	identifierstyle=\color{blue},
	stringstyle=\color{orange},
	%escapechar=\#,
	rulecolor = \color{mygray}, 
	showstringspaces = false,
	showtabs = false,
	tabsize = 2,
	emphstyle=\color{red},
	frame = single}

\lstset{language=R,frame=single}    

\hypersetup{colorlinks, urlcolor=blue, linkcolor = myred}

\AtBeginSection{\frame{\sectionpage}}

% Remove Section 1, Section 2, etc. as titles in section pages
\defbeamertemplate{section page}{mine}[1][]{%
	\begin{centering}
		{\usebeamerfont{section name}\usebeamercolor[fg]{section name}#1}
		\vskip1em\par
		\begin{beamercolorbox}[sep=12pt,center]{part title}
			\usebeamerfont{section title}\insertsection\par
		\end{beamercolorbox}
	\end{centering}
} 

\setbeamertemplate{section page}[mine] 

\beamertemplatenavigationsymbolsempty

\title{R403: Probabilistic and Statistical Computations\\ with R}
\subtitle{Topic 7: \textcolor{myred}{Aggregating and Summarizing Data}}
\author{Kaloyan Ganev}

\date{2022/2023} 

\begin{document}
\maketitle
	
\begin{frame}[fragile]
\frametitle{Lecture Contents}
	\tableofcontents
\end{frame}

\section{Introduction}
\begin{frame}[fragile]
\frametitle{Introductory Notes}
\begin{itemize}
	\item R has a lot of built-in functions that could be useful in aggregating and summarizing data
	
	\item We will make a brief tour through some of the most commonly used ones
	
	\item A part of the discussion will naturally contain some review of concepts already introduced
\end{itemize}
\end{frame}

\section{Summary Commands and Descriptive
Statistics}\label{summary-commands-and-descriptive-statistics}

\begin{frame}[fragile]
\frametitle{Summary Commands and Descriptive Statistics}
\begin{itemize}
	\item Provide information which might suggest which analytical procedure would be most appropriate to apply to data
	
	\item With their aid, better understanding of the properties of data is achieved
	
	\item Also, they might point to data issues requiring additional pre-processing, cleaning, etc.
\end{itemize}
\end{frame}

\begin{frame}[fragile]
\frametitle{Summary Commands}
\begin{itemize}
	\item A good idea to start is to use the \cc{ls()} command to get the list of all named objects in memory
	
	\item Of course, you can view the objects also in the Environment browser of RStudio
	
	\item Let's use some data so that we exemplify with it
	
	\item Load the data on new residential buildings (\emph{residential.csv})
	\begin{lstlisting}[style = rstyle, breaklines]
	residential <- read.csv2("residential.csv", header = TRUE, skip = 4, blank.lines.skip = TRUE)
	\end{lstlisting}

	\item We can have a look at the data by printing the data frame in the console or displaying it using RStudio's point-and-click tools
\end{itemize}
\end{frame}

\begin{frame}[fragile]
\frametitle{Summary Commands (2)}
\begin{itemize}
	\item If the data set is large, though, the data would be difficult to grasp
	
	\item Using the \cc{str()} command, you can get a concise summary of the characteristics of the data:
	\begin{lstlisting}[style = rstyle, breaklines]
	str(residential)
	\end{lstlisting}
	
	\item Of course, the same result is obtainable by clicking on the blue arrow next to the name of the data frame in the environment browser
	
	\item Note that \cc{str()} is specifically suited to explore the structure of the object of interest
\end{itemize}
\end{frame}

\begin{frame}[fragile]
\frametitle{Summary Commands (3)}
\begin{itemize}
	\item In order to get a statistical summary, just type:
	\begin{lstlisting}[style = rstyle, breaklines]
	summary(residential)
	\end{lstlisting}

	\item There might be nuances of this command's output when different object types are involved (character vectors, factors, matrices, models, etc.)
	
	\item Some more summary commands for lists and tables:
	\begin{lstlisting}[style = rstyle, breaklines]
	names() # Returns the names of list elements/data frame columns
	\end{lstlisting}

	\item \ldots and some such for vectors, matrices, arrays, and data frames:
	\begin{lstlisting}[style = rstyle, breaklines]
	names()
	colnames()
	rownames()
	dimnames()
	\end{lstlisting}
\end{itemize}
\end{frame}

\begin{frame}[fragile]
\frametitle{Summary Statistics for Vectors}
\begin{itemize}
	\item Maximum value:
	\begin{lstlisting}[style = rstyle, breaklines]
	max(x, na.rm = FALSE)
	\end{lstlisting}

	\item Usually missing observations should be removed from the calculation, therefore \cc{na.rm = TRUE} is used
	
	\item Otherwise, you will get an \emph{NA} value for a maximum
	
	\item Minimum value:
	\begin{lstlisting}[style = rstyle, breaklines]
	min(x, na.rm = FALSE) # Same logic as above
	\end{lstlisting}
\end{itemize}
\end{frame}

\begin{frame}[fragile]
\frametitle{Summary Statistics for Vectors (2)}
\begin{itemize}
	\item Length of vector:
	\begin{lstlisting}[style = rstyle, breaklines]
	length(x)
	\end{lstlisting}

	\item Note that \emph{NA}'s are also values so they still count against the length of a vector (to omit \emph{NA}'s, use \cc{length(na.omit(x))})
	
	\item Sum of all vector elements:
	\begin{lstlisting}[style = rstyle, breaklines]
	sum(x, na.rm = FALSE)
	\end{lstlisting}
	
	\item Arithmetic mean, median, standard deviation, and variance:
	\begin{lstlisting}[style = rstyle, breaklines]
	mean(x, na.rm = FALSE)
	median( x, na.rm = FALSE)
	sd(x, na.rm = FALSE)
	var(x, na.rm = FALSE)
	\end{lstlisting}
\end{itemize}
\end{frame}

\begin{frame}[fragile]
\frametitle{Summary Statistics for Vectors (3)}
\begin{itemize}
	\item Median absolute deviation (a robust\footnote{Meaning resilient to outliers in the present case.} measure of dispersion, sometimes better than the standard deviation):
	\[
		\mad(X) = \med(|X_{i} - \med(X)|)
	\]
	
	\begin{lstlisting}[style = rstyle, breaklines]
	mad(x, na.rm = FALSE)
	\end{lstlisting}
\end{itemize}
\end{frame}

\begin{frame}[fragile]
\frametitle{Summary Statistics for Vectors (4)}
\begin{itemize}
	\item Quantiles:
	\begin{lstlisting}[style = rstyle, breaklines]
	quantile(x) # The default 0, 25, 50, 75, and 100% quantiles
	\end{lstlisting}
	
	\item If you need to calculate different quantiles (e.g.~33, 66, and 99\%):
	\begin{lstlisting}[style = rstyle, breaklines]
	quantile(x, probs = c(0.33, 0.66, 0.99), na.rm = TRUE, names = TRUE)
	\end{lstlisting}

	\item Cumulative sum, product, maximum, and minimum:
	\begin{lstlisting}[style = rstyle, breaklines]
	cumsum(x)
	cumprod(x)
	cummax(x)
	cummin(x)
	\end{lstlisting}
\end{itemize}
\end{frame}

\begin{frame}[fragile]
\frametitle{Summary Statistics for Data Frames}
\begin{itemize}
	\item We've already discussed \cc{summary()}
	
	\item The following are also clear: \cc{min(),\ max(),\ length(),\ sum()}
	
	\item It also happened that we used \cc{rowSums()} and \cc{colSums()}; \cc{colMeans()} and \cc{rowMeans()} are analogical
\end{itemize}
\end{frame}

\begin{frame}[fragile]
\frametitle{Summary Statistics for Matrices}
\begin{itemize}
	\item To apply a statistical function to a part of a matrix, e.g. a column or a row:
	\begin{lstlisting}[style = rstyle, breaklines]
	m1 <- matrix(1:24, nrow = 4)
	mean(m1[3,])
	mean(m1[,2])
	\end{lstlisting}

	\item \cc{colMeans()} and \cc{rowMeans()} can do this job, too
	
	\item The commands \cc{rowSums()} and \cc{colSums()} also work for matrices
\end{itemize}
\end{frame}

\begin{frame}[fragile]
\frametitle{Summary Statistics for Lists}
\begin{itemize}
	\item Not straightforward to have such ({why?})
	
	\item \$ notation can be used but it may be somewhat inconvenient
	
	\item Still, there some ways to do that, but we will leave matters for the topic on loops and loop commands
\end{itemize}
\end{frame}

\section{Tabulation}
\begin{frame}[fragile]
\frametitle{Summary Tables}
\begin{itemize}
	\item In order to summarize data samples, the \cc{table()} command can be used
	
	\item It also allows creating, altering, and manipulating table objects including two types of special tables:
	\begin{itemize}
  	  \item Contingency tables
  	  \item Complex (flat) contingency tables
    \end{itemize}
\end{itemize}
\end{frame}

\begin{frame}[fragile]
\frametitle{Table Summaries for Vectors}
\begin{itemize}
	\item Let's simulate tossing a coint 1000000 times
	\begin{lstlisting}[style = rstyle, breaklines]
	coin <- sample(c("H", "T"), 1000000)
	\end{lstlisting}
	
	\item We can see a summary of counts using \cc{table()} again:
	\begin{lstlisting}[style = rstyle, breaklines]
	table2 <- table(coin)
	\end{lstlisting}
\end{itemize}
\end{frame}

\begin{frame}[fragile]{Contingency Tables}
\begin{itemize}
	\item The term was coined by Karl Pearson in 1904
	
	\item Also called cross-tabs/two-way tables
	
	\item Practically a table of counts; used to display the multivariate frequency distribution of the selected categorical variables
	
	\item In a sense, it is the analogue to scatter plots for continuous variables
	
	\item By this, they provide intuition on the potential statistical relationships among variables
	
	\item A complex (flat) table is a contingency table that is used to compress multiple dimensions of data to two dimensions only and a single table
\end{itemize}

\end{frame}
\begin{frame}[fragile]
\frametitle{An Example Simple Contingency Table}
\begin{longtable}[]{@{}lllll@{}}
\toprule
Gender / Hair colour & Black hair & Brown hair & Blond hair & Total\tabularnewline
\midrule
\endhead
Male & 70 & 25 & 40 & 135\tabularnewline
Female & 30 & 20 & 25 & 75\tabularnewline
Total & 100 & 45 & 65 & 210\tabularnewline
\bottomrule
\end{longtable}

\begin{itemize}
	\item In this table, there are two variables: gender and hair colour
	
	\item The former determines the row categories, and the latter determines the column categories
	
	\item Each combination of row and column is called a \textbf{cell}
\end{itemize}
\end{frame}

\begin{frame}[fragile]
\frametitle{Requirements for Contingency Tables}
\begin{itemize}
	\item The requirements stem from the specifics of statistical methods applied to such tables
	
	\item First, observations should be \textbf{independent} from each other
	
	\item Second, categories should be \textbf{exclusive}, i.e.~it is possible for an observation to fall only in one of the categories; in other words, categories cannot overlap
	
	\item Third, categories should be \textbf{exhaustive}, i.e.~the full range of possibilities is represented making it impossible for an observation to fail falling in a category
\end{itemize}
\end{frame}

\begin{frame}[fragile]
\frametitle{How to Create a Contingency Table in R}
\begin{itemize}
	\item Get the values of the two variables in two vectors (here we make a new random sample so the numbers in above table will not be repeated):
	\begin{lstlisting}[style = rstyle, breaklines]
	gender <- sample(c("Male", "Female"), 210, replace = TRUE)
	hair_col <- sample(c("Black", "Brown", "Blond"), 210, replace = TRUE)
	\end{lstlisting}
	
	\item Make a data frame and then create the contingency table from it:
	\begin{lstlisting}[style = rstyle, breaklines]
	data1 <- as.data.frame(cbind(gender, hair_col))
	table1 <- table(data1$gender, data1$hair_col)
	\end{lstlisting}
	
	\item In general, this can be done for any data frame containing categorical variables
\end{itemize}
\end{frame}

\begin{frame}[fragile]
\frametitle{Flat Contingency Tables}
\begin{itemize}
	\item Let's take another example
	
	\item Assume you gather a random sample of data on educational degree, salary and productivity of several workers (20 in this example)
	
	\item We generate the data at random with some prior restriction with the following code:
	\begin{lstlisting}[style = rstyle, breaklines]
	degree <- sample(c("BSc", "MSc", "PhD"), 20, replace = TRUE)
	salary <- sample(c("high", "medium size", "low"), 20, replace = TRUE)
	productivity <- sample(seq(120,160, by = 0.5), 20, replace = TRUE)
	\end{lstlisting}
\end{itemize}
\end{frame}

\begin{frame}[fragile]
\frametitle{Flat Contingency Tables (2)}
\begin{itemize}
	\item Let's get those data into a data frame:
	\begin{lstlisting}[style = rstyle, breaklines]
	df1 <- as.data.frame(cbind(degree, productivity, salary))
	\end{lstlisting}
	
	\item There are some issues of the data type of productivity which we correct in the following way:
	\begin{lstlisting}[style = rstyle, breaklines]
	df1$productivity <- as.numeric(as.character(df1$productivity))
	\end{lstlisting}

	\item Now, let's use the \cc{table()} command to produce a summary:
	\begin{lstlisting}[style = rstyle, breaklines]
	table(df1)
	\end{lstlisting}

	\item Because of the several dimensions, several tables are produced
\end{itemize}
\end{frame}

\begin{frame}[fragile]
\frametitle{Flat Contingency Tables (3)}
\begin{itemize}
	\item In order to get just one table instead of the many, we can create a flat contingency table that squeezes everything in two dimensions
	
	\item Flat contingency tables are created using the \cc{ftable()} command
	
	\item For example:
	\begin{lstlisting}[style = rstyle, breaklines]
	ftable(df1)
	\end{lstlisting}

	\item You can now take a close look at the table that is produced and make make a comparison with the previous \cc{table()} output
\end{itemize}
\end{frame}

\begin{frame}[fragile]
\frametitle{Proportions Tables}
\begin{itemize}
	\item Can be generated on data frames, matrices or tables
	
	\item Note that data frames should contain only numerical variables in order not to get an error
	
	\item To begin with, let's use \emph{table1} which is already available
	
	\item Type and run:
	\begin{lstlisting}[style = rstyle, breaklines]
	prop.table(table1)
	\end{lstlisting}

	\item The command outputs proportions (frequencies) as shares of the grand total
	
	\item You can also get proportions as shares of row totals or column totals, respectively by running:
	\begin{lstlisting}[style = rstyle, breaklines]
	prop.table(table1, margin = 1)
	prop.table(table1, margin = 2)
	\end{lstlisting}
\end{itemize}
\end{frame}

\begin{frame}[fragile]
\frametitle{Proportions Tables (2)}
\begin{itemize}
	\item See what happens when the same is applied to a numerics-only data frame
	
	\item Generate the following salary vector:
	\begin{lstlisting}[style = rstyle, breaklines]
	salary2 <- sample(seq(2500,3500, by = 0.01), 20, replace = TRUE)
	\end{lstlisting}
	
	\item Make the data frame:
	\begin{lstlisting}[style = rstyle, breaklines]
	df2 <- as.data.frame(cbind(productivity, salary2))
	\end{lstlisting}

	\item Check out:
	\begin{lstlisting}[style = rstyle, breaklines]
	prop.table(as.matrix(df2))
	\end{lstlisting}

	\item Try with the two other options
\end{itemize}
\end{frame}

\begin{frame}[fragile]
\frametitle{Further Readings}
\begin{itemize}
	\item Gardener, M. (2012): \emph{Beginning R: The Statistical Programming Language}, Wiley (ch.~4)

	\item Spector's book (see syllabus), p.~101-107
	
	\item R documentation and help
\end{itemize}
\end{frame}
\end{document}

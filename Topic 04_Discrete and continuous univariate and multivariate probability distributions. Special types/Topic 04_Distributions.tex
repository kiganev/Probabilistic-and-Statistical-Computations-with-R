\documentclass[10pt]{beamer}
\usetheme{CambridgeUS}
%\usetheme{Boadilla}
\definecolor{myred}{RGB}{163,0,0}
%\usecolortheme[named=blue]{structure}
\usecolortheme{dove}
\usefonttheme[]{professionalfonts}
\usepackage[english]{babel}
\usepackage{amsmath,amsfonts,amssymb}
\usepackage{mathtools}
\usepackage{commath}
\usepackage{xcolor}
\usepackage{tikz}
\usepackage{pgfplots}
\pgfplotsset{compat=newest,compat/show suggested version=false}
\usetikzlibrary{arrows,shapes,calc,backgrounds}
\usepackage{bm}
\usepackage{textcomp}
%\usepackage{gensymb}
%\usepackage{verbatim}
\usepackage{mathrsfs}  
\usepackage{paratype}
\usepackage{mathpazo}
\usepackage{listings}
\usepackage{csvsimple}
\usepackage{booktabs}

\newcommand{\cc}[1]{\texttt{\textcolor{blue}{#1}}}

\DeclareMathOperator{\adj}{adj}
\DeclareMathOperator{\tr}{tr}
\DeclareMathOperator{\diag}{diag}
\DeclareMathOperator{\E}{\mathsf{E}}
\DeclareMathOperator{\var}{\mathsf{Var}}
\DeclareMathOperator{\cov}{\mathsf{Cov}}
\DeclareMathOperator{\corr}{\mathsf{Corr}}
\DeclareMathOperator{\plim}{plim}
\DeclareMathOperator{\rank}{rank}


\definecolor{ttcolor}{RGB}{0,0,1}%{RGB}{163,0,0}
\definecolor{mygray}{RGB}{248,249,250}

% Number theorem environments
\setbeamertemplate{theorem}[ams style]
\setbeamertemplate{theorems}[numbered]

% Reset theorem-like environments so that each is numbered separately
\usepackage{etoolbox}
\undef{\definition}
\theoremstyle{definition}
\newtheorem{definition}{\translate{Definition}}

% Change colours for theorem-like environments
\definecolor{mygreen1}{RGB}{0,96,0}
\definecolor{mygreen2}{RGB}{229,239,229}
\setbeamercolor{block title}{fg=white,bg=mygreen1}
\setbeamercolor{block body}{fg=black,bg=mygreen2}

\lstdefinestyle{numbers}{numbers=left, stepnumber=1, numberstyle=\tiny, numbersep=10pt}
\lstdefinestyle{MyFrame}{backgroundcolor=\color{yellow},frame=shadowbox}

\lstdefinestyle{rstyle}%
{language=R,
	basicstyle=\footnotesize\ttfamily,
	backgroundcolor = \color{mygray},
	commentstyle=\slshape\color{green!50!black},
	keywordstyle=\color{blue},
	identifierstyle=\color{blue},
	stringstyle=\color{orange},
	%escapechar=\#,
	rulecolor = \color{mygray}, 
	showstringspaces = false,
	showtabs = false,
	tabsize = 2,
	emphstyle=\color{red},
	frame = single}

\lstset{language=R,frame=single}    

\hypersetup{colorlinks, urlcolor=blue, linkcolor = myred}

\AtBeginSection{\frame{\sectionpage}}

% Remove Section 1, Section 2, etc. as titles in section pages
\defbeamertemplate{section page}{mine}[1][]{%
	\begin{centering}
		{\usebeamerfont{section name}\usebeamercolor[fg]{section name}#1}
		\vskip1em\par
		\begin{beamercolorbox}[sep=12pt,center]{part title}
			\usebeamerfont{section title}\insertsection\par
		\end{beamercolorbox}
	\end{centering}
} 

\setbeamertemplate{section page}[mine] 

\beamertemplatenavigationsymbolsempty

\title{R403: Probabilistic and Statistical Computations\\ with R}
\subtitle{Topic 4: \textcolor{myred}{Discrete and continuous univariate and multivariate probability distributions. Special types. Working with probability distributions in R}}
\author{Kaloyan Ganev}

\date{2022/2023} 

\begin{document}
\maketitle

\begin{frame}[fragile]
\frametitle{Lecture Contents}
\tableofcontents
\end{frame}

\section{Introduction}
\begin{frame}[fragile]
\frametitle{Introduction}
\begin{itemize}
	\item For a beginning, we will review briefly some basics of probability
		
	\item We will then go through some of the probability distributions that are most used in economics
		
	\item At the same time, the corresponding R functions will also be discussed
\end{itemize}
\end{frame}

\section{A Brief and Superficial Review of Probability}
\begin{frame}[fragile]
\frametitle{Sample Space. Events}
\begin{itemize}
	\item Assume that outcomes (e.g. of an experiment) are unpredictable in advance
	
	\item Yet, the set of all possible outcomes is known. This set is called the \textit{sample space} (denoted by $ S $)
		
	\color{blue}		
	Examples: tossing a coin, rolling a die, etc.
		
	\color{black}
	\item Any subset of $ S $ is called an \textit{event}
	
	\color{blue}
	Examples: get heads when a coin is tossed; get heads on two coins tossed, etc.
		
	\color{black}
\end{itemize}
\end{frame}

\begin{frame}[fragile]
\frametitle{Sample Space. Events (2)}
\begin{itemize}
	\item Union of events:
	\[
		A \cup B; \quad \bigcup_{i=1}^{n}A_{i}
	\]
	
	\item Intersection of events:
	\[
		A \cap B; \quad \bigcap_{i=1}^{n}A_{i}
	\]
	
	\item Mutually exclusive events:
	\[
		A \cap B = \emptyset
	\]
		
	\item Complement of an event $ A $:
	\[
		\overline A : \textrm{all elements of } S \textrm{ that are not in } A
	\]
\end{itemize}
\end{frame}

\begin{frame}[fragile]
\frametitle{Probability Defined on Events}
\begin{itemize}
	\item Let $ A $ be any event from $ S $
		
	\item Then, probability is a number $ P(A) $ defined for $ A $ such that
	\begin{enumerate}
		\item $ 0 \leq P(A) \leq 1 $
			
		\item $ P(S) = 1 $
			
		\item If $ A_{1}, A_{2}, \ldots $ is a sequence of events and $ A_{i}\cap A_{j} = \emptyset, \ i \neq j $, then
		\[
			P\left(\bigcup_{i=1}^{\infty}A_{i}\right) = \sum_{i=1}^{\infty} P(A_{i}) 
		\]
	\end{enumerate}
	
	\item At the intuitive level, the probability of an event is the frequency of its occurrence
\end{itemize}
\end{frame}

\begin{frame}[fragile]
\frametitle{Probability Defined on Events (2)}
\begin{itemize}
	\item It is always true that $ A \cap \overline{A} = \emptyset $ and $ A \cup \overline{A} = S $
		
	\item Therefore,
	\[
		P(A \cup \overline{A}) = P(A) + P(\overline{A}) = P(S) = 1 
	\]
	
	\item The latter implies
	\[
		P(A) = 1 - P(\overline{A})
	\]
		
	\item Probability of union of two events:\footnote{This can be generalized to more than two events but the formula is spared here.}
	\[
		P(A \cup B) = P(A) + P(B) - P(A\cap B)
	\]
		
	\item For mutually exclusive events the latter becomes
	\[
		P(A \cup B) = P(A) + P(B)
	\]
		
\end{itemize}
\end{frame}

\begin{frame}[fragile]
\frametitle{Conditional Probability}
\begin{itemize}
	\item Take events $ A $ and $ B $; the probability
	\[
		P(B|A)
	\]
	is the probability of $ B $ given that $ A $ already occurred
	
	\item In fact, $ A $ is a subset of elements of the entire sample set
	
	\item Here, as the condition is that specifically $ A $ occurred, $ A $ itself becomes the sample set
	
	\item So in order to asses the conditional probability of $ B $ given $ A $, we have to calculate the frequency of $ B \cap A (= A\cap B) $ occurring relative to the frequency of $ A $ occurring, i.e.
	\[
		P(B|A)  = \dfrac{P(A \cap B)}{P(A)}, \quad P(A) > 0
	\]
\end{itemize}
\end{frame}

\begin{frame}[fragile]
\frametitle{Conditional Probability (2)}
\begin{itemize}
	\item The latter implies
	\[
		P(A\cap B)  = P(B|A)P(A) = P(A|B)P(B)
	\]
	
	\item It is also then true that
	\[
		\begin{array}{lcl}
			P(B|A) \geq 0\\
			\quad\\
			P(A|A) = 1\\
			\quad\\
			\displaystyle P(\bigcup_{i=1}^{\infty} B_{i}|A ) = \sum_{i = 1}^{\infty} P(B_{i}|A),\quad \textrm{if } B_{1}, B_{2},\ldots \textrm{ are mutually exclusive}
		\end{array}
	\]
\end{itemize}
\end{frame}

\begin{frame}[fragile]
\frametitle{Independence}
\begin{itemize}
	\item $ A $ and $ B $ are independent if
	\[
		P(A\cap B) = P(A)P(B)
	\]
		
	\item This implies that $ A $ and $ B $ are independent if
	\[
		P(B|A) = P(B)
	\]
		
	\item If $ A $ and $ B $ are not independent, they are called \textit{dependent}
\end{itemize}
\end{frame}

\begin{frame}[fragile]
	\frametitle{The Bayes Formula}
	\begin{itemize}
		\item From $ P(A\cap B)  = P(B|A)P(A) = P(A|B)P(B) $ follows that
		\[
			P(A|B) = \dfrac{P(B|A)P(A)}{P(B)}, \quad P(B) \neq 0 \quad \textrm{(Bayes formula)}
		\]
		
		\item Law of total probability:  if $ \displaystyle \left\{B_{i}:\ i=1,2,3,\ldots \right\}$ is a finite or countable partition of a sample space (i.e. all $ B_{i} $ are pairwise disjoint), then if $ A $ is on the same probability space,
		\[
			P(A) = \sum_{i} P(A|B_{i})P(B_{i})
		\]
		
		\item Using this, the Bayes formula can be written also as
		\[
			P(A|B) = \dfrac{P(B|A)P(A)}{\displaystyle\sum_{i} P(B|A_{i})P(A_{i})}
		\]
	\end{itemize}
\end{frame}

\section{Random Variables}
\begin{frame}[fragile]
\frametitle{Random Variables}
\begin{itemize}
	\item Random variables are functions that map events from a sample space to real numbers
		
	\item Indicator random variable:
	\[
		I = 
		\left\{
			\begin{array}{lcl}
				1, \quad \textrm{if a property is present}\\
				0, \quad \textrm{if a property is absent}\\
			\end{array}
		\right.
	\]
		
	\item \textbf{Discrete} random variable: can take on a finite or countable number of values
		
	\item \textbf{Continuous} random variable: can take on a continuum of values
\end{itemize}
\end{frame}

\begin{frame}[fragile]
\frametitle{Random Variables (2)}
\begin{itemize}
	\item Cumulative distribution function (cdf):
	\[
		F(x) = P(X \leq x)
	\]
	
	\item Properties of cdf:
	\begin{itemize}
		\item $ F(x) $ is non-decreasing
			
		\item $ \lim_{x\to\infty} F(x) = F(\infty) = 1$
			
		\item $ \lim_{x\to - \infty} F(x) = F(-\infty) = 0$
	\end{itemize}
	
	\item Implication:
	\[
		P(a < X \leq b) = F(b) - F(a)
	\]
		
	\item Quantiles: values of the random variable that correspond to a given value of the cdf\footnote{You can think of the quantile function as of the inverse of the cdf.}
\end{itemize}
\end{frame}

\begin{frame}[fragile]
\frametitle{Random Variables (3)}
\begin{itemize}
	\item The term \textit{cdf} is often used interchangeably with the term \textit{probability distribution}
		
	\item Both in fact stand for a mathematical model of probability, i.e. the rule by which we assign probabilities to values (events)
		
	\item Knowledge on distributions is used in hypotheses testing, for example
\end{itemize}
\end{frame}

\begin{frame}[fragile]
\frametitle{Expectation of a Random Variable}
\begin{itemize}
	\item Discrete:
	\[
		\E(X) = \sum_{x:\ p(x) > 0} x\cdot p(x)
	\]
		
	\item Continuous:
	\[
		\E(X) = \int_{-\infty}^{\infty} x\cdot f(x) \, dx
	\]
		
	\item Let $ g(\cdot) $ be a real-valued function; then:
	\[
		\begin{array}{lcl}
			\displaystyle \E(g(X)) = \sum_{x:\ p(x) > 0} g(x)\cdot p(x)\\
			\quad\\
			\displaystyle \E(g(X)) = \int_{-\infty}^{\infty} g(x)\cdot f(x) \, dx
		\end{array}
	\]
\end{itemize}
\end{frame}

\begin{frame}[fragile]
\frametitle{Expectation of a Random Variable (2)}
\begin{itemize}
	\item $ \E(X) $ is also called the \textit{mean} of a random variable
		
	\item $ \E(X^{n}) $ is called the \textit{n}th \textit{moment} of a random variable
		
	\item If $ a $ and $ b $ are constants,
	\[
		\E(aX + b) = a\E(X) + b
	\]
		
	\item Variance:
	\[
		\var(X)  = \E[X - \E(X)]^{2} = \E(X^{2}) - [E(X)]^{2}
	\]
	
	\item An important relationship
	\[
		\var(X) = \E[\var(X|Y)] + \var[\E(X|Y)]
	\]
	
	\item Law of iterated expectations:
	\[
		\E(X) = \E[\E(X|Y)]
	\]
\end{itemize}
\end{frame}

\subsection{Discrete Random Variables}
\begin{frame}[fragile]
\frametitle{Discrete Random Variables}
\begin{itemize}
	\item Probability mass function (pmf):
	\[
		p(x)  = P(X = x)
	\]
		
	\item The pmf is positive for a countable number of values of $ X $; for the remaining ones it is zero
		
	\item Property:
	\[
		\sum_{i = 1}^{\infty}p(x_{i}) = 1
	\]
		
	\item cdf:
	\[
		F(x) = \sum_{x_{i} \leq x} p(x_{i})
	\]
\end{itemize}
\end{frame}

\begin{frame}[fragile]
\frametitle{Bernoulli Random Variables}
\begin{itemize}
	\item Characterize experiments in which the outcome is classified either as a \textit{success} (1) or as a \textit{failure} (0)
		
	\color{blue}
	Example: tossing a coin
		
	\color{black}
	\item pmf (for $ p \in (0,1) $):
	\[
		\begin{array}{lcl}
			p(0) = P(X = 0) = 1 - p\\
			p(1) = P(X = 1) = p
		\end{array}
	\]
		
	Can be written also as
	\[
		p(x) = p^{x}(1-p)^{1-x}
	\]
		
	\item cdf:
	\[
		F(x) = 
		\left\{
		\begin{array}{lcl}
			0, \quad x < 0\\
			1 - p, \quad 0 \leq x < 1\\
			p, \quad x \geq 1
		\end{array}
	\right.
	\]	
\end{itemize}
\end{frame}

\begin{frame}[fragile]
\frametitle{Bernoulli Random Variables (2)}
\begin{itemize}
	\item Mean:
	\[
		E(X) = 0\cdot (1 - p) + 1\cdot p = p
	\]
	
	\item Variance:
	\[
		\begin{array}{lcl}
			\var(X) & = & (0 - p)^{2}\cdot (1 - p) + (1 - p) ^{2}\cdot p = \\
			\quad\\
			& = & p^{2}\cdot (1 - p) + (1 - p) ^{2}\cdot p = \\
			\quad\\
			& = & (1 - p)\cdot(p^{2} + p - p^{2}) = p\cdot (1 - p)
		\end{array}
	\]
	
	\item Bernoulli random variables are a special case of the binomial ones
	
	\item Therefore we will postpone R simulation for later	
\end{itemize}
\end{frame}

\begin{frame}[fragile]
\frametitle{Binomial Random Variables}
\begin{itemize}
	\item The binomial distribution shows the number of successes in $n$ \emph{independent}\footnote{If the experiment implies sampling, then there is replacement of elements.} Bernoulli trials
		
	\item Specifically, the probability of $x$ successes in $n$ trials (the pmf) is:
	\[
		p(x) = \dbinom{n}{x} p^{x}(1-p)^{n-x}
	\] 
	
	\item cdf:
	\[
		F(x) = P(X \leq x) = \sum_{i = 1}^{x}\dbinom{n}{i} p^{i}(1-p)^{n-i}
	\]
		
	\item Mean: $ np $
	
	\item Variance: $ np(1 - p) $
\end{itemize}
\end{frame}

\begin{frame}[fragile]
\frametitle{Binomial Random Variables (2)}
\begin{itemize}
	\item \textcolor{blue}{Example problem:} Find the probability of getting heads two times out of ten fair coin tosses
		
	\item Obviously, we have to use the pmf to solve it
		
	\item We will go to R for the solution:
	\begin{lstlisting}[style = rstyle, breaklines]
	dbinom(2, 10, 0.5, log = FALSE)
	\end{lstlisting}
	
	\item \textcolor{blue}{Example problem:} What is the probability of \textcolor{red}{not} getting a pass (60\%) if you answer randomly at an exam in which there are 20 multiple-choice questions (the number of options is 4)?
		
	\item Solution using the cdf:
	\begin{lstlisting}[style = rstyle, breaklines]
	pbinom(11, 20, 0.25, lower.tail = TRUE, log.p = FALSE)
	\end{lstlisting} 
\end{itemize}
\end{frame}

\begin{frame}[fragile]
\frametitle{Binomial Random Variables (3)}
\begin{itemize}
	\item \textcolor{blue}{Example problem:} find the number of correct answers so that your test score is in the 75th percentile
		
	\item Solution using the quantile function:
	\begin{lstlisting}[style = rstyle, breaklines]
	qbinom(0.75, 20, 0.25, lower.tail = TRUE, log.p = FALSE)
	\end{lstlisting} 
	
	\item Generate 100 random numbers from the binomial distribution, where $p = 0.6$ and $n = 20$:
	\begin{lstlisting}[style = rstyle, breaklines]
	rbinom(100, 20, 0.6)
	\end{lstlisting}
	
	\item Plot them:
	\begin{lstlisting}[style = rstyle, breaklines]
	plot(rbinom(100, 20, 0.6), type = "p")
	\end{lstlisting}
	
	\item Histogram:
	\begin{lstlisting}[style = rstyle, breaklines]
	hist(rbinom(100, 20, 0.6), breaks = 20)
	\end{lstlisting} 
\end{itemize}
\end{frame}

\begin{frame}[fragile]
\frametitle{Geometric Random Variables}
\begin{itemize}
	\item Assume that in independent Bernoulli trials, the probability of success is $ p $
	
	\item Problem to solve: What is the probability of getting $ x $ failures before a success occurs?
		
	\item The random variable that measures that probability is the \textit{geometric} random variable
		
	\item pmf:
	\[
		p(x) = P(X = x) = (1 - p)^{x}p, \quad x = 0, 1, 2, \ldots
	\]
		
	\item \textcolor{blue}{Example problem:} What is the probability of tossing 5 times a fair coin and getting tails before we get heads on the sixth toss?
	\begin{lstlisting}[style = rstyle, breaklines]
	dgeom(5, 0.5, log=FALSE)
	\end{lstlisting} 
	
	\item \textcolor{red}{Problem to do:} Prove that $ \displaystyle\sum_{x = 0}^{\infty} p(x) = 1$
\end{itemize}
\end{frame}

\begin{frame}[fragile]
\frametitle{Geometric Random Variables (2)}
\begin{itemize}
	\item The cdf is:
	\[
		F(x) = 1 - (1 - p)^{x+1}\quad x = 1,2,\ldots
	\]

	\item Example of usage: what is the probability of tossing a coin between zero and five times before getting heads?
	\begin{lstlisting}[style = rstyle, breaklines]
	pgeom(5, 0.5, lower.tail = TRUE, log.p = FALSE)
	\end{lstlisting}
	\item Quantiles and random number generation:
	\begin{lstlisting}[style = rstyle, breaklines]
	qgeom, rgeom
	\end{lstlisting} 
	
	\item Mean: $ \dfrac{1 - p}{p} $
		
	\item Variance: $ \dfrac{1 - p}{p^{2}} $
\end{itemize}
\end{frame}

\begin{frame}[fragile]
\frametitle{Hypergeometric Random Variables}
\begin{itemize}
	\item The hypergeometric distribution is another special case of the binomial distribution
		
	\item Difference: There is no replacement therefore the condition of independence is violated
		
	\item The pmf is (measures the probability of $ k $ successes):
	\[
		p(x) = \displaystyle\frac{\dbinom{K}{x}\dbinom{N-K}{n-x}}{\dbinom{N}{n}}
	\]
	
	where $N$ is the size of the population, $K$ is the total number of successes in the population.
	
	\item cdf: too long to write here :)
	\end{itemize}
\end{frame}

\begin{frame}[fragile]
\frametitle{Hypergeometric Random Variables (2)}
\begin{itemize}
	\item Mean: $ n\cdot\dfrac{K}{N} $
		
	\item Variance: $ n\cdot\dfrac{K}{N} \cdot \dfrac{N-K}{N}\cdot \dfrac{N - n}{N - 1}$
		
	\item The hypergeometric distribution in $\mathsf{R}$:
	\begin{lstlisting}[style = rstyle, breaklines]
	dhyper, phyper, qhyper, rhyper
	\end{lstlisting} 
\end{itemize}
\end{frame}

\begin{frame}[fragile]
\frametitle{Poisson Random Variables}
\begin{itemize}
	\item Measures the number of events that occur for a fixed interval of time/space
		
	\item Main assumptions:
	\begin{itemize}
		\item Each event occurs independently of the time since the last event occurred
			
		\item The mean rate of occurrence per unit of time is constant 
	\end{itemize}
		
	\item \textcolor{blue}{Examples:} customers per hour in a shop; accidents per day in a city; raindrops per square centimetre; etc.
		
	\item pmf:
	\[
		p(x) = \frac{\lambda^{x}e^{-\lambda}}{x!}
	\]
	where $\lambda = \E(X) = \var(X)$ is the constant mean rate of occurrence		
\end{itemize}
\end{frame}

\begin{frame}[fragile]
\frametitle{Poisson Random Variables (2)}
\begin{itemize}
	\item cdf: also not too pleasant to write, so it's skipped here
		
	\item \textcolor{blue}{Example:} 15 cars arrive at a gas station on average per hour. What is the probability that 10 cars arrive during a chosen hour?
	\[
		p(10) = P(X = 10) = \frac{15^{10}e^{-15}}{10!} \approx 4.9\%
	\] 
		
	(Obviously, here $ \lambda = 15 $)
	\end{itemize}
\end{frame}

\begin{frame}[fragile]
\frametitle{Poisson Random Variables (3)}
\begin{itemize}
	\item Calculate the same problem in $\mathsf{R}$:
	\begin{lstlisting}[style = rstyle, breaklines]
	dpois(10, 15, log = FALSE)
	\end{lstlisting}

	\item Calculate the probability of having 10 or less cars arrive:
	\begin{lstlisting}[style = rstyle, breaklines]
	ppois(10, 15, lower.tail = TRUE, log.p = FALSE)
	\end{lstlisting}

	\item Quantiles and random number generation:
	\begin{lstlisting}[style = rstyle, breaklines]
	qpois, rpois
	\end{lstlisting}
\end{itemize}
\end{frame} 

\subsection{Continuous Random Variables}
\begin{frame}[fragile]
\frametitle{Continuous Random Variables}
\begin{itemize}
	\item The set of possible values is \textit{uncountable}
		
	\item Probability density function (pdf): a non-negative function defined for all $ x \in (-\infty, \infty) $
		
	\item Denoted by $ f(x) $, has the property that for any $ B \subset \mathbb{R} $,
	\[
		P(X \in B) = \int\limits_{B} f(x)\, dx
	\]
		
	\item If $ B = (-\infty, x]  $, then
	\[
		P(X \in B) = P (X \leq x) = \int\limits_{-\infty}^{x} f(t) \, dt = F(x)
	\]
		
	is in fact the cdf
\end{itemize}
\end{frame}

\begin{frame}[fragile]
\frametitle{Continuous Random Variables (2)}
\begin{itemize}
	\item It is clear that
	\[
		F'(x) = f(x)
	\]
		
	\item If $ B = [a, b] $ then
	\[
		P(X \in B) = \int\limits_{a}^{b} f(x) \, dx = F(b) - F(a)
	\]
		
	\item Letting $ a = b $ leads to
	\[
		P(X \in B) = \int\limits_{a}^{a} f(x) \, dx = F(a) - F(a) = 0
	\]
	In other words, the probability of getting a specific value out of infinitely many is zero
\end{itemize}
\end{frame}

\begin{frame}[fragile]
\frametitle{Continuous Random Variables (3)}
\begin{itemize}
	\item If $ B = \mathbb{R} $, then
	\[
		P(X \in \mathbb{R}) = \int\limits_{-\infty}^{\infty} f(x) \, dx = 1
	\]
		
	The latter implies $ F(\infty)  = 1$ and $ F(-\infty) = 0 $
		
	\item Given that $ \displaystyle\int\limits_{a}^{a} f(x) \, dx = 0 $,
	\[
		P(X \leq x) = P(X < x)
	\]
\end{itemize}
\end{frame}

\begin{frame}[fragile]
\frametitle{Uniform Random Variables}
\begin{itemize}
	\item For the interval $ [a, b] $, the pdf is:
	\[
		f(x) = 
		\left\{
		\begin{array}{l}
			\displaystyle\frac{1}{b-a}, \quad x \in [a,b]\\
			\quad\\
			0,\quad elsewhere
		\end{array}
		\right.
	\]
	
	\item The cdf is:
	\[
		F(x) = 
		\left\{
		\begin{array}{l}
			0,\quad x < a\\
			\quad\\
			\displaystyle\frac{x-a}{b-a}, \quad x \in [a,b)\\
			\quad\\
			1,\quad x \geq b
		\end{array}
		\right.
	\]
\end{itemize}
\end{frame}

\begin{frame}[fragile]
\frametitle{Uniform Random Variables (2)}
\begin{itemize}
	\item Mean:
	\[
		\E(X) = \dfrac{1}{2}(a + b)
	\]
	
	\item Variance:
	\[
		\var(X) = \dfrac{1}{12}(b - a)^{2}
	\]
\end{itemize}
\end{frame}

\begin{frame}[fragile]
\frametitle{Uniform Random Variables (3)}
\begin{itemize}
	\item To calculate density at a given value in R:
	\begin{lstlisting}[style = rstyle, breaklines]
	dunif(0.7, min = 0, max = 1, log = FALSE)
	\end{lstlisting}

	\item The minimum and the maximum value of the argument are 0 and 1 respectively by default, so the latter is equivalent to:
	\begin{lstlisting}[style = rstyle, breaklines]
	dunif(0.7)
	\end{lstlisting}

	\item To plot the pdf:
	\begin{lstlisting}[style = rstyle, breaklines]
	plot(dunif,0,1,main = "The Uniform pdf")
	\end{lstlisting}

	\item To calculate the cdf.:
	\begin{lstlisting}[style = rstyle, breaklines]
	punif(0.7, min = 0, max = 1, lower.tail = TRUE, log.p = FALSE)
	\end{lstlisting}
	\end{itemize}
\end{frame}

\begin{frame}[fragile]
\frametitle{Uniform Random Variables (3)}
\begin{itemize}
	\item To plot the cdf:
	\begin{lstlisting}[style = rstyle, breaklines]
	plot(punif, 0, 1, main = "Uniform cdf")
	\end{lstlisting}

	\item Calculate median:
	\begin{lstlisting}[style = rstyle, breaklines]
	qunif(0.5, min = 0, max = 1, log = FALSE)
	\end{lstlisting}

	\item Generate 10000 values of a uniformly distributed random variable:
	\begin{lstlisting}[style = rstyle, breaklines]
	runif(10000, min = 0, max = 1)
	\end{lstlisting}

	\item And plot them:
	\begin{lstlisting}[style = rstyle, breaklines]
	plot(runif(10000, min = 0, max = 1), type = "l", col = "red", main = "Uniform RV")
	\end{lstlisting}
	
	\item Histogram: 
	\begin{lstlisting}[style = rstyle, breaklines]
	hist(runif(10000, min = 0, max = 1), breaks = 50, col = "red", main = "Uniform RV (histogram)")
	\end{lstlisting}
\end{itemize}
\end{frame}

\begin{frame}[fragile]
\frametitle{The Poisson Process}
\begin{itemize}
	\item Homogeneous case: the process counts events that occur at a constant rate in time:
	\[
		P[N(t+h) - N(t) = k] = \frac{e^{-\lambda h}(\lambda h)^{k}}{k!}, \quad k = 0, 1, 2,\ldots
	\]
	
	\item Non-homogeneous case: the rate of occurrence is not constant; we do not deal with this case
\end{itemize}
\end{frame} 

\begin{frame}[fragile]
\frametitle{Exponential Random Variables}
\begin{itemize}
	\item The continuous analogue of geometric random variables

	\item Model the time between events in a Poisson process

	\item The pdf is:
	\[
		f(x) = \left\{
		\begin{array}{l}
			\lambda e^{-\lambda x}, \quad x \geq 0\\
			\quad\\
			0, \quad x < 0
		\end{array}\right.
	\]
	
	\item $\lambda$ is the mean rate at which events occur, while $\displaystyle \frac{1}{\lambda}$ is the mean waiting time between two events
\end{itemize}
\end{frame}

\begin{frame}[fragile]
	\frametitle{Exponential Random Variables (2)}
	\begin{itemize}
		\item Mean:
		\[
			\E(X) = \dfrac{1}{\lambda}
		\]
		
		\item Variance:
		\[
			\var(X) = \dfrac{1}{\lambda^{2}}
		\]		
		
		\item \textcolor{red}{Note:} The exponential distribution is a special case of the gamma distribution (along with the Erlang and the $ \chi^{2} $ distributions)
	\end{itemize}
\end{frame}

\begin{frame}[fragile]
\frametitle{Exponential Random Variables (3)}
\begin{itemize}
	\item Implementation in $\mathsf{R}$:
	\begin{lstlisting}[style = rstyle, breaklines]
	dexp(2, 0.25) 
	\end{lstlisting}
	
	\item Plot the pdf:
	\begin{lstlisting}[style = rstyle, breaklines]
	x <- seq(0,20, by = 0.01)
	y <- dexp(x,0.25)
	plot(x,y,type = "l")
	\end{lstlisting}

	\item The c.d.f. is:
	\[
		F(x) = 
		\left\{
		\begin{array}{l}
			1 - e^{-\lambda x}, \quad x \geq 0\\
			\quad\\
			0, \quad x < 0
		\end{array}
		\right.
	\]
		
	\item Plot the cdf
	\begin{lstlisting}[style = rstyle, breaklines]
	y <- pexp(x, 0.25)
	plot(x, y, type = "l")
	\end{lstlisting}
\end{itemize}
\end{frame}

\begin{frame}[fragile]
\frametitle{Exponential Random Variables (4)}
\begin{itemize}
	\item \textcolor{blue}{Example:} Let $x$ be the amount of time that a customer is processed by a cashier at a department store. What is the probability that a customer is processed between 3 and 4 minutes if on average a customer is processed 4 minutes?
		
	\item Solution:	We have to find $P(3 < X < 4)$. This is accomplished by:
	\begin{lstlisting}[style = rstyle, breaklines]
	pexp(4, 0.25) - pexp(3,0.25)
	\end{lstlisting}

	\item Next, find the time for which half of the customers are being processed
	\begin{lstlisting}[style = rstyle, breaklines]
	qexp(0.5, 0.25)
	\end{lstlisting}
	i. e. we found the median time
\end{itemize}
\end{frame} 

\begin{frame}[fragile]
\frametitle{The Gamma Function}
\begin{itemize}
	\item An extension of the factorial ($n!$) function

	\item For positive integers:
	\[
		\Gamma(n) = (n-1)!
	\]

	\item For complex numbers with a positive real part:
	\[
		\Gamma(z) = \int\limits_{0}^{\infty}x^{z-1}e^{-x}\,dx, \quad Re(z) > 0
	\]

	\item Not defined for negative integers or complex numbers with negative real part, and also for zero
\end{itemize}
\end{frame} 

\begin{frame}[fragile]
\frametitle{The Gamma Function: Implementation in $\mathsf{R}$}
\begin{itemize}
	\item Through the \cc{factorial()} function
	\item Example:
	\begin{lstlisting}[style = rstyle, breaklines]
	x <- seq(1, 10, by=.1)
	y <- factorial(x)
	plot(x, y, type = "l")
	\end{lstlisting}
\end{itemize}
\end{frame} 

\begin{frame}[fragile]
\frametitle{Gamma Random Variables}
\begin{itemize}
	\item pdf of Gamma distribution with parameters  $ \lambda > 0,\, \alpha > 0 $:
	\[
		f(x) = 
		\left\{
		\begin{array}{lcl}
			\dfrac{\lambda e^{-\lambda x}(\lambda x)^{\alpha-1}}{\Gamma(\alpha)}, \quad x \geq 0 \\
			\quad\\
			0, \quad x < 0
		\end{array}
		\right.
	\]
	
	\item $ \lambda $ is called the \textit{rate} parameter, while $ \alpha $ is called the \textit{shape} parameter
	
	\item (Check what happens with the pdf if $ \alpha = 1 $, for example)
	
	\item How does it differ from the exponential distribution? It models waiting time for 1 \textit{or more} events ahead\footnote{When only the waiting time to the next event is modelled, this is the case of the exponential distribution.}
\end{itemize}
\end{frame}

\begin{frame}[fragile]
\frametitle{Gamma Random Variables (2)}
\begin{itemize}
	\item Mean:
	\[
		\E(X) = \dfrac{\alpha}{\lambda}
	\]
	
	\item Variance
	\[
		\var(X) = \dfrac{\alpha}{\lambda^{2}}
	\]
\end{itemize}
\end{frame}

\begin{frame}[fragile]
\frametitle{Normal Random Variables} 
\begin{itemize}
	\item The  pdf of the normal distribution is given by the formula:
	\[
		f(x) = \frac{1}{\sqrt{2\pi\sigma^{2}}}\exp\left(-\frac{(x - \mu)^{2}}{2\sigma^{2}}\right), \quad x \in (-\infty, \infty)
	\]
	
	\item $ \mu $ and $ \sigma^{2} $ are correspondingly the mean and the variance of the normal distribution
	
	\item Those two parameters uniquely define each normal distribution
	
	\item If $ \mu = 0 $ and $ \sigma^{2} = 1$, then the distribution becomes the \textit{standard normal distribution}
	
	\item If $ Y = a + bX $, then
	\[
		\begin{array}{lcl}
			\E(Y) = \E(a + bX) = a + b\mu\\
			\quad\\
			\var(Y) = \var(a + bX) = b^{2}\sigma^{2}
		\end{array}
	\]
\end{itemize}
\end{frame}

\begin{frame}[fragile]
\frametitle{Normal Random Variables (2)}
\begin{itemize}
	\item To see the density at a given value (i. e. the height of the pdf at that value) -- in this case we take as an example the density at 5:
	\begin{lstlisting}[style = rstyle, breaklines]
	dnorm(5, mean = 0, sd = 1, log = FALSE)	
	\end{lstlisting}
	
	\item In the case of the standard normal pdf, you can issue just:
	\begin{lstlisting}[style = rstyle, breaklines]
	dnorm(5)
	\end{lstlisting}
	
	\item You can plot the pdf with:
	\begin{lstlisting}[style = rstyle, breaklines]
	plot(dnorm,-3,3, main = "Standard normal distribution")
	\end{lstlisting}

	\item What if you want to plot a `non-standard' normal distribution?
	\begin{lstlisting}[style = rstyle, breaklines]
	x <- seq(-20, 20, by=0.1)
	y <- dnorm(x, mean = 2, sd = 5)
	plot(x, y, type = "l")
	\end{lstlisting}
\end{itemize}
\end{frame}

\begin{frame}[fragile]
\frametitle{Normal Random Variables (3)}
\begin{itemize}
	\item The cdf is the integral of the pdf from $-\infty$ to $x$
	
	\item To calculate the cdf:
	\begin{lstlisting}[style = rstyle, breaklines]
	pnorm(5, mean = 2, sd = 5, lower.tail = TRUE, log.p = FALSE)
	\end{lstlisting}
	
	\item For example, to find the probability that a value is within 1.96 standard deviations from the mean:
	\begin{lstlisting}[style = rstyle, breaklines]
	pnorm(1.96, lower.tail = TRUE) - pnorm(-1.96, lower.tail = TRUE)
	\end{lstlisting}

	\item To plot the normal cdf:
	\begin{lstlisting}[style = rstyle, breaklines]
	plot(pnorm, -5, 5, main = "Normal cdf")
	\end{lstlisting}
\end{itemize}
\end{frame}

\begin{frame}[fragile]
\frametitle{Normal Random Variables (4)}
\begin{itemize}
	\item The quantile function is the inverse of the distribution -- it gives the value $x$ at which $P(X\leq x) = p$, where $p$ is pre-specified
	
	\item It is implemented via:
	\begin{lstlisting}[style = rstyle, breaklines]
	qnorm(0.25, mean = 0, sd = 1, lower.tail = TRUE, log.p = FALSE)
	\end{lstlisting}
	
	\item For example, this can be used to find the median of the standard normal distribution:
	\begin{lstlisting}[style = rstyle, breaklines]
	qnorm(0.5)
	\end{lstlisting}
	\item \ldots or to find the lower and upper bounds of a 95\% confidence interval:
	\begin{lstlisting}[style = rstyle, breaklines]
	c(qnorm(0.025),qnorm(0.975))
	\end{lstlisting}
\end{itemize}
\end{frame}

\begin{frame}[fragile]
\frametitle{Normal Random Variables (5)}
\begin{itemize}
	\item To generate a normal random variable with 10000 values:
	\begin{lstlisting}[style = rstyle, breaklines]
	rnorm(10000, mean = 2, sd = 5)
	\end{lstlisting}
	
	\item To plot such a variable:
	\begin{lstlisting}[style = rstyle, breaklines]
	plot(rnorm(10000, mean = 2, sd = 5), type = "l", col = "blue", main = "White noise")
	\end{lstlisting}
	
	\item To plot its histogram:
	\begin{lstlisting}[style = rstyle, breaklines]
	hist(rnorm(10000, mean = 2, sd = 5), breaks = 50, col = "orange")
	\end{lstlisting}
\end{itemize}
\end{frame} 

\begin{frame}[fragile]
\frametitle{The $\chi^{2}$ Distribution}
\begin{itemize}
	\item Let $X_{1}, \ldots, X_{k}$ be $k$ independent standard normal random variables. Then: 
	\[
		(X_{1}^{2} + \ldots + X_{k}^{2}) \sim \chi^{2} (k)
	\]

	\item Read as ``chi-square distribution with $k$ degrees of freedom''

	\item Main usage: for hypotheses testing or for construction of confidence intervals

	\item \textcolor{red}{Note:} We are considering the so-called centred chi-square distribution; it is generalized in its non-centred version
	\end{itemize}
\end{frame}

\begin{frame}[fragile]
\frametitle{The $\chi^{2}$ Distribution (2)}
\begin{itemize}
	\item The pdf is:
	\[
		f(x,k) = 
		\left\{
		\begin{array}{l}
			\displaystyle\frac{x^{\frac{(k-2)}{2}}e^{-\frac{x}{2}}}{2^{\frac{k}{2}}\Gamma\left(\frac{k}{2}\right)},\quad x \geq 0\\
			\quad\\
			0,\quad \textrm{otherwise}
		\end{array}
		\right.
		\]

	\item Note that the $ \chi^{2} $ distribution is a special case of the gamma distribution where $ t = \dfrac{k}{2} $ and $ \lambda = \dfrac{1}{2} $

	\item $\mathsf{R}$ implementation:
	\begin{lstlisting}[style = rstyle, breaklines]
	dchisq(3, 5)
	\end{lstlisting}

	\item Plot the pdf:
	\begin{lstlisting}[style = rstyle, breaklines]
	x <- seq(0, 20, by = 0.01)
	y <- dchisq(x, 5)
	plot(x, y, type="l")
	\end{lstlisting}
\end{itemize}
\end{frame}

\begin{frame}[fragile]
\frametitle{The $\chi^{2}$ Distribution (3)}
\begin{itemize}
	\item The cdf is:
	\[
		F(x,k) = 	\frac{\gamma\left(\frac{k}{2},\frac{x}{2}\right)}{2^{k/2}\Gamma\left(\frac{k}{2}\right)}
	\]
	where $\gamma$ is the lower incomplete gamma function defined as:
	
	\[
		\gamma(s,x) = \int\limits_{0}^{x}t^{s-1}e^{t}dt
	\]
	
	\item Implementation in $\mathsf{R}$:
	\begin{lstlisting}[style = rstyle, breaklines]
	pchisq(3, 5)
	\end{lstlisting}
	
	\item Plot the cdf:
	\begin{lstlisting}[style = rstyle, breaklines]
	x <- seq(0, 20, by = 0.01)
	y <- pchisq(x, 5)
	plot(x, y, type = "l")
	\end{lstlisting}
	
	\item Quantiles and random numbers:
	\begin{lstlisting}[style = rstyle, breaklines]
	qchisq, rchisq
	\end{lstlisting}
\end{itemize}
\end{frame}

\begin{frame}[fragile]
\frametitle{The Beta Function}
\begin{itemize}
	\item Known also as `the Euler integral'

	\item Defined as:
	\[
		B(x, y) = \int_{0}^{1}t^{x-1}(1-t)^{y-1}dt
	\]
	where $x$ and $y$ are complex numbers having positive real parts.
\end{itemize}
\end{frame}
 
\begin{frame}[fragile]
\frametitle{The $F$ Distribution}
\begin{itemize}
	\item If $X_{1} \sim \chi^{2}(k_{1})$ and $X_{2} \sim \chi^{2}(k_{2})$, then $\displaystyle \frac{X_{1}/k_{1}}{X_{2}/k_{2}} \sim F(k_{1},k_{2})$

	\item Used mainly in hypotheses testing

	\item pdf:
	\[
		f(x,k_{1},k_{2}) = \sqrt{\frac{(k_{1}x)^{k_{1}}\,k_{2}^{k_{2}}}{xB\left(\frac{k_{1}}{2},\frac{k_{2}}{2}\right)}}
	\]

	\item $\mathsf{R}$ implementation:
	\begin{lstlisting}[style = rstyle, breaklines]
	df(3, 2, 5)
	\end{lstlisting}
\end{itemize}
\end{frame}

\begin{frame}[fragile]
\frametitle{The $F$ Distribution (2)}
\begin{itemize}
	\item The c.d.f.:
	\[
		F(x,k_{1},k_{2}) = I_{\frac{k_{1}x}{k_{1}x + k_{2}}}\left(\frac{k_{1}}{2},\frac{k_{2}}{2}\right)
	\]
	where $I$ is the regularized incomplete beta function (we will leave these complications to your own curiosity)

	\item In $\mathsf{R}$:
	\begin{lstlisting}[style = rstyle, breaklines]
	pf(3, 2, 5)
	\end{lstlisting}

	\item Quantiles, random variables:
	\begin{lstlisting}[style = rstyle, breaklines]
	qf, rf
	\end{lstlisting}
\end{itemize}
\end{frame}

\begin{frame}[fragile]
\frametitle{The $t$ Distribution}
\begin{itemize}
	\item Arises in estimating the mean of a normally distributed variable when sample size is small and the population variance is unknown

	\item Used in statistical tests, linear regression analysis, confidence interval construction, etc.
\end{itemize}
\end{frame}

\begin{frame}[fragile]
\frametitle{The $t$ Distribution (2)}
\begin{itemize}
	\item Consider the random variable $x$. Let the sample size is $n$ and sampling is from a normally distributed population with population mean $\mu$

	\item Let the sample mean of the random variable be:
	\[
		\overline{x} = \frac{1}{n}\sum_{i=1}^{n} x_{i}
	\]

	\item and let the sample variance be:
	\[
		s^{2} = \frac{1}{n-1}\sum_{i=1}^{n} (x_{i} - \overline{x})^{2}
	\]
\end{itemize}
\end{frame}

\begin{frame}[fragile]
\frametitle{The $t$ Distribution (3)}
\begin{itemize}
	\item The statistic:
	\[
		t = \frac{\overline{x} - \mu}{s/\sqrt{n}}
	\]
	has $t$ distribution with $n-1$ degrees of freedom

	\item Interpreted as the distribution of the difference between the population mean and the sample mean divided by the sample standard deviation (the latter normalized by $\sqrt{n}$)

	\item Continuous distribution, symmetrical

	\item The pdf is:
	\[
		f(t) = \frac{\Gamma\left(\frac{n}{2}\right)}{\sqrt{(n-1)\pi}\Gamma\left(\frac{n-1}{2}\right)}\left(1 + \frac{t^{2}}{n-1}\right)^{-\frac{n}{2}}
	\]
\end{itemize}
\end{frame}

\begin{frame}[fragile]
\frametitle{The $t$ Distribution (4)}
\begin{itemize}
	\item $\mathsf{R}$ implementation:
	\begin{lstlisting}[style = rstyle, breaklines]
	dt(0, 15)
	\end{lstlisting}
	
	\item Plot the pdf:
	\begin{lstlisting}[style = rstyle, breaklines]
	x <- seq(-20, 20, by = 0.01)
	y = dt(x, 99)
	plot(x, y, type = "l")
	\end{lstlisting}

	\item The c.d.f.: again too long to write; implementation in $\mathsf{R}$:
	\begin{lstlisting}[style = rstyle, breaklines]
	pt(0, 15)
	\end{lstlisting}
	
	\item Quantiles and random number generation:
	\begin{lstlisting}[style = rstyle, breaklines]
	qt, rt
	\end{lstlisting}
\end{itemize}
\end{frame} 

\section{Multivariate Distributions}
\begin{frame}[fragile]
\frametitle{What is a Multivariate Distribution?}
\begin{itemize}
	\item Similarly to the univariate case, a multivariate probability distribution is a model of probability

	\item However, this model describes more than one random variable at a time

	\item In other words, instead of having a scalar random variable, we work with a random vector

	\item A synonym of \textit{multivariate distribution} is \textit{joint distribution}

	\item The latter is maybe more informative since it captures the potential interaction of random variables
\end{itemize}
\end{frame} 

\begin{frame}[fragile]
\frametitle{The Joint Probability Distribution}
\begin{itemize}
	\item Take the random vector $(X_{1}, X_{2},\ldots, X_{n})$

	\item The joint probability distribution gives the probability that each of the random variables takes a specific value or falls in a specified range

	\item Joint probability distributions are expressed through:
	\begin{itemize}
		\item Their joint cumulative distribution function (cdf)
		\item Their joint probability density function (pdf)/probability mass function (pmf)
	\end{itemize}
\end{itemize}
\end{frame}

\begin{frame}[fragile]
\frametitle{The Joint cdf, pdf, and pmf}
\begin{itemize}
	\item Joint cdf:
	\[
		\begin{array}{lcl}
			F_{X_{1},X_{2},\ldots,X_{n}}(x_{1},x_{2},\ldots,x_{n}) = P(X_{1} \leq x_{1}, X_{2} 	\leq x_{2},\ldots, X_{n} \leq x_{n})
		\end{array}
	\]
	
	\item Joint pmf (discrete case):
	\[
		p_{X_{1},X_{2},\ldots,X_{n}}(x_{1},x_{2},\ldots,x_{n}) = P(X_{1} = x_{1}, X_{2} = x_{2},\ldots, X_{n} = x_{n})
	\]
	
	\item Joint pdf (continuous case)
	\[
		f_{X_{1},X_{2},\ldots,X_{n}}(x_{1},x_{2},\ldots,x_{n}) = \dfrac{\partial F_{X_{1},X_{2},\ldots,X_{n}}(x_{1},x_{2},\ldots,x_{n}) }{\partial x_{1}\partial x_{2}\ldots \partial x_{n}}
	\]
\end{itemize}
\end{frame}

\begin{frame}[fragile]
\frametitle{Example: Multivariate Normal}
\begin{itemize}
	\item pdf:
	\[
		f_{\mathbf X}(x_1,\ldots,x_{n}) = \frac{\exp\left(-\frac 1 2 ({\mathbf x}-{\boldsymbol\mu})^\mathrm{T}{\boldsymbol\Sigma}^{-1}({\mathbf x}-{\boldsymbol\mu})\right)}{\sqrt{(2\pi)^{n}|\boldsymbol\Sigma|}}
	\]
	where $\mathbf X = (X_{1}, X_{2},\ldots, X_{n})$

	\item No analogical capabilities in base R

	\item The \textbf{mvtnorm} package: for working with multivariate $t$ and normal distributions
\end{itemize}
\end{frame}

\begin{frame}[fragile]
\frametitle{R Example: Bivariate Normal}
\begin{itemize}
	\item Calculate density:
	\begin{lstlisting}[style = rstyle, breaklines]
	library(mvtnorm)
			
	sigmamat <- matrix(c(1, 0.6, 0.6, 1), nrow = 2)
	sigmamat
			
	dmvnorm(x = c(1,2), mean = c(0.05, -0.05), sigma = sigmamat)	
	\end{lstlisting}
	
	\item Cumulative:
	\begin{lstlisting}[style = rstyle, breaklines]
	pmvnorm(lower = -Inf, upper = c(1,2), mean = c(0.05, -0.05), sigma = sigmamat)	
	\end{lstlisting}
\end{itemize}
\end{frame}

\begin{frame}[fragile]
\frametitle{R Example: Bivariate Normal (2)}
\begin{itemize}
	\item Random number generation, correlated variables:
	\begin{lstlisting}[style = rstyle, breaklines]
	rvec1 <- rmvnorm(n = 10000, mean = c(0.05, -0.05), sigma = sigmamat)
	rvec1 <- as.data.frame(rvec1)	
	\end{lstlisting}

	\item See the following slide for a picture
\end{itemize}
\end{frame}

\begin{frame}[fragile]
\frametitle{R Example: Bivariate Normal (3)}
\begin{center}
	\includegraphics[scale=0.35]{fig1}
\end{center}
\end{frame}

\begin{frame}[fragile]
\frametitle{R Example: Bivariate Normal (4)}
\begin{itemize}
	\item Same but if the two variables are uncorrelated
	\begin{lstlisting}[style = rstyle, breaklines]
	sigmamat2 <- matrix(c(1, 0, 0, 1), nrow = 2)
	sigmamat2
			
	rvec2 <- rmvnorm(n = 10000, mean = c(0.05, -0.05), sigma = sigmamat2)
	rvec2 <- as.data.frame(rvec2)	
	\end{lstlisting}
	
	\item See next picture
\end{itemize}
\end{frame}

\begin{frame}[fragile]
\frametitle{R Example: Bivariate Normal (5)}
\begin{center}
	\includegraphics[scale=0.35]{fig2}
\end{center}
\end{frame} 

\begin{frame}[fragile]
	\frametitle{References}
	\begin{itemize}
		\item \href{https://towardsdatascience.com/gamma-distribution-intuition-derivation-and-examples-55f407423840}{Gamma Distribution — Intuition, Derivation, and Examples}
		
		\item Hogg, R., McKean, J., and A. Craig (2018): \textit{Introduction to Mathematical Statistics}, Pearson, 8th ed.
		
		\item Krishnamoorty, K. (2016): \textit{Handbook of	Statistical Distributions with Applications}, CRC Press, 2nd ed.
		
		\item Pinsky, M., and S. Karlin (2010): \textit{An Introduction to Stochastic Modelling}, Academic Press, fourth ed.
		
		\item Ross, S. (2014): \textit{A First Course in Probability}, Peason, 9th ed.
		
		\item Ross, S. (2019): \textit{Introduction to Probability Models}, Academic Press, 12th ed.		
	\end{itemize}
\end{frame}



\end{document}
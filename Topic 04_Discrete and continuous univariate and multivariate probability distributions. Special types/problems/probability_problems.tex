\documentclass[12pt,a4paper]{article}
\usepackage[utf8]{inputenc}
\usepackage[english]{babel}
\usepackage{amsmath}
\usepackage{amsfonts}
\usepackage{amssymb}
\usepackage{graphicx}
\usepackage{mathpazo}
\usepackage{setspace}
\usepackage[left=2cm,right=2cm,top=2cm,bottom=2cm]{geometry}

\DeclareMathOperator{\E}{\mathsf{E}}

\begin{document}
\title{Some Problems in Probability}
\author{Kaloyan Ganev}
\date{\empty}
\maketitle

Note: The problems are taken from the 7th edition of Hogg and Craig's \textit{Introduction to Mathematical Statistics}, and the solutions are mine.

\onehalfspacing
\paragraph{1.3.10.} A bowl contains 16 chips, of which 6 are red, 7 are white, and 3 are blue. If four chips are taken at random and without replacement, find the probability that:
(a) each of the four chips is red; (b) none of the four chips is red; (c) there is at least one chip of each color.

\paragraph{Solution:}
\begin{itemize}
	\item[(a)] Four chips can be selected in $16 \choose 4$ ways. Red chips can be selected in $6 \choose 4$ ways. The probability is then:
	
	\[
		P(\textrm{4 red chips}) = \displaystyle\frac{{6 \choose 4}}{{16 \choose 4}} \approx 0.008
	\]
	
	\item[(b)] Analogical to (a), only the four chips are selected among the white and blue ones (10 in total). The probability is:
	
	\[
		P(\textrm{4 non-red chips}) = \displaystyle\frac{{10 \choose 4}}{{16 \choose 4}} \approx 0.115
	\]
	
	\item[(c)] 
	
	\[
		P(\textrm{at least 1 of each colour}) = \displaystyle\frac{{6 \choose 1}{7 \choose 1}{3 \choose 2}}{{16 \choose 4}} + \frac{{6 \choose 1}{7 \choose 2}{3 \choose 1}}{{16 \choose 4}} + \frac{{6 \choose 2}{7 \choose 1}{3 \choose 1}}{{16 \choose 4}} = 0.45
	\]
\end{itemize} 

\paragraph{1.3.12.} Compute the probability of being dealt at random and without replacement a 13-card bridge hand consisting of: (a) 6 spades, 4 hearts, 2 diamonds, and 1 club; (b) 13 cards of the same suit.

\paragraph{Solution:} 
\begin{itemize}
	\item[(a)] 
	\[
		P(\textrm{hand}) = \frac{{13 \choose 6}{13 \choose 4}{13 \choose 2}{13 \choose 1}}{{52 \choose 13}} \approx 0.002
	\]	
	
	\item[(b)]
	\[
		P(\textrm{hand}) = \frac{{4 \choose 1}}{{52 \choose 13}} \approx 6.3\cdot 10^{-12}
	\]
\end{itemize}

\paragraph{1.3.15.} In a lot of 50 light bulbs, there are 2 bad bulbs. An inspector examines five bulbs, which are selected at random and without replacement.
(a) Find the probability of at least one defective bulb among the five.
(b) How many bulbs should be examined so that the probability of finding at least
one bad bulb exceeds $\dfrac{1}{2}$?

\paragraph{Solution:}
\paragraph{1.3.15.}

\begin{itemize}
	\item[(a)] The probability of having no defective bulb among the five selected is:

	\[
		P(\textrm{no defective bulb}) = \frac{{48 \choose 5}}{{50 \choose 5}}
	\]

Therefore, the probability of having at least one defective bulb is:

	\[
		P(\textrm{at least one defective bulb}) = 1 - \frac{{48 \choose 5}}{{50 \choose 5}} \approx 0.19
	\]
	
	\item[(b)] We have to solve the inequality:
	
	\[
		P(\textrm{at least one defective bulb}) = 1 - \frac{{48 \choose x}}{{50 \choose x}} > 0.5
	\]
	
	We have:
	
	\[
		{48 \choose x} < 0.5 \, {50 \choose x}
	\]
	
	or:
	
	\[
		\frac{48!}{x!(48-x)!} < 0.5\, \frac{50!}{x!(50-x)!}
	\]
	
	\[
		(50-x)(49-x) < 0.5\cdot 50\cdot 49
	\]
	
	\[
		x^{2} - 99x + 1225 < 0
	\]
	
	\[
		x > 14.5 \Rightarrow x = 15		
	\]
\end{itemize}

\paragraph{1.4.4.} From a well-shuffled deck of ordinary playing cards, four cards are turned over one at a time without replacement. What is the probability that the spades and red cards alternate?

\paragraph{Solution:}
	\[
		P(\textrm{spades and red cards alternate}) = \frac{13}{52}\cdot\frac{26}{51}\cdot\frac{12}{50}\cdot\frac{25}{49}  + \frac{26}{52}\cdot\frac{13}{51}\cdot\frac{25}{50}\cdot\frac{12}{49} \approx 0.03 
	\] 
	
\paragraph{1.4.10.} In an office there are two boxes of computer disks: Box C1 contains seven
Verbatim disks and three Control Data disks, and box C2 contains two Verbatim disks and eight Control Data disks. A person is handed a box at random with prior probabilities $P(C_{1}) = \dfrac{2}{3}$ and $P(C_{2}) = \dfrac{1}{3}$, possibly due to the boxes’ respective locations. A disk is then selected at random and the event $C$ occurs if it is from Control Data. Using an equally likely assumption for each disk in the selected box, compute $P(C_{1}|C)$ and $P(C_{2}|C)$.

\paragraph{Solution:}
	\[
		P(C_{1}|C) = \frac{P(C_{1})P(C|C_{1})}{P(C_{1})P(C|C_{1}) + P(C_{2})P(C|C_{2})} = \frac{\frac{2}{3}\frac{3}{10}}{\frac{2}{3}\frac{3}{10} + \frac{1}{3}\frac{8}{10}} = \frac{6}{14} = \frac{3}{7}
	\]

	\[
		P(C_{2}|C) = \frac{P(C_{2})P(C|C_{2})}{P(C_{1})P(C|C_{1}) + P(C_{2})P(C|C_{2})} = \frac{\frac{1}{3}\frac{8}{10}}{\frac{2}{3}\frac{3}{10} + \frac{1}{3}\frac{8}{10}} = \frac{8}{14} = \frac{4}{7}
	\] 
	
\paragraph{1.4.32.} Hunters A and B shoot at a target; the probabilities of hitting the target
are $p_{1}$ and $p_{2}$, respectively. Assuming independence, can $p_{1}$ and $p_{2}$ be selected so that
	\[
		P(\textrm{zero hits}) = P(\textrm{one hit}) = P(\textrm{two hits})?
	\]

\paragraph{Solution:}

	\[
		P(\textrm{zero hits}) = (1-p_{1})(1-p_{2})
	\]

	\[
		P(\textrm{one hit}) = p_{1}(1-p_{2}) + p_{2}(1-p_{1})
	\]

	\[
		P(\textrm{two hits}) = p_{1}p_{2}
	\]

	Equate the three probabilities:

	\[
	\begin{array}{lclcl}
		(1-p_{1})(1-p_{2}) & = & p_{1}(1-p_{2}) + p_{2}(1-p_{1}) & = & p_{1}p_{2}\\
		1 - p_{2} - p_{1} + p_{1}p_{2} & = & p_{1} - p_{1}p_{2} + p_{2} - p_{1}p_{2} & = & p_{1}p_{2}
	\end{array}
	\]

	Subtract $p_{1}p_{2}$ from each part of the triple equality:
	\[
		1 - p_{2} - p_{1} = p_{1} - 3p_{1}p_{2} + p_{2} =  0
	\]

	Write it as a system:

	\[
		\left|
		\begin{array}{lcl}
			1 - p_{2} - p_{1} & = & 0\\
			p_{1} - 3p_{1}p_{2} + p_{2} & = &  0
		\end{array}
		\right.
	\]

	From this:
	\[
	\left|
	\begin{array}{lcl}
			p_{1} + p_{2} & = & 1\\
			p_{1}p_{2} & = & \frac{1}{3}
	\end{array}
	\right.
	\]

	Using Vieta's formulas, we can show that $p_{1}$ and $p_{2}$ are the roots of the quadratic equation:
	\[
		p^{2} - p + \frac{1}{3} = 0 
	\]

	The discriminant of this equation is:

	\[
		D = -\frac{1}{3} < 0
	\]

	Therefore, no real $p_{1}$ and $p_{2}$ exist so that the equalities are satisfied. 
	
\paragraph{1.5.4.} For each of the following, find the constant $c$ so that $p(x)$ satisfies the condition of being a pmf of one random variable X.
	\begin{enumerate}
		\item[(a)] $p(x) = c\left(\dfrac{2}{3}\right)^{x},\ x = 1, 2, 3, . . .$, zero elsewhere.
		
		\item[(b)] $p(x) = cx,\ x = 1, 2, 3, 4, 5, 6,$ zero elsewhere.
	\end{enumerate}

\begin{itemize}
	\item[(a)]
	\[
		1 = \sum_{x=1}^{\infty} c \left(\frac{2}{3}\right)^{x} = c\frac{\frac{2}{3}}{1 - \frac{2}{3}} = 2c \Rightarrow c = \frac{1}{2}
	\]
	
	\item[(b)]
	\[
		1 = \sum_{x=1}^{6}cx = 21c \Rightarrow c = \frac{1}{21}
	\]
\end{itemize} 

\paragraph{1.7.18.} Let $X$ be the number of gallons of ice cream that is requested at a certain store on a hot summer day. Assume that $f(x) = 12x(1000-x)^{2}/1012, 0 < x < 1000$, zero elsewhere, is the pdf of $X$. How many gallons of ice cream should the store have on hand each of these days, so that the probability of exhausting its supply on a particular day is 0.05?

\paragraph{Solution:} We look for that $x$ for which the probability that $F(X > x) = 0.05$. The problem to solve is equivalent to finding $x$ for which $F(X \leq x) = 0.95$:

\[
\begin{array}{lcl}
	F(X \leq x) & = & \displaystyle \int_{0}^{x} \frac{12t(1000-t)^{2}}{10^{12}}\,dt = \frac{12}{10^{12}} \int_{0}^{x} (1000t - 2000t^{2} + t^{3})\,dt = \\
	\quad\\
	& = & \displaystyle \frac{12}{10^{12}}\left(1000000\frac{t^{2}}{2} - 2000\frac{t^{3}}{3} + \frac{t^{4}}{4}\right)\big|_{0}^{x} = \frac{1}{10^{12}}\left(6000000t^{2} - 8000t^{3} + 3t^{4}\right)\big|_{0}^{x} = \\
	\quad\\
	& = & \displaystyle\frac{1}{10^{12}}\left(6000000x^{2} - 8000x^{3} + 3x^{4}\right)
\end{array}
\]

So, we need to solve:

\[
3x^{4} - 8000x^{3} +  6000000x^{2} = 0.95\cdot 10^{12}
\]

Solving this numerically, we find the only positive real root is $x\approx 751.4$ gallons.

\paragraph{1.8.2.} Let $X$ have the pdf $f(x) = (x + 2)/18, -2 < x < 4$, zero elsewhere. Find $ \E(X), \E[(X + 2)^{3}] $, and $ \E[6X - 2(X + 2)^{3}] $.

\paragraph{Solution:} 

\[
	\mathsf{E}(X) = \int\limits_{-2}^{4} x\cdot\frac{x+2}{18}\, dx = \int\limits_{-2}^{4} \frac{x^{2} +2x}{18}\, dx = \frac{x^{3} + 3x^{2}}{54}\Big|_{-2}^{4} = \frac{112}{54} - \frac{4}{54} = 2
\]

\quad\\

\[
\begin{array}{lcl}
	\mathsf{E}[(X+2)^3] & = & \displaystyle \int\limits_{-2}^{4}(x + 2)^{3}\cdot\frac{x+2}{18}\, dx = \int\limits_{-2}^{4}\frac{(x+2)^{4}}{18}\, dx =\\
	\quad\\
	& = & \displaystyle \frac{(x+2)^5}{90}\Big|_{-2}^{4} = \frac{7776 - 0}{90} = 86.4
\end{array}
\]

\quad\\

\[
\begin{array}{lcl}
	\mathsf{E}[6X -2(X+2)^3] & = & \displaystyle \int\limits_{-2}^{4}[6x - 2(x + 2)^{3}]\cdot\frac{x+2}{18}\, dx = \\
	\quad\\
	& = & \displaystyle \int\limits_{-2}^{4}x\cdot\frac{(x+2)}{3}\, dx - \int\limits_{-2}^{4}\frac{(x+2)^{4}}{9}\, dx = \\
	\quad\\
	& = & \displaystyle \frac{x^{3} + 3x^{2}}{9}\Big|_{-2}^{4} - \frac{(x+2)^5}{45}\Big|_{-2}^{4} = \\
	\quad\\
	& = & \displaystyle \frac{112-4}{9} - \frac{7776-0}{45} = 12 - 172.8 =  -160.8
\end{array}
\] 

\paragraph{1.8.5.} Let the pmf $ p(x) $ be positive at $x = -1, 0, 1$ and zero elsewhere.
\begin{enumerate}
	\item[(a)] If $ p(0) = \dfrac{1}{4}$ , find $ \E(X^{2}) $.
	
	\item If $ p(0) = \dfrac{1}{4}$ and if If $ \E(X) = \dfrac{1}{4}$, determine $ p(-1)$ and $ p(1) $.
\end{enumerate}

\paragraph{Solution:}
\begin{itemize}
	\item[(a)] We can easily say that $\displaystyle p(-1) + p(1) = 1 - p(0) = \frac{3}{4}$. We can proceed directly to the calculation of $\mathsf{E}(X^{2})$ using this. Applying the formula:
	\[
	\mathsf{E}(X^{2}) = \sum_{x=-1}^{1}x^{2}p(x) = (-1)^{2}p(-1) + 0^{2}\frac{1}{4} + 1^2p(1) =  p(-1) + p(1) = \frac{3}{4}
	\]
	
	\item[(b)]
	\[
	\frac{1}{4} = \mathsf{E}(X) = \sum_{x=-1}^{1}xp(x) = -1\cdot p(-1) + 0\cdot p(0) + 1\cdot p(1) = p(1) - p(-1)
	\]
	
	We also have:
	\[
	p(1) + p(-1) = 1 - p(0) = \frac{3}{4}
	\]
	
	We have two equations in two unknowns. Solving them leads to $\displaystyle p(1) = \frac{1}{2}$ and $\displaystyle p(-1) = \frac{1}{4}$.
\end{itemize}

\paragraph{1.8.10.} Let X have a Cauchy distribution which has the pdf
	\[
		f(x) = \dfrac{1}{\pi}\cdot \dfrac{1}{x^{2} + 1}, \quad -\infty < x < \infty
	\]
Then $X$ is symmetrically distributed about 0 (why?). Why isn't $ \E(X) = 0 $?

\paragraph{Solution:} $X$ is symmetrically distributed around zero because it is an inverted quadratic function which is symmetrical around zero ($x^2+1$). If we have to find the mean of this distribution, we would have to compute the integral: 

\[
	\int_{-\infty}^{\infty} x\cdot \frac{1}{\pi}\cdot \frac{1}{1+x^{2}}\,dx = \frac{1}{2\pi}\ln(1+x^{2})\Big|_{-\infty}^{\infty} = \infty - \infty
\]

The latter expression does not define a finite number (in other words, the integral does not converge) so the mean does not exist, i.e. it is not defined.

\end{document}
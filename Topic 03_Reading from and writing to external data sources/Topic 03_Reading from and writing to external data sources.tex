\documentclass[10pt]{beamer}
\usetheme{CambridgeUS}
%\usetheme{Boadilla}
\definecolor{myred}{RGB}{163,0,0}
%\usecolortheme[named=blue]{structure}
\usecolortheme{dove}
\usefonttheme[]{professionalfonts}
\usepackage[english]{babel}
\usepackage{amsmath,amsfonts,amssymb}
\usepackage{mathtools}
\usepackage{commath}
\usepackage{xcolor}
\usepackage{tikz}
\usepackage{pgfplots}
\pgfplotsset{compat=newest,compat/show suggested version=false}
\usetikzlibrary{arrows,shapes,calc,backgrounds}
\usepackage{bm}
\usepackage{textcomp}
%\usepackage{gensymb}
%\usepackage{verbatim}
\usepackage{mathrsfs}  
\usepackage{paratype}
\usepackage{mathpazo}
\usepackage{listings}
\usepackage{csvsimple}
\usepackage{booktabs}

\newcommand{\cc}[1]{\texttt{\textcolor{blue}{#1}}}

\DeclareMathOperator{\adj}{adj}
\DeclareMathOperator{\tr}{tr}
\DeclareMathOperator{\diag}{diag}
\DeclareMathOperator{\E}{\mathsf{E}}
\DeclareMathOperator{\var}{\mathsf{Var}}
\DeclareMathOperator{\cov}{\mathsf{Cov}}
\DeclareMathOperator{\corr}{\mathsf{Corr}}
\DeclareMathOperator{\plim}{plim}
\DeclareMathOperator{\rank}{rank}


\definecolor{ttcolor}{RGB}{0,0,1}%{RGB}{163,0,0}
\definecolor{mygray}{RGB}{248,249,250}

% Number theorem environments
\setbeamertemplate{theorem}[ams style]
\setbeamertemplate{theorems}[numbered]

% Reset theorem-like environments so that each is numbered separately
\usepackage{etoolbox}
\undef{\definition}
\theoremstyle{definition}
\newtheorem{definition}{\translate{Definition}}

% Change colours for theorem-like environments
\definecolor{mygreen1}{RGB}{0,96,0}
\definecolor{mygreen2}{RGB}{229,239,229}
\setbeamercolor{block title}{fg=white,bg=mygreen1}
\setbeamercolor{block body}{fg=black,bg=mygreen2}

\lstdefinestyle{numbers}{numbers=left, stepnumber=1, numberstyle=\tiny, numbersep=10pt}
\lstdefinestyle{MyFrame}{backgroundcolor=\color{yellow},frame=shadowbox}

\lstdefinestyle{rstyle}%
{language=R,
	basicstyle=\footnotesize\ttfamily,
	backgroundcolor = \color{mygray},
	commentstyle=\slshape\color{green!50!black},
	keywordstyle=\color{blue},
	identifierstyle=\color{blue},
	stringstyle=\color{orange},
	%escapechar=\#,
	rulecolor = \color{mygray}, 
	showstringspaces = false,
	showtabs = false,
	tabsize = 2,
	emphstyle=\color{red},
	frame = single}

\lstset{language=R,frame=single}    

\hypersetup{colorlinks, urlcolor=blue, linkcolor = myred}

\AtBeginSection{\frame{\sectionpage}}

% Remove Section 1, Section 2, etc. as titles in section pages
\defbeamertemplate{section page}{mine}[1][]{%
	\begin{centering}
		{\usebeamerfont{section name}\usebeamercolor[fg]{section name}#1}
		\vskip1em\par
		\begin{beamercolorbox}[sep=12pt,center]{part title}
			\usebeamerfont{section title}\insertsection\par
		\end{beamercolorbox}
	\end{centering}
} 

\setbeamertemplate{section page}[mine] 

\beamertemplatenavigationsymbolsempty

\title{R403: The R Language for Statistical Computing}
\subtitle{Topic 3: \textcolor{myred}{Reading from and Writing to External Data Sources}}
\author{Kaloyan Ganev}

\date{2022/2023} 

\begin{document}
\maketitle

\begin{frame}[fragile]
\frametitle{Lecture Contents}
\tableofcontents
\end{frame}

\section{Introduction}
\begin{frame}[fragile]
\frametitle{Introductory notes}
\begin{itemize}
	\item In statistics, we rarely work with small amounts of data allowing manual entry
	\item Most often data are contained in readily-available tables
	\item This information needs to be brought inside R so that it is processed and analysed
	\item Analysis of output is also substantial so there is the need to export it quickly and flawlessly to outer information storage
	\item We will demonstrate a number of tools designed for the above-mentioned tasks
\end{itemize}
\end{frame}

\section{Reading and writing text files}
\begin{frame}[fragile]
\frametitle{The \cc{read.table()} function}
\begin{itemize}
	\item We take a real dataset: Airline routes database (from \url{http://openflights.org/data.html})
	\item The file name is \texttt{routes.txt}
	\item \textcolor{red}{Note: You can also read files directly from their url!}
	\item Take a look at the file in a text editor (get a decent one, such as Notepad++, for example, don't count on Notepad)
	\item We will read the data in the following way:
	\begin{lstlisting}[style = rstyle, breaklines]
	routes.df <- read.table("routes.txt", 
		delim = ",", header = FALSE)
	\end{lstlisting}
	\item We used only three (of many) arguments of this function
	\item The command provides a very high level of flexibility
	\item Now read the associated help page
	\begin{lstlisting}[style = rstyle, breaklines]
	?read.table
	\end{lstlisting}
\end{itemize}
\end{frame}

\begin{frame}[fragile]
\frametitle{\cc{read.csv()}, \cc{read.csv2()}, and \cc{read.delim()}}
\begin{itemize}
	\item Wrappers (`shortcuts') for using \cc{read.table()}
	\item \cc{read.csv()} and \cc{read.csv2()}: For cases when data are respectively comma-separated or semi-column-separated
	\item Note that for example MS Excel saves \texttt{*.csv} files using semi-columns instead of commas
	\item \cc{read.delim()}: For cases when data are tab-separated (if you use the command \cc{read.table()} with such data, you have to use the \cc{sep = "\textbackslash t"} option)
	\item (There is also the \cc{read.fwf()} command for cases when  columns in data have fixed width, which we will not discuss as it is rarely used)
\end{itemize}
\end{frame}

\begin{frame}[fragile]
\frametitle{Writing to text files}
\begin{itemize}
	\item Analogical commands are available for writing:
	\begin{itemize}
		\item \cc{write.table()}: for all kinds of text files
		\item \cc{write.csv()} and \cc{write.csv2()}: for csv files
	\end{itemize}
	\item Files are written in the current working directory unless specified otherwise
	\begin{lstlisting}[style = rstyle, breaklines]
	write.table(routes.df, "routes.tsv", sep = "\t")
	\end{lstlisting}
\end{itemize}
\end{frame}

\section{Reading and writing foreign formats}
\begin{frame}[fragile]
\frametitle{The \textbf{foreign} package}
\begin{itemize}
	\item For reading data files in other applications' formats
	\item The package is a system one
	\item We will review only some of the most popular formats, for the remaining see the documentation
	\item First, load the package so that it can be used:
	\begin{lstlisting}[style = rstyle, breaklines]
	library(foreign)
	\end{lstlisting}
	\item Then use the appropriate options to read your file
\end{itemize}
\end{frame}

\begin{frame}[fragile]
\frametitle{The \textbf{foreign} package: Examples}
\begin{itemize}
	\item Read SAS XPORT file on \textcolor{red}{alcohol use} in the U.S. (\url{http://wwwn.cdc.gov/nchs/nhanes/search/DataPage.aspx?Component=Questionnaire&CycleBeginYear=2009}):
	\begin{lstlisting}[style = rstyle, breaklines]
	sas1 <- read.xport("ALQ_F.XPT")
	\end{lstlisting}
	\item (Also check out \texttt{read.ssd()} for reading native SAS datasets)
	\item Read Weka ARFF file (\url{http://storm.cis.fordham.edu/~gweiss/data-mining/datasets.html}):
	\begin{lstlisting}[style = rstyle, breaklines]
	weka1 <- read.arff("cpu.with.vendor.arff")
	\end{lstlisting}
\end{itemize}
\end{frame}

\begin{frame}[fragile]
\frametitle{The \textbf{foreign} package: Examples (2)}
\begin{itemize}
	\item Read Stata file (Wooldridge, 2012, ch. 18):
	\begin{lstlisting}[style = rstyle, breaklines]
	stata1 <- read.dta("phillips.dta")
	\end{lstlisting}
	\item Read SPSS file (\url{http://calcnet.mth.cmich.edu/org/spss/Prjs_DataSets.htm}):
	\begin{lstlisting}[style = rstyle, breaklines]
	spss1 <- read.spss("MathAssess-SpssFormat.sav", to.data.frame = TRUE)
	\end{lstlisting}
\end{itemize}
\end{frame}

\begin{frame}[fragile]
\frametitle{The \textbf{foreign} package: Writing to files}
\begin{itemize}
	\item Options available for some formats only
	\item Examples:
	\begin{lstlisting}[style = rstyle, breaklines]
	write.arff(routes.df, "routes.arff")
	write.dta(routes.df, "routes.dta") # will likely produce an error so
	library(haven)
	write_dta(routes.df, "routes.dta")
	write.foreign(sas1, "stata2.csv", "stata2.do", package = "Stata") # SAS and SPSS are also possible
	\end{lstlisting}
\end{itemize}
\end{frame}

\begin{frame}[fragile]
\frametitle{Reading and writing MS Excel files}
\begin{itemize}
	\item Not built-in in base R
	\item Contributed packages such as \textbf{xlsx} or \textbf{readxl} need to be installed
	\item Another (newer) one that provides good functionality is also \textbf{openxlsx} (we will use it for exercises)
	\item Reading (Wooldridge, 2012):
	\begin{lstlisting}[style = rstyle, breaklines]
	library(openxlsx)
	xlsx1 <- read.xlsx("benefits.xlsx", sheetName = "benefits")
	\end{lstlisting}
	\item or:
	\begin{lstlisting}[style = rstyle, breaklines]
	library(readxl)
	xlsx2 <- read_excel("data/benefits.xlsx", sheet = "benefits", range = "A1:S1849")
	\end{lstlisting}
\end{itemize}
\end{frame}

\begin{frame}[fragile]
	\frametitle{Reading and writing MS Excel files (2)}
	\begin{itemize}
		\item Writing:
		\begin{lstlisting}[style = rstyle, breaklines]
		write.xlsx(sas1, "sas1.xlsx") # With the xlsx package
		library(writexl)
		write_xlsx(sas1, "sas2.xlsx") # With the writexl package
		\end{lstlisting} 
	\end{itemize}
\end{frame}

\begin{frame}[fragile]
\frametitle{Reading EViews wf1 files}
\begin{itemize}
	\item EViews has been quite popular in FEBA and in public and private institutions in Bulgaria
	\item To be able to read wf1 files, it is necessary to install the \textbf{hexView} package
	\item Reading is done with (Wooldridge, 2012):
	\begin{lstlisting}[style = rstyle, breaklines]
	library(hexView)
	eviews1 <- readEViews("consump.wf1")
	\end{lstlisting}
\end{itemize}
\end{frame}

\section{R and Databases}
\begin{frame}[fragile]
\frametitle{R and databases: General remarks}
\begin{itemize}
	\item R can successfully perform some tasks performed by DBMS
	\item However, it is sometimes useful to use the power of relational databases to complement R's capabilities
	\item Natural setup: your data already resides in a relational database and you need to access it and analyse it
	\item Why not do it directly from R?
\end{itemize}
\end{frame}

\begin{frame}[fragile]
\frametitle{Some SQL basics}
\begin{itemize}
	\item (You have a separate SQL course but we need to get ahead of things now)
	\item SQL: Structured Query Language
	\item Not a programming language (like R, Python, C++, etc.)
	\item Many DBMS implementations, both commercial and free
	\item We will use SQLite as the alternative (free) that provides the easiest access for our demonstration purposes
	\item There is a GUI that will be best for our teaching and learning environment: SQLiteStudio
	\item You can download it from \url{https://github.com/pawelsalawa/sqlitestudio/releases/download/3.2.1/SQLiteStudio-3.2.1.zip} and then just unzip it in a folder of your choice
\end{itemize}
\end{frame}

\begin{frame}[fragile]
\frametitle{Some SQL basics (2)}
\begin{itemize}
	\item We will be using the SQLite Sample Database (download from \url{http://www.sqlitetutorial.net/sqlite-sample-database/})
	\item Unzip \texttt{chinook.db} and load it in SQLiteStudio
	\item There, you can inspect the contents of tables
	\item We will execute an SQL query from R to demonstrate how stuff works
	\item There basically two ways to connect to a database: through the \textbf{DBI} package and through the \textbf{RODBC} package
\end{itemize}
\end{frame}

\begin{frame}[fragile]
\frametitle{Using the \textbf{DBI} package}
\begin{itemize}
	\item \textbf{DBI} is decrypted as `database interface'
	\item This interface provides access to many RDBMSs such as MySQL, PostgreSQL, Oracle, etc.
	\item What \textbf{DBI} does is to split the client-server interaction into three parts:
	\begin{itemize}
		\item Driver
		\item Connection
		\item Result
	\end{itemize}
	
\end{itemize}
\end{frame}

\begin{frame}[fragile]
\frametitle{Using the \textbf{DBI} package (2)}
\begin{itemize}
	\item The \underline{driver} serves to make easier the communication between R and a particular RDBMS 
	\item Therefore, a driver for each of the listed RDBMS is available
	\item In our case we will be using the SQLite driver which is provided with the \textbf{RSQLite} package
	\item The \underline{connection} is a wrapper for the actual connection between R and the RDBMS; it is the means through which all queries to and results from the RDMBS are transported
	\item The \underline{result} describes the result of a query or statement (and contains some methods for formatting, printing, summarizing, etc.)
\end{itemize}
\end{frame}

\begin{frame}[fragile]
\frametitle{Back to the example: Connecting to the database}
\begin{itemize}
	\item First load the \textbf{RSQLite} package:
	\begin{lstlisting}[style = rstyle, breaklines]
	library(RSQLite)
	\end{lstlisting}
	\item Note that this automatically loads \textbf{DBI}, too
	\item Next, we load the database driver:
	\begin{lstlisting}[style = rstyle, breaklines]
	drv <- dbDriver("SQLite")
	\end{lstlisting}
	\item Make a character variable to hold the path to the database:
	\begin{lstlisting}[style = rstyle, breaklines]
	dbpath <- "j:/Pcloud/COURSES_SU/MSc/Probabilisitic and Statistical Computations with R/Topic 03/data/"
	\end{lstlisting}
	\item Make the connection:
	\begin{lstlisting}[style = rstyle, breaklines]
	mydb <- dbConnect(drv, dbpath)
	\end{lstlisting}
\end{itemize}
\end{frame}

\begin{frame}[fragile]
\frametitle{Exploring the database}
\begin{itemize}
	\item We can now list the database tables from within R;
	\begin{lstlisting}[style = rstyle, breaklines]
	dbListTables(mydb)
	\end{lstlisting}
	\item Take for example the \texttt{artists} table
	\item We can explore its fields (i.e. names of variables/columns):
	\begin{lstlisting}[style = rstyle, breaklines]
	dbListFields(mydb, "artists")
	\end{lstlisting}
	\item \ldots or read the entire table into an R data frame:
	\begin{lstlisting}[style = rstyle, breaklines]
	artists <- dbReadTable(mydb, "artists")
	\end{lstlisting}
	\item etc.
\end{itemize}
\end{frame}

\begin{frame}[fragile]
\frametitle{Query the database from R}
\begin{itemize}
	\item We issue the query and put the result in a data frame in the following way:
	\begin{lstlisting}[style = rstyle, breaklines]
	db_data <- dbGetQuery(mydb, "SELECT	trackid, tracks.name AS Track, albums.title AS Album, artists.name AS Artist FROM tracks INNER JOIN albums ON albums.albumid = tracks.albumid INNER JOIN artists ON artists.artistid = albums.artistid WHERE artists.artistid = 22;")
	\end{lstlisting}
	\item (Note that the SQL code should not be split across multiple lines as in
	the slide! Here it’s done only for better visibility)
	\item You can check what has been extracted by viewing the data frame
	\item From this point onwards the data is yours to perform statistical analysis on it
\end{itemize}
\end{frame}

\begin{frame}[fragile]
\frametitle{A MySQL example}
\begin{itemize}
	\item We will need for this the \textbf{RMySQL} package so you have to install it
	\item Load it and load the respective driver:
	\begin{lstlisting}[style = rstyle, breaklines]
	library(RMySQL)
	drv2 = dbDriver("MySQL")
	\end{lstlisting}
	\item Make the connection (this time the data is on a web server):
	\begin{lstlisting}[style = rstyle, breaklines]
		mydb2 <- dbConnect(drv2, dbname = "ensembl_compara_51", user="anonymous", password="", host = "ensembldb.ensembl.org")
	\end{lstlisting}
	\item List tables and fields:
	\begin{lstlisting}[style = rstyle, breaklines]
	dbListTables(mydb2)
	dbListFields(mydb2, "genome_db")
	\end{lstlisting}
	\item Read data into a data frame:
	\begin{lstlisting}[style = rstyle, breaklines]
	genome <- dbReadTable(mydb2, "genome_db")		
	\end{lstlisting}
\end{itemize}
\end{frame}

\begin{frame}[fragile]
\frametitle{ODBC}
\begin{itemize}
	\item Decrypted as `Open Database Connectivity'
	\item Developed originally by Microsoft in the beginning of the 90s
	\item Currently available on all major OS (Windows, Linux, Mac)
	\item Aimed to provide an API for accessing DBMS
	\item Provides independence from DBMS through the usage of an ODBC driver
	\item Most DBMS producers provide ODBC connectors
	\item Besides for DBMS, there are ODBC drivers for MS Excel and even for csv files
\end{itemize}
\end{frame}

\begin{frame}[fragile]
\frametitle{Setting up ODBC on Windows}
\begin{itemize}
	\item Type `odbc' in the search box, this appears:
	
	\includegraphics[scale=0.4]{odbc_win1.png}
	
	\item Click on \texttt{Set up ODBC data sources (64-bit)}
\end{itemize}
\end{frame}

\begin{frame}[fragile]
\frametitle{Setting up ODBC on Windows (2)}
\begin{itemize}
	\item A window appears showing the available connectors
	
	\includegraphics[scale=0.4]{odbc_win2.png}
	
	\item There are some more if you click \texttt{Add}
	\item If you miss something, get it online, e.g. \url{http://www.ch-werner.de/sqliteodbc/}
\end{itemize}
\end{frame}

\begin{frame}[fragile]
\frametitle{Create a new DSN}
\begin{itemize}
	\item DSN is `data source name'
	\item In the last window that opened, click \texttt{Add}, select \texttt{SQLite3 ODBC driver}
	\item Name your DSN and point it to your database in what follows:
	
	\includegraphics[scale=0.4]{odbc_win3.png}
	
	\item Click OK
\end{itemize}
\end{frame}

\begin{frame}[fragile]
\frametitle{The \textbf{RODBC} package}
\begin{itemize}
	\item Allows using this common interface through R
	\item Should be installed additionally as it is not a system package
	\item We will try to connect to the \texttt{sampledb} DSN we created
	\item Load the package and make the connection:
	\begin{lstlisting}[style = rstyle, breaklines]
	library(RODBC)
	odbc1 <- odbcConnect("sampledb", believeNRows = FALSE, rows_at_time = 1)
	\end{lstlisting}
	\item The option \texttt{believeNRows = FALSE} checks whether the number of rows returned by the ODBC connection is believable
	
\end{itemize}
\end{frame}

\begin{frame}[fragile]
\frametitle{The \textbf{RODBC} package (2)}
\begin{itemize}
	\item List tables and import a whole table as a data frame:
	\begin{lstlisting}[style = rstyle, breaklines]
	sqlTables(odbc1)
	odbc_data <- sqlFetch(odbc1, "albums")
	\end{lstlisting}
	\item make a query (put data in a data frame again)
	\begin{lstlisting}[style = rstyle, breaklines]
	odbc_data2 <- sqlQuery(odbc1, paste("SELECT	trackid, tracks.name AS Track, albums.title AS Album, artists.name AS Artist FROM tracks INNER JOIN albums ON albums.albumid = tracks.albumid INNER JOIN artists ON artists.artistid = albums.artistid WHERE artists.artistid = 22;"))
	\end{lstlisting}
\end{itemize}
\end{frame}

\begin{frame}[fragile]
\frametitle{An important note on R and SQL}
\begin{itemize}
	\item We discussed only how to retrieve data from DBMS
	\item But the connections allow much more than this
	\item Depending on your SQL user rights, it is possible to do with a database what you would normally be able in the very DBMS
\end{itemize}
\end{frame}

\begin{frame}[fragile]
\frametitle{Further readings}
\begin{itemize}
	\item Packages' documentation
	\item Spector's book (somewhat obsolete but still useful)
	\item Nield, T. (2016): \emph{Getting Started with SQL: A Hands-on Approach for Beginners}, O'Reilly
\end{itemize}
\end{frame}

\end{document}



Spector
p. 20: MLE
p. 30: Combinatorics

http://www.burns-stat.com/r-database-interfaces/